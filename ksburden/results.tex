\section{Results}
\label{sec:results}

Across our simulations we have choosen a liability threshold $q$ of 1\% while we increased heritability from 0.1\% to 1\%.
Simulation of KS, KSburden and burden was done in c++ while for we used the SKAT R-package for SKAT and SKAT-O. %TODO add citation

\subsection{Type-I Error Rates}
\label{sub:type_i_error_rates}

Type-I error rate was accessed under the null for all tests and results are displayed in Table~1. 
%TODO make type 1 error rate

\subsection{Relationship between KS and Burden}
\label{sub:relationship_between_ks_and_burden}

As described in the previous section we aimed to investigate the relationship between the KS and Burden test.
The Figure~\ref{fig:correlation_ks_burden} displays the relationship among the test statistcs of both tests within all tested genes.
As one can see there seems to be no realtionship between the two tests under the null of both tests.
While this simulation cannot exclude the posibility that the two test statistics are independent it indicates that under most common situations the two test statistics can be considered independent.

\begin{figure}[ht!]
  \centering
  \includegraphics[width=0.8\linewidth]{example-image-a}
  \caption{Correlation between test statistics of KS and Burden under the Null acorss selected genes. No obvious correlation pattern can be detected.}\label{fig:correlation_ks_burden}
\end{figure}

\subsection{Power Comparisons}
\label{sub:power_comparisons}

Figure~\ref{fig:simulatedGeneRealData} displays the empirical power for each test under various scenarions.
In each scenario we continously expanded a single causal region by 10\% of the total genes variants until the complete gene was covered by causal mutatations.
We compared the power of the SKAT, SKAT-O, KS, burden (CMC, BURDEN), KSburden tests.
As expected the KS test looses dramatically statistical power in situations if all variants in a given gene are causally related to the phenotype.
In contrast if only a small fraction of the gene is causally relavant the KS tests outperforms both burden tests as well as SKAT\@.
Further, the combined KSburden is less powerful that SKAT-O when 100\% of rare variants are causal but is able to outperform SKAT-O when only 50\% of variants are causal.

\begin{figure}[ht!]
	\centering
  \begin{subfigure}[t]{0.48\textwidth}
    \centering
	\includegraphics[width=0.8\linewidth]{example-image-a}
  \caption{Simulated Power for a causal region of 10\%}
  \end{subfigure}
  \begin{subfigure}[t]{0.48\textwidth}
    \centering
	\includegraphics[width=0.8\linewidth]{example-image-a}
  \caption{Simulated Power for a causal region of 30\%}
  \end{subfigure}
  \begin{subfigure}[t]{0.48\textwidth}
    \centering
	\includegraphics[width=0.8\linewidth]{example-image-a}
  \caption{Simulated Power for a causal region of 70\%}
  \end{subfigure}
  \begin{subfigure}[t]{0.48\textwidth}
    \centering
	\includegraphics[width=0.8\linewidth]{example-image-a}
  \caption{Simulated Power for a causal region of 0.8\%}
  \end{subfigure}
	\caption{Estimated statistical power for four different causal cluster size.\label{fig:simulatedGeneRealData}}
\end{figure}
