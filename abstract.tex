\documentclass[header.tex]{subfiles}
\begin{document}
%%%%%%%%%%%
%  Title  %
%%%%%%%%%%%
\begin{center}
  Abstract of thesis entitled \\
  \vspace*{\baselineskip}
  {\LARGE Application and Development of \\ Quantitative Methods to explore the Genetic Architecture of Human Aggression}\\[0.2\baselineskip]
  \vspace*{\baselineskip}
  Submitted by\\
  \vspace*{\baselineskip}
  {\LARGE Robert Milan Porsch}\\
  \vspace*{\baselineskip}
  for the degree of Doctor of Philosophy \\ at The University of Hong Kong \\ in May 2017
\end{center}

%%%%%%%%%%%%%%%%%%%%%%
%  Abstract Content  %
%%%%%%%%%%%%%%%%%%%%%%
% TODO: Implications of findings <18-05-17, rmporsch> %

Aggression is the delivery of an aversive stimulus form one person to another, with intent to cause harm.
Such behavior has potential beneficial and harmful consequences for the aggressor and can be seen to originate in the evolutionary principles of natural selection.
Thus suggesting that genetic factors have a considerable impact on aggressive behavior.
While previous research has found that about half of the variation of aggression can be explained by genetic factors, sex differences in the genetic expression of such behavior remains unclear.
While the overall contribution of genetic factors on aggression is known, the specific molecular markers associated with this trait remain unknown.
Similar the genetic overlap with other traits, as well as their causal relationships, known to be associated with aggressive behavior, such as risk taking and others, are unexplored.
The aim of this thesis is to explore not only the overall genetic effect on aggressive behavior, as well as the identification of specific molecular markers, but also to elucidate the interrelationships among related traits. 

Within this thesis I have made use of two large data sets to explore the genetic architecture underlying aggressive behavior.
This includes two large longitudinal twin cohorts of over $17,000$ twin pairs as well as the genomes of over $150,000$ unrelated subjects. 
Structural equation models were used to estimate the overall influences of genetic factors on aggressive behavior, as well as to explore potential sex differences.
Furthermore, genome wide genetic variations associated in impulsive aggressive behavior as well as risk taking were explored.
This analysis was complimented with a detailed study of the genetic correlations across multiple traits related to aggressive behavior, such as smoking, neuroticism, alcohol consumption, and risk taking.    
In addition, I explored potential causal relationships between psychiatric disorders and aggressive behavior with a Mendelian randomization. 
At last, I describe and examine a novel method to detect clusters of causal rare genetic variations which might affect impulsive aggression. 

Detailed analysis of the twin sample showed that stability and heritability of aggressive behavioral problems was high.
Genome-wide association study detected genetic variants associated with risk taking, but not with impulsive aggression, while genetic correlations between risk taking and aggression was high.
Further analysis of genetic correlations showed very high estimates between depression and impulsive aggression.
In addition, the applied Mendelian randomization suggested a causal relationship between schizophrenia and both aggression and risk taking.
Rare genetic variants were found not to have an effect on aggressive behavior, but the newly developed rare variant test demonstrated competitive statistical power.

To conclude, these studies have helped to foster our understanding of the underlying genetic mechanisms of aggressive behavior.
Furthermore, it was shown that aggression is complex and genetically correlated with a number of other traits.
Implying that future studies should consider to examine such behavior in conjunction with other related traits.     




\end{document}
