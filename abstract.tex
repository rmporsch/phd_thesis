\documentclass[header.tex]{subfiles}
\begin{document}
%%%%%%%%%%%
%  Title  %
%%%%%%%%%%%
\begin{center}
  Abstract of thesis entitled \\
  \vspace*{\baselineskip}
  {\LARGE Application and Development of \\ Quantitative Methods to explore the Genetic Architecture of Human Aggression}\\[0.2\baselineskip]
  \vspace*{\baselineskip}
  Submitted by\\
  \vspace*{\baselineskip}
  {\LARGE Robert Milan Porsch}\\
  \vspace*{\baselineskip}
  for the degree of Doctor of Philosophy \\ at The University of Hong Kong \\ in May 2017
\end{center}

%%%%%%%%%%%%%%%%%%%%%%
%  Abstract Content  %
%%%%%%%%%%%%%%%%%%%%%%

Aggression is the delivery of an aversive stimulus form one person to another with intent to cause harm.
Such behavior has potential beneficial and harmful consequences for the aggressor and can be seen to originate in the evolutionary principles of natural selection.
Thus suggesting that genetic factors have a considerable impact on aggressive behavior.

Previous research has shown that about half of the variation of aggression can be explained by genetic factors.
However, various aspects of the underlying genetic architecture remain unknown.
Indeed, the presents of genetic sex differences within aggressive behavior are uncertain and specific molecular markers associated with aggression remain unidentified.
Similar the genetic overlap with other related traits, such as risk taking, as well as their causal relationships are unexplored.
The aim of this thesis is to investigate the overall genetic effect on aggressive behavior, the identification of specific molecular markers, as well as to elucidate the interrelationships among related traits. 

Within this thesis I make use of two large data sets to explore the genetic architecture underlying aggressive behavior.
This includes two large longitudinal twin cohorts of over $17,000$ twin pairs as well as the genomes of over $150,000$ unrelated subjects. 
Structural equation models are used to estimate the overall influences of genetic factors on aggressive behavior, as well as to explore potential sex differences.
Furthermore, genome wide genetic variations associated with impulsive aggressive behavior as well as risk taking are investigated.
This analysis is complemented with a detailed study of the genetic correlations across multiple traits related to aggressive behavior, such as smoking, neuroticism, alcohol consumption, and risk taking.    
In addition, I explore potential causal relationships between psychiatric disorders and aggressive behavior with a Mendelian randomization. 
Moreover, I describe and examine a novel method to detect clusters of causal rare genetic variations which might affect impulsive aggression. 

A detailed analysis of the twin sample shows that stability and heritability of aggressive behavioral problems is high.
However, Genome-wide association studies detect genetic variants associated with risk taking, but not with impulsive aggression, while genetic correlations between risk taking and aggression is considerable.
Furthermore analysis of genetic correlations showed high estimates between depression and impulsive aggression.
In addition, a Mendelian randomization suggests a causal relationship between schizophrenia and both aggression and risk taking.
Rare genetic variants seem not to have an effect on aggressive behavior, but the newly developed rare variant test demonstrates competitive statistical power compared to commonly used tests.

To conclude, these studies have helped to foster our understanding of the underlying genetic mechanisms of aggressive behavior.
Furthermore, it was shown that aggression is complex and genetically correlated with a number of other traits.
Implying that future studies should consider to examine such behavior in conjunction with other related traits.     




\end{document}
