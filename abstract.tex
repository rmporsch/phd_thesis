\documentclass[header.tex]{subfiles}
\begin{document}
%%%%%%%%%%%
%  Title  %
%%%%%%%%%%%
\begin{center}
  Abstract of thesis entitled \\
  \vspace*{\baselineskip}
  {\LARGE Application and Development of \\ Quantitative Methods to explore the Genetic Architecture of Human Aggression}\\[0.2\baselineskip]
  \vspace*{\baselineskip}
  Submitted by\\
  \vspace*{\baselineskip}
  {\LARGE Robert Milan Porsch}\\
  \vspace*{\baselineskip}
  for the degree of Doctor of Philosophy \\ at The University of Hong Kong \\ in May 2017
\end{center}

%%%%%%%%%%%%%%%%%%%%%%
%  Abstract Content  %
%%%%%%%%%%%%%%%%%%%%%%

Aggression is the delivery of an aversive stimulus form one person to another with intent to cause harm.
Such behavior has potential beneficial and harmful consequences for the aggressor and can be seen to originate in the evolutionary principles of natural selection.
Thus suggesting that genetic factors have a considerable impact on aggressive behavior.

Indeed, previous research has shown that about half of the variation of aggression can be explained by genetic factors.
However, it remains unknown to which extend these genetic effect differ across sex or overlap with factors affecting other phenotypes.
Furthermore, specific genetic loci associated with aggression remain unidentified.
The aim of this thesis is to investigate the overall genetic effect on aggressive behavior, the identification of specific molecular associations, as well as to elucidate the interrelationships among related traits. 

This thesis makes use of two large data sets to explore the genetic architecture underlying aggressive behavior.
A longitudinal sample of $17,662$ twin pairs is used to explore the overall influence and stability of genetic factors on aggression over age as well as sex differences via Structural Equation Models.
Complementary, a genotyped sample of $152,247$ unrelated samples from the UK Biobank are investigated in order to identify specific genetic loci associated with impulsive aggressive behavior as well as risk taking.
The same sample is also used to study the genetic correlations across multiple traits related to aggressive behavior, such as smoking, neuroticism, alcohol consumption, and risk taking as well as in an Mendelian randomization to infer potential causal relationships between psychiatric disorders and aggressive behavior.
Separately, a novel method to detect clusters of causal rare genetic variations which might affect impulsive aggression is discussed and applied to the UK Biobank. 

Computed twin models show high heritability (between 50\% and 80\%) and demonstrate stability of aggressive behavior over age.
The analysis also suggested significant but small sex differences.
Further the stability of aggressive behavior is mainly driven by longitudinal genetic correlations.
Despite these high heritability estimates no genome-wide significant association was present in the analysed sample in respect to impulsive aggression, but two independent loci were associated with risk taking.

Analysis of the genetic correlations of aggression showed high estimates with risk taking ($r_g=0.44, SE=0.103$), neuroticism ($r_g=0.63, SE=0.083$), as well as depression ($r_g=0.6741$, $SE=0.0919$).
Thus demonstrating considerable genetic overlap between aggression and other traits.
Interestingly, while an applied Mendelian randomization suggests a causal effect of schizophrenia on both aggression and risk taking, a causal relationship between aggression and depression was not supported.

Similar to the analysis of common variants, rare genetic variations seem not to influence aggressive behavior.
However, the newly developed rare variant test demonstrates competitive statistical power compared to commonly used tests.

To conclude, these studies have helped to foster our understanding of the underlying genetic mechanisms of aggressive behavior.
Furthermore, it was shown that aggression is complex and genetically correlated with a number of other traits.
Implying that future studies should consider to examine such behavior in conjunction with other related phenotypes.     

\end{document}
