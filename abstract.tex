\documentclass[header.tex]{subfiles}
\begin{document}
\pagenumbering{gobble}
%%%%%%%%%%%
%  Title  %
%%%%%%%%%%%
\begin{center}
  Abstract of thesis entitled \\
  \vspace*{\baselineskip}
  {\LARGE Application and Development of \\ Quantitative Methods to Explore the Genetic Architecture of Human Aggression}\\[0.2\baselineskip]
  \vspace*{\baselineskip}
  Submitted by\\
  \vspace*{\baselineskip}
  {\LARGE Robert Milan Porsch}\\
  \vspace*{\baselineskip}
  for the degree of Doctor of Philosophy \\ at The University of Hong Kong \\ in August 2017
\end{center}
%\vfill
%\begin{center}
%Temporary Binding for Examination Purpose
%\end{center}

%\clearpage\null\newpage

%%%%%%%%%%%%%%%%%%%%%%
%  Abstract Content  %
%%%%%%%%%%%%%%%%%%%%%%
%\begin{center}
%  {\textbf{Abstract}}
%\end{center}

Aggression is the delivery of an aversive stimulus from one person to another with intent to cause harm.
Such behavior has potential beneficial and harmful consequences for the aggressor and can be seen to originate in the evolutionary principles of natural selection,
suggesting that genetic factors have a considerable impact on aggressive behavior.

Indeed, previous research has shown that about half of the variation of aggression can be explained by genetic factors.
However, it remains unknown to what extent these genetic effects differ across sex or overlap with factors affecting other phenotypes.
Furthermore, specific genetic loci associated with aggression remain unidentified.
The aim of this thesis is to investigate the  genetic architecture of aggressive behavior, including the identification of specific molecular associations, as well as to elucidate the interrelationships among related traits. 

This thesis makes use of two large datasets to explore the genetic architecture underlying aggressive behavior.
A longitudinal sample of twin pairs, measured at three different ages (age 7 ($n=17,662$), age 9/10 ($n=11,585$) and age 12 ($n=12,892$)) is used to explore the overall influence and stability of genetic factors on aggression, and their moderation by  age and sex  via structural equation models.
A genotyped sample of $152,247$ unrelated individuals (age 40 to 69) from the UK Biobank is used to identify specific genetic loci associated with impulsive aggressive behavior as well as risk taking.
The same sample is  used to study the genetic correlations across multiple traits related to aggressive behavior, namely smoking, neuroticism, alcohol consumption, and risk taking, and is used in a Mendelian randomization approach to infer potential causal relationships between psychiatric disorders and aggressive behavior.
In addition, a novel method to detect clusters of causal rare genetic variants which might affect impulsive aggression is discussed and applied to this sample. 

Twin models show high heritability (between 50\% and 80\%) and demonstrate stability of aggressive behavior across ages.
The analysis also suggested significant but small sex differences.
Furthermore, the stability of aggressive behavior is mainly driven by genetic factors.
Despite these high heritability estimates, no genome-wide significant association was present for impulsive aggression, but two independent loci were associated with risk taking.

Analysis of the genetic correlations with aggression showed high estimates for risk taking ($r_g=0.44, SE=0.10$), neuroticism ($r_g=0.63, SE=0.07$), and depression ($r_g=0.7$, $SE=0.08$).
Interestingly, while a Mendelian randomization approach suggests a causal effect of schizophrenia on both aggression and risk taking, a causal relationship between aggression and depression was not supported.

Similar to the analysis of common variants, rare genetic variants seem not to influence aggressive behavior, though the number of rare variants imputed was necessarily limited.
However, the newly developed rare variant test demonstrates competitive statistical power compared to commonly used tests.

To conclude, these studies have helped to foster our understanding of the underlying genetic mechanisms of aggressive behavior.
Furthermore, it was shown that aggression is complex and genetically correlated with a number of other traits,
implying that future studies should consider  such behavior in conjunction with the related phenotypes.     

\end{document}
