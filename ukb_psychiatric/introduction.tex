\section{Introduction}
\label{sec:uk_biobank_psych_introduction}

Impulsive aggressive behavior and risk taking have long been associated with psychiatric disorders, such as Bipolar Disorder (BP), Schizophrenia (SZ) and Major Depressive Disorder (MDD).
Specifically, impulsive aggressive acts are often associated with risky and rapid decision making~\cite{Moeller2001} and are predictive for alcohol abuse disorder~\cite{Courtney2012} as well as poor clinical outcome in schizophrenia, bipolar disorder~\cite{Gut-Fayand2001} and depression~\cite{Dutton2013}.
Indeed, a recent meta analysis of $23,972$ psychiatric inpatients suggested that at least $17\%$ committed one, or more acts of violence during their stay.
This is more evident in clinical centers with higher rate of patients affected with alcohol use disorder and schizophrenia which reported higher rates of violence~\cite{Iozzino2015}. 
In addition, risky decision making has been described in depression~\cite{Wilson2010}, bipolar disorder~\cite{Johnson2012} and schizophrenia~\cite{Cheng2012}.
However, little is known about the genetic overlap between impulsive aggression, risk taking and psychiatric disorders despite the importance of these phenotypes in the clinical representation.

During manic episodes of bipolar disorder patients often have a tendency to work towards a particular goal without proper planing~\cite{Johnson2012}.
This impulsiveness can often result in risky behaviors and the American Psychiatric Association sees an increase in risk taking as one of the diagnostic criteria for manic episodes in bipolar disorder~\cite{APA1994,AmericanPsychiatricAssociation2013}.
Further, in- and outpatients in manic states have shown to be more likely to express violent behaviors~\cite{Ballester2012}.
Further, acts of aggression could potentially be attributed to substance abuse by patients affected with bipolar disorder~\cite{Fazel2010}.
Indeed, substance abuse is a common comorbidity in BP patients and is often associated with a less favorable clinical disease course~\cite{Cassidy2001}.

Similarly, a number of studies have been conducted to investigate impulsivity and risk taking behavior in schizophrenia.
However, in contrast to BP, results in schizophrenia are mixed~\cite{Reddy2014}.
In particular, while self-reported impulsiveness is usually higher in SZ patients, performance-based risk taking and impulsiveness assessment varied~\cite{Ouzir2013}.
In his review~\citet{Ouzir2013} suggested that differences in instruments across studies could potentially explain these discrepancies.
Nevertheless, risky behavior in SZ patients have been explained by cognitive deficits~\cite{Cheng2012} and a faulty mental reward representation~\cite{Heerey2011}.
In addition, impulsiveness has also been attributed to aggression in SZ patients~\cite{Hoptman2015}.
Similar to BP, aggression in SZ has been associated with psychotic symptoms and~\citet{Hoptman2015} identified impulsivity as the main driver for aggressive behavior in SZ\@.

In contrast MDD and aggression is commonly seen as contrary forces which show little resemblance.
Specifically, internalising, blaming negative events on internal factors, is commonly considered a core characteristic of depression while aggression is commonly seen as an externalizing behavior, blaming negative events on external factors~\cite{APA1994,AmericanPsychiatricAssociation2013}.
However, these seemingly opposing behaviors are not mutually exclusive. 
Indeed a common feature of both behaviors is irritability~\cite{Dutton2013}.
Multiple studies have found depression to be an independent risk factor for aggression~\cite{Sher2005,Roland2002,Taft2009, Dutton2013}.
For example,~\citet{Windham2004} showed that maternal depression was a risk factor for sever child abuse and multiple other studies showed that depression was prevalent in spousal homicides~\cite{Stith2004}.
The relationship between depression and risk taking is more mixed.
Most studies investigating the relationship between the two look at adolescent sexual risk taking.
For example,~\citet{Wilson2010} showed that the risk of depression was independently associated with adolescent sexual risky behaviors in both male and female in a sample of $1,120$ male and $1,177$ female students.
Similar results were obtained in other studies looking at sexual behaviors and depression in young adults~\cite{Auerbach2010,Auerbach2007,Othieno2015}.
However, other studies looking at risky behavior which are not related to sexual risky behaviors, such as reckless driving, showed no association between risk taking and depression in adults~\cite{Vassallo2008}, but did so in adolescent~\cite{McDonald2014}. 

While there is clear phenotypical evidence for the connection between impulsive aggression, risky behavior and psychiatric disorders little has been done to investigate the genetic overlap between behavioral manifestations and clinical disorders.
Further, the availability of genetic data in regards to both behavioral and clinical phenotypes allows the use Mendelian randomization to infer potential causal connections.
Specifically, I hypothesise potential causal effects of psychiatric disorders on both risk taking and aggression.
