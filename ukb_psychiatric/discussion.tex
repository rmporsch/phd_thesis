\section{Discussion}
\label{sec:discussion_ukb_psych}

Analysis of the genetic correlations between psychiatric disorders, risk taking, and impulsive aggression, agree with previous phenotypic findings.
However, exceptionally high genetic correlations were found between depressive symptoms and impulsive aggression.
Further analysis, which made use of a Mendelian randomization framework, suggested that these high correlations are unlikely due to causal effects of depression on impulsive aggression.
Nevertheless, the same analysis suggested causal effects of schizophrenia on both risk taking and aggression. 

Overall, these results are in line with findings in both observational and experimental studies on both risk taking and aggression~\cite{Ballester2012,Ouzir2013,Hoptman2015,Sher2005,Roland2002,Taft2009, Dutton2013}.
However, it is important to emphasize the unusually high genetic correlations between impulsive aggression and depression (both MDD and DS).  
These high overall pleiotropic effects, while partly in line with previously described phenotypic effects, and would indicate considerable proportion of shared genetic factors across these two traits.
However, it is unclear if these genetic correlations are due to the same genetic factors influencing both traits (direct pleiotropy), via another phenotype (mediated pleiotropy), or both.
While MR analysis is able to provide some insight into the potential correlation structure by investigating mediated pleiotropic effects the present data does not suggest a causal effect of depression on impulsive aggression.
Nevertheless, it seems unlikely that the high genetic overlap between aggression and depression is exclusively via shared genetic factors,
especially, given the complexity of the phenotype.
It seems more likely that both common genetic factors as well as mediated pleiotropic effects via other variables might have led to this high genetic correlation.
This has potential important clinical implication since it would elevate further the importance of environmental factors during a depressive episode.
Indeed, a study of 96 school children showed that adaptability moderated the relationship between aggression and depression at least partially~\cite{Lee2015a}, indirectly suggesting that environmental stressors might contribute to the observed high genetic correlation. 

However, considerable direct pleiotropic effects might be possible.
The  instrument to measure impulsive aggression does not necessary assess physical aggression, but instead assesses externally expressed aggression as the result of irritability,  commonly observed in patients affected with depressive symptoms~\cite{Dutton2013,Clark1994},
 suggesting that the observed genetic overlap due to commonly shared genetic factors might be considerable.

In contrast to the correlation between impulsive aggression and depression, genetic correlations with risk taking are modest.
Specifically, corrections between SZ and BP suggest moderate genetic overlap with risk taking.
Interestingly, MR analysis indicates that this observed overlap might be due, at least partially, to a causal effect of SZ on risk taking.
Indeed, the selected instrumental variables used to infer causality between SZ and risk taking show little to no pleiotropic effects,
 suggesting that either common genetic factors are of lower effect size in SZ or genetic correlations arise through mediated pleiotropy (direct causal effects of SZ on risk taking). 
This also closely corresponds to previous phenotypic findings which showed that a higher level of risk taking is closely associated with psychotic and manic episodes~\cite{Johnson2012,APA1994,AmericanPsychiatricAssociation2013}.

Interestingly, while genetic correlation analysis between impulsive aggression and schizophrenia did not pass multiple testing, MR analysis suggested causal effects of SZ on impulsive aggression. 
This is in line with previous studies investigating impulsivity in patients affected with psychotic episodes~\cite{Ouzir2013} and has potential important implications in the treatment of such patients.
Specifically, it indicates that SZ does indeed potentially cause impulsive aggressive behavior.
These results are not very surprising since it is long known that psychotic patients are more prone to violence~\cite{Douglas2009} and that these violent behaviors are a consequence of the psychopathological symptoms, such as delusions and hallucinations~\cite{Swanson2006}.
Further, impulsivity is long known to be a common characteristic in SZ patients~\cite{Ouzir2013}.
Thus the results of the MR analysis support the notion that SZ has a causal effect on aggressive behavior and that possible pleiotropic effects between the two traits are small.

However, these results should be viewed in light of a few limitations.
Specifically, the genetic correlations between samples from the UK BioBank and depressive symptoms are overlapping.
This could potentially result in a bias in both correlation and MR estimates.
However, estimations involving the chosen sample of major depression are unaffected by this.
In addition, it is important to emphasize that the  presented MR analysis can only give an indication of causality, but lacks more stringent requirements to suggest robust causal relationships.
Furthermore, the choice of instrumental variables was statistically driven and selected SNPs lacked  clear biological implications thus potentially affecting the validity of SNPs as instrumental variables.

\subsection{Conclusion}
\label{sub:conclusion_psych}

Impulsive aggression and risk taking are not only considerably genetically correlated with  each other, but both phenotypes also show strong correlations with psychiatric disorders as well as with other behavioral phenotypes.
Specifically, large genetic overlaps between impulsive aggression and depressive symptoms were observed.
I showed that this particular relationship is unlikely to be mediated pleiotropy and could represents a true pleiotropic relationship.
In addition, I have shown that schizophrenia is likely to have a causal effect on both risk taking and aggression,
confirming previous observational and experimental studies.

