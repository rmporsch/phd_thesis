\section*{Discussion}
\label{sec:discussion}

We have shown considerable genetic correlations between risk taking, impulsive aggression, psychiatric disorders and behaviors.
Further we provided some evidence for a causal connection between schizophrenia and impulsive aggression as well as risk taking.

We have observed genetic correlations between impulsive aggression and major depression as well as depressive symptoms.
Here it is important to note that the original instrument used to measure the phenotype in question does not consider violent aggression.
Specially, it considers irritability and argumentative, aggressive behavior.
A feature commonly observed in patients affected with depressive symptoms~\cite{Dutton2013}.
Further, analysis of the Mendelian randomization suggests that effect of selected SNPs to impulsive aggression is not exclusively through depression.
Thus, indicating that the observed genetic correlations are not due to mediated pleiotropy, but either display true pleiotropy or are mediated through another, unknown, variable.

Genetic correlations with Schizophrenia and Bipolar disorder did not pass multiple testing and effect sizes are lower than those of depressive symptoms.
Nevertheless, MR suggests some indication that SZ and BP might causally influence impulsive aggression, although the effect of these influences are weak.
This is surprising since epidemiological studies commonly report more violent incidence in patients affected with psychosis than those with depression~\cite{Perroud2011}.
However, this could suggests that environmental stimuli and not genetic predispositions are the predominant factors in influencing aggressive argumentative behavior in psychosis patients.

In addition to correlations with psychiatric disorders, impulsive aggression is also correlated with a number of behavioral traits.
This includes risk taking, childhood IQ, Age of smoking, age of first birth, years of schooling, as well as neuroticism.
Especially large genetic correlations were present in childhood IQ, age of smoking initiation, and neuroticism.
Large genetic correlations with childhood IQ and years of schooling are not surprising.
Already~\citet{Huesmann1987} showed that childhood measurements of IQ were predictive for aggression at later ages.
In addition, considerable genetic correlations were present with neuroticism.
This is also in line with phenotypical findings which showed significant correlations between aggression and neuroticism~\cite{Barlett2012}.
Interestingly,~\citet{Barlett2012} showed that neuroticism was highly correlated with aggressive emotions and attitudes, but showed little correlations with physical violence.
These results further corresponds to the instrumentalization of impulsive aggression within this study which did not assess violent behavior explicitly. 
Interestingly we were also able to observe significant genetic correlations with age of smoking onset as well, but not with smoking status.
This is similar to recent findings showing associations between hostility and onset of smoking, but not with smoking status~\cite{Bernstein2014}
Also high genetic correlations with age of first birth are not surprising given the high genetic correlations between risk taking and impulsive aggression.

Only two psychiatric disorders are significantly correlated with risk taking.
That is bipolar and schizophrenia.
This closely corresponds to previous phenotypical findings suggesting that higher level of risk taking is closely associated with psychotic and manic episodes~\cite{Johnson2012,APA1994,AmericanPsychiatricAssociation2013}.
These medium genetic correlations would suggest that some patients might be more prone to risky behaviors than others.
This is also reflected in the MR analysis which gave some indication that SZ might lead to an increase in risk taking.

Further to there here presented genetic correlations with psychiatric disorders we also replicated genetic overlap between risk taking and sexual behavior.
Specially, we were able to replicated~\citet{Day2016} findings which showed significant genetic correlations between risk taking and age of fist birth as well as number of children.

While we were able to find a number of genetic correlations between risk taking and impulsive aggression with psychiatric disorders, there are a few limitations of this study.
We made use of two dichotomous instruments from the UK BioBank which were not validated psychometric questionnaires.
In addition, independent replication populations were not available and it is unclear if these results will hold up in other populations outside of the UK.
However, given the large number of assessed subjects these limitations do not affect the general conclusion of this study.

\subsection*{Conclusion}
\label{sub:conclusion}

Impulsive aggression and risk taking is not only considerable genetic correlated with it each other, but both phenotypes also show strong correlations with psychiatric disorders as well as with other behavioral phenotypes.
Specifically, strong correlations are present between impulsive aggression and depressive symptoms.
We showed that this particular relationship is unlikely to be mediated pleiotropy, but could represents a true pleiotropic relationship.
