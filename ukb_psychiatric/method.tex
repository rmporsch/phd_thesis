\section{Method}
\label{sec:method_psych}

\subsection{Data}
\label{sub:data}

\subsubsection{UK Biobank}
\label{ssub:uk_biobank_psych}

The UK Biobank~\cite{Allen2014} has been described in the previous chapter in detail (see~\ref{sub:uk_biobank}).
It is a population-based sample of people age 40--69 years old. 
Participants were recruited between 2006 and 2010 via mail invitation (response rate $5.47\%$)~\cite{Sudlow2015} and were assessed using self-report questionnaires.
As in the previous chapter, risk taking and impulsive aggression was chosen as the main phenotype of interest.

\subsubsection{Psychiatric Data}
\label{ssub:psychiatric_data}

Summary statistics of GWAS investigating MDD, SZ, BP, as well symptoms of depression, were used to compute genetic correlations with risk taking and impulsive aggression.
Further, these samples were also utilized to infer potential causal effects of psychiatric disorders on risk taking and aggression via a Mendelian randomization.

Summary statistics for SZ were acquired from a study by~\citet{Ripke2014} which assessed 36,989 cases and 113,075 controls.
The authors identified 108 genome-wide significant loci and was the largest study on schizophrenia to date.
In regards to bipolar disorder, I made use of a study by the \citet{PsychiatricGWASConsortiumBipolarDisorderWorkingGroup2011} which used 11,974 cases and 51,792 controls.
The study identified two genome-wide significant signals in \textit{CACNA1C} and \textit{ODZ24}.

Depression was assessed both via diagnosis of major depressive disorder (MDD) and symptom count. 
The summary statistics of a mega-analysis of MDD from 9,240 cases and 9519 controls were used to infer MDD's causal connection with risk taking and impulsive aggression~\cite{MajorDepressiveDisorderWorkingGroupofthePsychiatricGWASConsortium2013}.
The authors were unable to identify a genome-wide significant signal, likely due to the small sample size.
In contrast, summary statistics of a population based GWAS on depressive symptoms on $161,460$ samples~\cite{Okbay2016} found two independent loci reaching genome-wide significance.  

\subsection{Genetic Analysis}
\label{sub:genetic_analysis}

The same genomic analysis was applied as in Chapter~\ref{cha:assocation_study_in_agggressive_behavior_and_risk_taking}.
This study made use of the imputed genetic data of the UK Biobank, compromising $\sim73$ million SNPs, short indels, and larger structural variations of $152,249$ subjects.
Individuals were genotyped on two different custom genotype arrays, namely the UK BiLEVE as well as the UK Biobank Axiom array from Affymetrix. 
UK BiLEVE was used for most samples ($102,325$), while the array chip from Affymetrix, which was optimised to support genotype imputation, was used for the remaining participants. 
Quality control and imputation was performed by the UK Biobank~\cite{Marchini2015}.
However, I further conducted additional quality control.
Variants whose missing call rate exceeded 10\% as well as those with minor allele frequency below 1\% were excluded.
Further variants which did not pass a Hardy-Weinberg equilibrium test threshold of $1\times10^{-9}$ were also removed from the analysis.
Thus leaving $8,802,909$ genetic variants for further analysis.

In addition I only included participants with `white European' ancestry identified by a $k$-mean cluster of the first four principal components of genotype data.
The total number of participants remaining after quality control of genotype are displayed in Table~\ref{tab:descriptive_gwas}.

Association analysis of autosomal SNPs was performed with Plink~\cite{Purcell2007,Chang2015} with age, sex, genotype array chip, and the first 10 principal components as covariants.

\subsection{Genetic Correlation}
\label{sub:genetic_correlation}

Summary statistics of above-described psychiatric studies were obtained from LD-Hub~\cite{ZHENG2016} and test statistics for aggression and risk raking, computed in the previous chapter, were used in order to estimate genetic correlations across phenotypes using LD-score regression~\cite{Bulik-Sullivan2015a}.
Multiple testing was addressed with the more stringent Bonferroni correction.

\subsection{Mendelian Randomization}
\label{sub:Selection_of_Intstuments}

\acrfull{mr} allows  inferring potential causal effects from observational data in the presence of confounding factors. 
It allows us to assess whether a specific risk factor (exposure) has a causal effect on a disease (outcome).
MR makes use of measured variation in genetic variants with known association to a modifiable exposure, also called instrument (see Section~\ref{sec:joint_association_study}).
In the absence of specific biological knowledge of individual SNPs, the choice of instrumental variables for a Mendelian randomization  is primarily statistically-motivated.
In this case, assumptions of MR are only assessed post-hoc and one cannot speak of a \textit{true} Mendelian randomization~\cite{Burgess2016a}.
Nevertheless, these statistically-driven MR, also called `joint association study'~\cite{Burgess2016a}, can provide suggestive evidence for causal effects.

Here I used a liberal approach to investigate any causal relationship between psychiatric disorders and impulsive aggression as well as risk taking.
Summary statistics were obtained from 4 different GWAS covering schizophrenia~\cite{Ripke2014}, bipolar disorder~\cite{PsychiatricGWASConsortiumBipolarDisorderWorkingGroup2011}, major depressive disorder~\cite{MajorDepressiveDisorderWorkingGroupofthePsychiatricGWASConsortium2013} as well as depressive symptoms~\cite{Okbay2016}.
Only LD pruned variants (no pairwise LD $r^2>0.01$ in a 500kb region) were selected as instruments with $p\leq 5\times 10^{-5}$.
Variants were then harmonized with summary statistics computed from the UK Biobank on risk taking and impulsive aggression. 
MR analysis was performed with MR-Base~\cite{Hemani2016}.

Here it is important to note that selected variants of psychiatric disorders might be primary associations (directly causing the trait), or are associations with some latent trait that influence aggression/risk taking as well as psychiatric disorders.
While direct associations are in line with the underlying assumptions of MR, possible associations with one or more confounding factor would violate these assumptions. 

Similar the direction of effect between the exposure and outcome can be tested with the help of the Steiger test~\cite{Steiger1980}.
In general, the test examines if the variance explained by the instrument SNPs in the outcome is less than that within the exposure. 
However, the test might not be reliable if either outcome or exposure have been poorly measured.
Thus SNPs might simply explain little variance due to a low quality instrument. 
Hence interpretation of the Steiger test should be done with care and in respect to potential limitations.
