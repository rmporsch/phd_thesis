\section{Method}
\label{sec:method}

\subsection{Data}
\label{sub:data}

\subsubsection{UK BioBank}
\label{ssub:uk_biobank_psych}
The UK BioBank~\cite{Allen2014} has been described in the previous chapter (see~\ref{sub:uk_biobank}).
It is a population based sample of people age 40--69 years old. 
Participants were recruited between 2006 and 2010 via mail invitation (response rate $5.47\%$)~\cite{Sudlow2015} and were assessed with the help of self-report questionaires as well as other anthropometirc assessments.

\subsubsection{Psychiatric data}
\label{ssub:psychiatric_data}

Summary statistics of 4 large genome wide association studies were used to compute genetic correlations as well as estiamted causal effects via Mendelian randomization.
Care was taken to exclude potential overlap between samples of the UK BioBank and controls used in these studies in order to minimize potential biases.

Summary statistics for Schizophrenia was taken from a study by~\citet{Ripke2014} which assesed 36,989 cases and 113,075 controls.
The authors identified 108 genome wide significant loci and is the largest study on Schizophrenia to date.
In regards to Bipolar disorder, I made use of a study by the~\citet{PsychiatricGWASConsortiumBipolarDisorderWorkingGroup2011} which used 11,974 cases and 51,792 controls.
This helped to identify two genome wide significant signals in \textit{CACNA1C} and \textit{ODZ24}.

Depression was assesed in two seperate, but related phenotypes. 
Namely Major Depressive Disorder as well as depressive symptomes.
The summary statistics of a mega analysis of major depressive disorder was used to infer its causal connection with risk taking and impulsive aggression~\cite{MajorDepressiveDisorderWorkingGroupofthePsychiatricGWASConsortium2013}.
The study is with 9,240 cases and 9519 controls the smallest. 
The authors were unable to identify a genome wide significant signal.
In contrast, summary statistics of a GWAS on depressive symptoms is with $161,460$ the largers used study~\cite{Okbay2016} and two independent loci reached genome wide significe.  
\subsection{Genetic analysis}
\label{sub:genetic_analysis}
The same genomice analysis was applied as in Chapter~\ref{chap:ukb_assoc}.
The study made use of the imputed genetic data of the UK BioBank, compromising around $\sim73$ million SNPs, short indels and larger structural variations of $152,249$ subjects.
Individuals were genotyped on two different custom genotype arrays, namely the UL BiLEVE as well as the UK Biobank Axiom array from Affymetrix. 
UL BiLEVE was used for most samples ($\sim100,000$), while the array chip from Affymetrix, which was optimsied to support genotype imputation, was used for the remaining participants. 
Quality control and imputation was performed central with the UK BioBank.
However, further additional quality control was conducted.
Variants which whose missing call rate did exceed 10\% as well as those with minor allele frequency below 1\% were exlcluded.
Further variants which did not pass a Hardy-Weinberg equilibrium test threshold of $1e-9$ were also removed from the analysis.
In addition participants with `white European' ancestry identified by a k-mean cluser of the first four principle components of genotype data were included.
The total number of participants remaining after quality control of geno- and phenotype are displyed in Table~\ref{tab:descriptive_gwas}.

Assocation analysis of autosomal SNPs was performed with Plink~\cite{Purcell2007,Chang2015} with age, sex, genotype array chip and the first 10 principle components as covariants.

\subsection{Genetic Correlation}
\label{sub:genetic_correlation}

Computation of genetic correlations were done with LD-score (see Secton~\ref{sub:ld_score_regression}).
Estiamted effect sizes from this study were used to computate the genetic correlations between risk taking and impulsive aggression.
Further, LD-hub~\cite{ZHENG2016}, a database containing over 200 different GWAS summary statistics, was used to compute genetic correlations between behavioral phenotypes and psychiatric disorders.
Multiple testing was addressed with the more stringend Bonferroni correction.

\subsection{Mendelian Randomization}
\label{sub:joint_association_study}

\subsubsection{General Methedology}
\label{ssub:General_Methedology}

Mendelian randomization (MR) allows to infer potential causal effects from observational data in the presents of confounding factors. 
It makes use of measured variation in genetic variants with known association to a modifiable exposure to an outcome.
This causal model is described in Figure~\ref{fig:causal}.
Assuming $J$ genotypes of $n=\{1, 2, \ldots , i\}$ subjects were measured ($G_{i1}, G_{i2}, \ldots , G_{iJ}$),
an exposure $X_i$ as well as an outcome $Y_i$.
Further, the potential confounder $U_i$ is unknown. 
The exposure is a function of the genetic variant, condounder and an independent error $\epsilon_i^X$ 
and the effect of each variant $j$ on the exposure is represented by $\gamma_j$.
Thus, the outcome is the the result of the linear function of the genetic variants, the exposure, the condfounder as well as an error term ($\epsilon_i^Y$).
The to estiamte causal effect between exposre and outcome is $\beta$, while $\alpha_j$ represents the undesired direct effect between the genetic variant $j$ to the outcome.

\begin{equation} \label{eq:rm_basic}
  \begin{split}
    X_i &= \sum^J_{j=1} \gamma_jG_{ij} + U_i + \epsilon_i^X \\
    Y_i &= \sum^J_{j=1} \alpha_jG_{ij} + \beta X_i + U_i + \epsilon_i^Y \\
  \end{split}
\end{equation}

\begin{figure}[!h]
  \centering
  \resizebox{0.5\textwidth}{!}{\begin{tikzpicture}
  \node (G) at (0,0) {$G_j$};
  \node (X) at (2,0) {$X$};
  \node (Y) at (4,0) {$Y$};
  \node (U) at (3,1) {$U$};

  \draw[black, ->] (G) -- node[below] {$\gamma_j$}(X) ;
  \draw[black, ->] (X) -- node[above] {$\beta$}(Y);
  \draw[black, ->] (U) -- (X);
  \draw[black, ->] (U) -- (Y);

  \draw[dotted, black, ->] (G) -- (U);
  \path[dotted,black, ->] (G) edge [bend right=30] node[below] {$\alpha_j$} (Y);
\end{tikzpicture}
}
  \caption{Causal Model.
    The model assumes that the instrumental variable $G_j$ influces the outcome $Y$ only via the exposure $X$.
    Hence assuming that $\alpha_j=0$, $\gamma_j\neq0$ and that $G_j$ does not influece $Y$ via a third variable $U$. 
    The dottet lines indicate potential assumptions vilations.
  }\label{fig:causal}
\end{figure}


Should $G_j$ be independent of confounder $U$,
as well as associated with exposure $X$ and independent of outcome $y$ conditional on $X$ and $U$, then variant $j$ is a valid instrument.

The reduced-form equation of~\ref{eq:rm_basic}, relating outcome with $G_j$, can be written as
\begin{equation}
	\begin{split}
		Y_i &= \Gamma_j G_{ij} + \epsilon_{ij}^{'Y} \\
		&= (\alpha_j + \beta\gamma_j)G_{ij} + \epsilon_{ij}^{'Y}
	\end{split}
\end{equation}

One can estimate the causal effect $\beta$ with the help of the Wald method~\cite{Wald1940}
by dividing the effect of variant $j$ on the outcome $\Gamma_j$ by the effect on the exposure $\gamma_j$.
When assumsing in addition that $\alpha=0$ then the causal effect $\beta$ is
\begin{equation} \label{eq:causal_estiamte}
	\beta = \frac{\beta\gamma_j}{\gamma_j}= \frac{\Gamma_j}{\gamma_j}
\end{equation}

Equation~\ref{eq:causal_estiamte} is often extended to multiple variants as a weighted average of multiple ratio estimates across uncorrelationed genetic variants.
\begin{equation} \label{eq:IVW}
  \frac{\sum^J_{j=1} \hat{\gamma}_j^2\sigma_{Yj}^{-2} \hat{\beta}_j}
  {\sum^J_{j=1} \hat{\gamma}_j^2\sigma_{Yj}^{-2}}
\end{equation}
In which $\hat{\beta_j} = \frac{\hat{\Gamma_j}}{\hat{\gamma_j}}$ and often the chosen weight $\sigma_{Yj}$ is the standard error of the outcome on the $jth$ variant.

However, often one cannot completly exclude that $\alpha_j \neq 0$.
In this case the Wald ratio estimate of variant $j$ will equal the true causal effect plus the error $\frac{\alpha_j}{\gamma_j}$. 
Equation~\ref{eq:TSLS} is then re-writen as
\begin{equation} \label{eq:TSLSbias}
  \beta + \frac{\sum^J_{j=1} \gamma_j^2\sigma_{Y_j}^{-2} \alpha_j}
  {\sum^J_{j=1} \gamma_j^2\sigma_{Y_j}^{-2}} = \beta + Bias(\alpha, \gamma)
\end{equation}
Importantly this implies that the assumption of the independency of the genetic variants with outcome $y$ conditional on $X$ holds if the bias term is equal to zero.

\subsubsection{Used Metheds}
\label{ssub:Used_Metheds}

Wthin this study I used three seperated classes of methods.
That is the inverse variance method (IVW), the median method, as well as MR-Egger regression.
Overall, these methods differ in their robustness to pleiotropy as well as statistical power.
The IVW has been described already in the previous section (see Equation~\ref{eq:IVW}).
The method, while having greater statistical power than other methods, assumes that all used variants are valid instruments.
Thus IVW is especially susceptible to presents of pleiotropy.
In contrast the median method first estimates the causal effect for each variant seperatly. 
Following, estiamtes are then ranked and the median of this distribution is taken to the estimated causal effect between exposre and outcome.
The simple approach has the benefit that if at least 50\% of variants are valid instrument this method will give consitent causal estiamates.
However, this comes with a lose in precision.

At last MR-egger is a new methods which relaxes the assumption of $\alpha_j=0$, instead it assumes that the correlation between $\alpha_j$ and $\gamma_j$ is $0$.
Under this assumption the bias (see Equation~\ref{eq:TSLSbias}) is inversly proportional to $\gamma_j$ and variants with stronger instrument strength (large $\gamma_j$) will on average be closer to the true causal effect.
MR-egger makes use ot this my regessing $\hat{\Gamma_j}$ on $\hat{\gamma_j}$
\begin{equation}\label{eq:egger}
  \hat{\Gamma_j} = \beta_{0E} + \beta_{E} \hat{\gamma_j}
\end{equation}
Interestingly, testing if the intercept $\beta_E$ differs from $0$, or in other words testing if weaker instruments are upward or downward biased, also gives an indication of the overal average pleiotropy.
Thus MR-egger uses a relative relaxed assumption but comes with the cost of a considerable lower statistical power.

\subsubsection{Selection of Intstuments}
\label{ssub:Selection_of_Intstuments}

In the absence of specific biological knowledge of individual SNPs the choice of instrumental variables for a Mendelian Randomization (MR) are primarily statistical motivated.
In this case, assumptions of MR are only assessed post-hoc and one cannot speak of a `true' Mendelian Randomization~\cite{Burgess2016a}.
Nevertheless, these statistical driven MR, also called `joint association study'~\cite{Burgess2016a}, can provide suggestive evidence for causal effects.

Here I used a liberal approach to investigate any causal relationship between psychiatric disorders and impulsive aggression as well as risk taking.
Summary statistics were obtained from 4 different GWAS covering schizophrenia~\cite{Ripke2014}, bipolar disorder~\cite{PsychiatricGWASConsortiumBipolarDisorderWorkingGroup2011}, major depressive disorder~\cite{MajorDepressiveDisorderWorkingGroupofthePsychiatricGWASConsortium2013} as well as depressive symptoms~\cite{Okbay2016}.
Only pruned variants ($r^2=0.01$) were selected as instrumentents with $p\leq 5\times 10^{-5}$.
Variants were then harmonized with summary statistics computed from the UK BioBank on risk taking and impulsive aggression. 

Joint association analysis was performed with MR-Base~\cite{Hemani2016}.
Overall estimated causal effect was judged based on the performance of all applied models.
Further, a sensitivity analysis of each joint association study was performed to investigate vadility of the underlying assumptions~\cite{Burgess2016}.  
This includes assessments of pleiotropic effects via MR-egger as well as Funnel plots.
