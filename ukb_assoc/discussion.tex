\section{Discussion}
\label{sec:ukb_assc_discussion}

The aim of this study was to identify specific genetic variants associated with aggression and risk taking,
as well as estimate the genetic overlap across multiple related phenotypes.
Making use of a large population-based sample I was able to identify two genetic loci associated with risk taking on chromosomes 3 and 6.
Neither the GWAS nor cFDR was able to identify genome-wide significant signals in impulsive aggression.
Nevertheless, considerable genetic correlations were identified with a multitude of related phenotypes, including smoking, alcohol consumption, and neuroticism.

Overall this study was able to replicate the results of previous studies on risk taking~\cite{Day2016} (who used the same data), but given the limited available sample size for impulsive aggressive behavior the null findings for this particular phenotype are not surprising.
Interestingly, cFDR analysis suggests a close genetic relationship between risk taking, alcohol consumption, and smoking.
In particular, smoking was able to aid the identification of at least one additional SNP associated with risk taking.
However, the exact function of this SNP remains unknown and further functional studies are necessary in order to identify potential biological mechanisms associated with this locus.

SNP heritability was with 5\% for both risk taking and aggressive behavior low.
Estimates for risk taking are in line with previous studies~\cite{Day2016} but a comparison with other studies regarding aggression is mixed.
Indeed,~\citet{Pappa2016a} showed considerable fluctuation of heritability estimates across included cohorts ($10-54\%$). 
The authors suggest that differences in genotyping platform as well as sample characteristics might explain the differences between cohorts.
This might also explain the differences between the here presented study and the estimated by~\citet{Pappa2016a}.
Specifically,~\citet{Pappa2016a} made use of validated psychometric instruments to measure parent-rated aggressive behavior in children, in contrast my study made use of a self-reported single item questionnaire in an adult sample.
Furthermore, the here presented study defined aggression as an impulsive act while~\citet{Pappa2016a} used a broader definition.

Nevertheless, as predicted, considerable genetic correlations were identified between risk taking and impulsive aggression.
This high genetic overlap supports the suggestion that these two phenotypes are part of a wider sensation-seeking personality cluster~\cite{Zuckerman2000} which might have been evolutionary useful in exploring new territories, as well as for foraging for food and mates.

However, contrary to a number of phenotypic studies showing a strong causal link between alcohol consumption and aggressive behavior~\cite{FRANZKOWIAX1987,Zuckerman2000,Dakwar2011}, the estimated genetic correlation between these two phenotypes is negative, suggesting that genetic components increasing aggressive behavior might also reduce alcohol consumption.
Similarly, phenotypic correlations between the two traits are rather small.
These results stand in contrast to studies investigating the environmental connection between aggressive behavior and alcohol consumption~\cite{Bushman1990}.
However, there are multiple possible explanations for this discrepancy.
Foremost, previous studies investigating the causal link between alcohol consumption and aggression were randomized controlled trials and used only a few participants.
In addition, most participants were college students.
In contrast, the present investigation made use of a population-based sample of mostly older participants.
This could potentially explain the differences between previous studies and the  present phenotypic results.

It is further surprising that the overall genetic correlation between alcohol consumption and risk taking is relatively low.
This contradicts the considerable inflation within the conditional QQ-plot.
A possible explanation for this discrepancy can be found in the application of these two methodologies.
While cFDR only assesses certain SNPs passing a certain threshold, the genetic correlation takes all SNPs into account, hence accounting for a greater proportion of polygenicity.
An alternative consideration is that the genetic correlation between risk taking and alcohol consumption is not significant, thus suggesting that the overall estimated genetic correlation might be random.
However, this argument stands in contrast to the considerable genetic correlation between impulsive aggression and alcohol consumption, while the cFDR for aggressive behavior displays no indication of  inflation.

Here it is important to note that genetic correlations can arise from a multitude of different factors (as described in Section~\ref{sec:heritability_and_genetic_correlation}).
Assessing the whole genome to estimate a pairwise correlation could potentially be more biased by additional third variables than assessing the overall inflation of SNPs selected under some $p$-value threshold.
Similarly, a recent study by~\citet{Shi2016a} (under review) showed regional genetic correlations among traits which had negligible overall correlations,
thus supporting the argument that the discrepancy between cFDR and genetic correlations is due to the selection of highly significant SNPs.

In addition to the identified genetic overlap of alcohol consumption with risk taking and aggressive behavior, the involvement of smoking is also of relevance.
In particular, two genome-wide significant genetic variants have, next to their association with risk taking, also associations with spirometric measurements of smokers.
This potential genetic overlap is supported by both cFDR and the genetic correlation.
These genetic overlaps are surprising given the very limited phenotypic correlations between the smoking and risk taking as well as aggression.
A possible explanation is that not smoking is influenced by the ability of an individual to  inhibit temptation.
While this effect might not be strong enough to be observed in the phenotypic data, it does manifest itself by a genetic overlap with both risk taking and aggressive behavior,
two phenotypes which have been linked to a sensation-seeking personality cluster~\cite{Zuckerman2000}.
Thus, one could hypothesize that risk taking and aggressive behavior might causally affect smoking status.
Unfortunately, there is currently no experimental data available which could test this hypothesis.

Similarly, analysis of the interrelationship between neuroticism and risk taking as well as aggression suggests some genetic overlap.
Specifically, while the genetic correlation between impulsive aggression and neuroticism is high, the correlation between the personality trait and risk taking is not significant, nor of large effect.
This is in line with the phenotypic correlations observed.
Furthermore, previous studies have shown that neurotic individuals are less likely to engage in risky behavior~\cite{Lauriola2001,InstituteofMedicine2011,Paulus2003}, but subjects with a high level of neuroticism are known to be more likely to act aggressively towards others~\cite{Meesters2007}.
The present genetic evidence supports this phenotypic connection and further suggests that neuroticism and aggressive behavior share substantial genetic factors.

Nevertheless, the  present study has a few limitations.
Foremost, I was unable to detect a genome-wide significant signal in impulsive aggression.
This can be explained by the limited sample size which potentially also affects the analysis of the cFDR\@.
However, the sample size is sufficient to compute genetic correlations across all  phenotypes.

In addition, while the instrument used to measure  neuroticism, alcohol consumption, risk taking, and smoking status has been used in the past, the usage of a single-item to assess aggression is not common.
However, as shown in Chapter~\ref{cha:longHera},  aggression scores often have a skewed distribution and individuals can appropriately be dichotomised into either aggressive or not aggressive,
Thus suggesting that the use of a single-item to measure aggressive behavior may be justified.

\subsection{Conclusion}
\label{sub:conclusion_assoc}

This study was able to replicate previous genome-wide significant signals for risk taking.
In addition, I have observed considerable genetic overlap between risk taking and aggressive behavior, as well as with other traits.
In particular, between aggression and neuroticism, as well as between smoking and risk taking, 
suggesting that these traits might be influenced by common genetic factors.
However, the exact causal connection between these traits remains unknown and further studies, possibly with the use of Mendelian randomization, might be necessary to further understand the interrelationships between these complex human behaviors. 
