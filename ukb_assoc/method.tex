\section{Methods}
\label{sec:methods_assoc}

\subsection{UK Biobank}
\label{sub:uk_biobank}
The UK Biobank~\cite{Allen2014} is a population based study of people aged 40--69 years old. 
Participants were recruited between 2006 and 2010 via mail invitation (response rate $5.47\%$)~\cite{Sudlow2015} and were assessed with the help of self-report questionnaires as well as other anthropometric assessments.
Written informed consent was acquired from all participants and the study was approved by National Research Ethics Service Committee North West---Haydock, UK\@.

\subsection{Phenotype}
\label{sub:phenotype}

Descriptive statistics on the two main phenotypes, risk taking and impulsive aggression,
as well as secondary phenotypes, namely alcohol consumption, smoking status, and neuroticism, are presented in Table~\ref{tab:descriptive_gwas}.  
Number of subjects available for each phenotype varies since not all subjects were assessed on each instrument.
An overview of the number of samples for Caucasian and non-Caucasian participants is also presented.

\subsubsection{Impulsive Aggression}
\label{ssub:impulsive_aggression}
Impulsive aggression was measured with a single item.
Participants were asked to answer the following question:
\begin{displayquote}
  Have you ever had a period of time lasting at least two days when you were so irritable that you found yourself shouting at people or starting fights or arguments?
\end{displayquote}
The phenotype has a relatively high missingness rate within analysed subjects ($67.984\%$) with $40,861$ subjects answering `no' whhile $9,397$ answered `yes'.

\subsubsection{Risk Taking}
\label{ssub:risk_taking}
Risk taking is a single item instrument which subjects answer with either yes or no.
Participants were asked to answer the following question:
\begin{displayquote}
  Would you describe yourself as someone who takes risks?
\end{displayquote}
In contrast to the previous described question, risk taking has a relatively large sample size and low missingness rate ($3.81\%$).
In total $107,011$ subjects answered `yes', while $39,435$ answered `no' (see Table~\ref{tab:descriptive_gwas}).

\subsubsection{Secondary Phenotypes}
\label{ssub:sec_pheno}

Neuroticism was measured using the \acrfull{epq-r-s}~\cite{Eysenck1985}. 
The questionnaires is a 12 item long instrument in which respondents answer either `yes' (scored as 1), or `no' (scored as 0),
resulting in a total neuroticism score between 0 and 12.
Alcohol consumption was measured with a likert-scale.
Participants were asked to indicate their usual alcohol intake from either `Daily or almost daily' (scored as 6), `Three or four times a week' (scored as 5), `Once or twice a week' (scored as 4), `One to three times a month' (scored as 3), `Special occasions only' (scored as 2), or `Never' (scored as 1).
Smoking status was obtained by asking participants if they ever have smoked (scored 1 for `yes' and 0 for `no').


\begin{table}[!htpb]
	\centering
	\resizebox{\textwidth}{!}{%latex.default(samples, title = "", cgroup = c.group, n.cgroup = c(2,     2), table.env = FALSE, file = "./tables/descriptive.tex",     digits = 2)%
\begin{tabular}{llrclr}
\hline\hline
\multicolumn{1}{l}{\bfseries }&\multicolumn{2}{c}{\bfseries All Samples}&\multicolumn{1}{c}{\bfseries }&\multicolumn{2}{c}{\bfseries Caucasian Samples}\tabularnewline
\cline{2-3} \cline{5-6}
\multicolumn{1}{l}{}&\multicolumn{1}{c}{Unaffected/Affected}&\multicolumn{1}{c}{Missingness (in \%)}&\multicolumn{1}{c}{}&\multicolumn{1}{c}{Unaffected/Affected}&\multicolumn{1}{c}{Missingness (in \%)}\tabularnewline
\hline
Risk Taking&107011/39436&$ 3.81$&&86552/29703&$ 3.351$\tabularnewline
Neuroticism&122511&$19.53$&&98086&$18.456$\tabularnewline
Smoking&80984/70678&$ 0.39$&&64115/55853&$ 0.264$\tabularnewline
Impulsive Aggression&40861/9397&$66.99$&&31513/6998&$67.984$\tabularnewline
Drinking&151985&$ 0.17$&&120208&$ 0.065$\tabularnewline
\hline
\end{tabular}
}
  \caption[Sample Size and Missingness of UK Biobank]{
    Sample size and missingness across Caucasians and non-Caucasians participants.
    Missingness indicates the percentage of participants who were not phenotyped for a particular trait.
    The Caucasian sample represents all participants which were used to conduct the genome wide association study.
}\label{tab:descriptive_gwas} 
\end{table}

\subsection{Genetic analysis}
\label{sub:genetic_analysis_assoc}
This study made use of the imputed genetic data of the UK Biobank, compromising $\sim73$ million SNPs, short indels, and larger structural variations of $152,249$ subjects.
Individuals were genotyped on two different custom genotype arrays, namely the UL BiLEVE as well as the UK Biobank Axiom array from Affymetrix. 
UL BiLEVE was used for most samples ($102,325$), while the array chip from Affymetrix, which was optimised to support genotype imputation, was used for the remaining participants. 
Quality control and imputation was performed by the UK Biobank~\cite{Marchini2015}.
However, I further conducted additional quality control.
Variants whose missing call rate exceeded 10\% as well as those with minor allele frequency below 1\% were excluded.
Further variants which did not pass a Hardy-Weinberg equilibrium test threshold of $1\times10^{-9}$ were also removed from the analysis.
Thus leaving $8,802,909$ genetic variants for further analysis.

In addition I only included participants with `white European' ancestry identified by a k-mean cluster of the first four principal components of genotype data.
The total number of participants remaining after quality control of genotype are displayed in Table~\ref{tab:descriptive_gwas}.

Association analysis of autosomal SNPs was performed with Plink~\cite{Purcell2007,Chang2015} with age, sex, genotype array chip, and the first 10 principal components as covariants.

\subsection{Identification of lead SNPs}
\label{sub:Clumping}

Lead SNPs were identified using clumping.
Clumping is the process of grouping SNPs within a given genomic window and level of LD together.
Various thresholds and window sizes can be used.
Within this study the more commonly $250kb$ window region was applied.
SNPs were considered to be within the same LD group when passing an threshold of $r^2 \ge 0.2$.
The SNPs with the lowest p-value within each group was considered the lead SNPs.

\subsection{Conditional False Discovery Rate}
\label{sub:conditional_false_discovery_rate}

Pleiotropy informed \acrfull{cfdr} was used to identify additional genetic loci associated with risk taking or impulsive aggression. 
In principle, cFDR makes use of pleiotropic effect between two phenotypes to increase statistical power.
Within this analysis I computed the empirical cumulative distribution function on nominal p-values of phenotype A conditional on p-values of phenotype B passing a given threshold~\cite{Andreassen2013}.
Further, I applied genomic control to all nominal p-values using the genomic inflation factor $\lambda_{GC}$.
This was done by converting p-values to z-scores, estimating the median z-score for each GWAS, and dividing the squared median z-score by the expected median of a $\chi^2$-distribution  with $df=1$.

Conditional QQ-plots were constructed by plotting adjusted nominal p-values on the y-axis, with the empirical distribution function of p-values on the x-axis.

In addition to conditional QQ plots, I also made use of the conditional FDR (cFDR) which is defined as~\cite{Andreassen2013}
\begin{equation}
  FDR(p_1|p_2)=\pi_0(p_2)p_1/F(p_1|p_2)
\end{equation}
in which $p_1$ and $p_2$ are the p-values for the first and second phenotype.
The function $F(p_1|p_2)$ is the conditional cdf which can be replaced by the conditional empirical distribution function.
Further, $\pi_0(p_2)$ is the assumed conditional proportion of null SNPs for phenotype 1 given that that its p-values are smaller or equal that of the second phenotype.
Within this study I followed a conservative approach by setting $\pi(p_2)=1$~\cite{Andreassen2013}.
A SNP was considered of interest when its conditional FDR $\leq 0.01$.

In addition, SNPs which passed $cFDR\leq 0.01$ were replicated in an independent sample.
This independent sample comprised the non-Caucasian samples excluded in the initial discovery stage.
In order to minimize the effect of population stratification in this diverse sample, the same covariates were included during the computation as in the original discovery GWAS\@.
Quality and potential population stratification was assessed using LD-score regression as well as visual inspection of the QQ plot.
This careful inspection did not detect any major population stratification in the non-Caucasian sample after including PCs as covariates.

\subsection{Genetic Correlation}
\label{sub:genetic_correlation_ukb_assoc}

Computation of genetic correlations was done with LD-score regression (see Section~\ref{sub:ld_score_regression} for further explanations).
Thus, pairwise genetic correlations were computed on the summary statistics of each GWAS\@.
Multiple testing was addressed with the more stringent Bonferroni correction.
