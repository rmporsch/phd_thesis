%! TEX root = /home/robert/Documents/projects/thesis/header.tex

Within this thesis I have analysed the underlying genetic architecture of aggressive behavior in both adults and children.
I have outlined previous studies which investigated genetic factors affecting aggression.
Following I described general methodological approaches in investigating the genetic factors affecting various traits.

My first study investigated the longitudinal heritability of childhood aggression.
I showed that genetic factors which influence aggression are rather stable over age and that quantitative sex differences are minor.
Following, I attempted to investigate specific molecular marker which might influence both impulsive aggression and risk taking behavior.
I was able to identify two independent SNPs associated with risk taking, but failed to detect a genome wide signal for impulsive aggression.
Next, I investigated the relationship of impulsive aggression and risk taking with various psychiatric disorders.
I showed unusual high genetic correlations between impulsive aggression and depression (MDD and DS).
In addition, application of a Mendelian randomization showed that schizophrenia might causally affect both impulsive aggression and risk taking.
Within my last study I investigate the properties of the KS-Burden test.
A test designed to investigate distributional differences of rare variants between affected and unaffected individuals.
The test showed general better performance than other rare variant tests when clusters of causal variants were present.
However, application of the KS-Burden test on impulsive aggression did not result in any significant associations.

Within this chapter I am going to discuss my findings in general.
First I am going to contrast heritability estimates from  my twin study and GWAS\@.
Following, I will discuss the results of the association studies.
At last I will discuss genetic correlations in general, followed by an examination of Mendelian randomizations.

\section{Heritability of Aggression}
\label{sec:heritability_of_aggression}

The missing heritability problem is well known in genetics.
Thus it is not surprising to find that heritability estimates from the twin study and those computed on SNPs differ sharply in regards to aggression.
While I have showed that heritability of aggressive behavior in children is stable across age and ranged from 50 to 80\%, heritability estimates from the conducted GWAS point to 5\%. 

First of all it is important to note that these two heritability estimates are not directly comparable.
Specifically twin estimations were done on children while SNP heritability was derived from mostly middle aged adults.
However, previous heritability estimations in adult twins arrived to similar estimates~\cite{Miles1997a}, hence suggesting that the influence of genetic factors between children and adults might not differ.

The observed differences can also be the result of the differences in used instruments and definition of aggressive behavior.
While both used twin studies made use of a validated psychometric instrument to measure aggressive behavior in children, measurements in the UK BioBank were based on a single question with dichotomous answers choices.
Furthermore, aggressive behavior within the UK BioBank was defined as an impulsive act while the used instruments in the twin studies used a more general definition of aggression.
Hence the used instruments show considerable degree of diversity in both definition and psychometric properties, potentially affecting heritability estimations.

The GWAS by~\citet{Pappa2016a} demonstrated great variability across assessed cohorts in terms of SNP heritability of aggressive behavior in children as well ($10-54\%$).
However, heritability was not assessed within the combined sample and 95\% confidence intervals across cohort estimates are overlapping indicating that differences in heritability should be interpreted with care.
Indeed, only the two smaller cohorts of this particular study show relatively high heritability estimates ($h^2=0.54, N=2,101; h^2=46, N=908$), while the largest cohort displayed similar estimates as the one presented in Chapter~\ref{cha:assocation_study_in_agggressive_behavior_and_risk_taking} ($h^2=0.1, N=5,505$).
Thus suggesting that estimated SNP heritability of impulsive aggression within the UK BioBank is similar to that estimated with the help of validated psychometric instruments. 

However, there are a number of additional potential reasons for the observed discrepancy between twin and GWAS heritability estimates as I already described in Section~\ref{sec:heritability_and_genetic_correlation}.
GWAS rarely capture potential epistatic effects, discount the influence of rare variants, commonly do not assess gene-environment interactions, and are unable to investigated the influence of regulatory components. 
However, to what extend these additional factors influence the size of the observed discrepancy remains unknown.

Interestingly~\citet{Munoz2016a} showed that heritability estimation which were taking shared environmental factors into account were closer to that of estimations based on SNPs.
However, this study only looked at diseases such as Stroke, various forms of Cancer and Diabetes.
Interestingly, the only behavioral related disorder assessed within this study, namely depression, showed the largest missing heritability.
A slightly different suggestion was made by~\citet{Yang2015} who argued that the missing heritability can be explained by taking rare variants into account.
Indeed, in their study rare and common variants were able to fully explain the missing heritability in BMI and height.
While it is unclear to which extend this seems also the case for other phenotypes, most studies to date failed to identify rare variants which were able to account a considerable proportion of phenotyical variability in a complex trait~\cite{Chabris2015,Wray2011}.

Indeed, aggressive behavior, like many other behavioral phenotypes, is rather complex.
This is also demonstrated in the complexity of its definitions (see Section~\ref{sub:forms_of_aggression}).
Furthermore, as shown in Section~\ref{sec:evolutionary_theories}, aggressive behavior has a variety of beneficial and harmful consequences depending on environmental circumstances. 
For example, while aggressive behavior in social situations can be beneficial to gain social status, it can also be highly penalized by others in the group~\cite{Buss1997}.
Another example are the findings by~\citet{Figueredo1995} who showed that aggressive behavior of husbands towards their wife was profoundly affected by the distance or presents of brothers or powerful fathers.
These findings would suggest considerable environmental influence on the expression of aggressive behavior.
However, twin studies commonly report high heritability estimates of aggression.
Hence one can speculate that the remaining heritability might be explainable by considering gene-environmental interactions.
This is in line with previous findings in regards to \textit{MAOA}-environment interactions described in Section~\ref{sub:maoa_interactions}.

However, the here presented studies are unable to determine the origin of the differences in heritability estimates between twin and GWAS\@.
One can speculate that the simplified instrument used with the GWAS, as well as the different operationalization, as well as definition of aggression has considerable contributed to the discrepancy, but other sources might also be possible.

Nevertheless, the here presented studies represent the first extensive investigation of heritability in human aggression.
The use of over $16,000$ twin pairs not only enabled a robust examination of genetic and environmental factors involved in childhood aggression, but also empowered a detailed analysis of potential sex and age differences.
This analysis demonstrated that sex and age only had a minor affect on the estimated heritability.
I further showed that SNP heritability in impulsive aggression is with only 5\% relatively small.
Next I will discuss performed genetic association studies on impulsive aggressive behavior as well as risk taking. 

\section{Genomic Associations in Aggression and Risk Taking}
\label{sec:genomic_associations_in_aggression_and_risk_taking}

\subsection{Common Variants}
\label{sub:common_variants_discussion}

Within Chapter~\ref{cha:assocation_study_in_agggressive_behavior_and_risk_taking} I conducted a GWAS on both risk taking and impulsive aggression.
While I was able to replicated previous findings in regards to risk taking~\cite{Day2016}, I was unable to identify any genome-wide signal within impulsive aggression.

Interestingly, genetic variants within \textit{CADM2} which were shown to influence risk taking are also associated with a number of other personality traits~\cite{Boutwell2017} (under review).
Thus suggesting genetic overlap between risk taking and other personality characteristics.
Indeed, conditional FDR analysis (see Section~\ref{sub:conditional_fdr}) as well as genetic correlations (see Section~\ref{sub:genetic_correlation_ukb_assoc}) between risk taking and neuroticism suggest some overlap.
However, compared to other traits, such as smoking and aggression, these overlaps seem to be small. 

Indeed, it is not clear to which extent as well as how risk taking and smoking behaviors are linked.
Previous research has shown that sensation seeking~\cite{Carton1994} and impulsivity~\cite{Glicksohn2007,Mitchell1999} is more prominent in smokers than non-smokers.
Furthermore, a study by~\citet{Ert2013} explored how smokers and non-smokers differ in regards to risk taking and found that smokers are more easily tempted by high rewards.
Thus suggesting that higher prevalence of risk taking behavior in smokers reflects an impulsive behavior to give into immediate temptations. 
One can therefore speculate that both traits might have shared genetic factors which influence both risk taking as well as smoking.
However, an alternative explanation is that risk taking is causally influencing smoking status.
Suggesting that an intervention which is aimed to reduce risk taking behavior might reduce smoking prevalence. 
Nevertheless, it is unclear how such a general personality trait can be experimental modified to have a long term health effect.

In contrast to risk taking I was unable to identify any genome wide associations for impulsive aggression while I estimated the heritability of aggressive behavior between 50--80\% (see Chapter~\ref{cha:longHera}).
A likely reason for the current failure to identify genome wide significant variants that the study presented in Chapter~\ref{cha:assocation_study_in_agggressive_behavior_and_risk_taking} as well as the study by~\citet{Pappa2016a} have limited statistical power.
While the study by~\cite{Pappa2016a} on aggression in young children and teenagers  has limited sample size ($N=18,988$),
my study, with nearly twice as many subjects, measured aggressive behavior without the help of a validated psychometric instruments and consisted only out of one single `yes' or `no' question. 
In contrast, the study by~\citet{Pappa2016a} made use of well known instruments to assess aggressive behavior.
These factors considerable affect statistical power to detect genome wide associated genetic variants.

Thus an obvious is to increase sample size in order to boost statistical power.
However, it remains difficult to obtain high quality phenotype data while increasing sample size.
The data obtained by~\citet{Pappa2016a}, as well as those used in Chapter~\ref{cha:longHera} are based on multi-question instruments, commonly used with multiple raters.
While the validity of these tools have been shown multiple times~\cite{Goodman1997,Goodman2001,Achenbach2003} it is difficult to see how these can be translated to studies with hundred thousands of subjects.
This opens the need to develop instruments which can accurately measure aggressive behavior in a large samples.  
Indeed, the used instrument to measure impulsive aggression in Chapter~\ref{cha:assocation_study_in_agggressive_behavior_and_risk_taking} has not been validated with existing instruments.
However, it is important to note that the distribution of aggression scores of both CBCL and SDQ (see Figure~\ref{fig:hist_aggression}) is skewed.
Therefore a dichotomous instrument might be able to accurately assess aggressive behavior. 
Indeed, the UK BioBank will increase its current sample of around $150,000$ genotyped subjects to around $500,000$, thus enabling a GWAS with sufficient statistical power to investigate specific molecular markers of impulsive aggression.  

\subsection{Rare Variants}
\label{sub:rare_variants_disccusion}

Rare variants, in respect to impulsive aggression, were analysed in Chapter~\ref{cha:distribuional_differences_of_rare_variants}.
Specifically, I developed a new rare variant association test, called KS-Burden, in order to detect clusters of causal mutations within genomic regions such as genes.
While simulations have shown relative good performance of the KS-Burden compared to other tests, application towards impulsive aggression yield no gnome wide significant findings. 

There are multiple possible reasons for these null findings (see Section~\ref{sec:discussion_ks}), most notably it remains unlikely that rare variants will have a considerable effect of behavioral traits~\cite{Chabris2015}.
In this respect it is important to emphasis that, in contrast to the association study of common variants, the outlined rare variant association study has considerable statistical power~\cite{Lee2011}.
Thus these null findings could suggest that rare variants are unlikely to have a major influence on impulsive aggression.

However, it is important to mention that the rare variant association study in Chapter~\ref{cha:distribuional_differences_of_rare_variants} made use of chip array data, a method which is unable to detect private mutations.
The~\acrfull{exac} has shown in a sample of over 60,706 unrelated individuals that the majority of rare mutations are singletons~\cite{Lek2016,Kobayashi2017}.
Thus underlying the prevalence of these private mutations.
This does not imply that rare mutations might play an important role in common traits, such as aggression~\cite{Chabris2015}.
Hoewever, rare, as well as private mutations could be helpful in explaining extreme expressions of aggressive behavior.
Indeed, a study by~\cite{Peloso2016} recently showed that extreme phenotype selection results in a greater increase in statistical power in rare variants compared to common variants.
Therefore suggesting that future studies might look at highly violent criminal offenders in order to identify genes with higher burden of rare variants.

\section{Phenotypical and Genomic Correlations}
\label{sec:phenotypical_and_genomic_correlations}

As described in Chapter~\ref{cha:introduction}, the genetic overlap between aggression and other behavioral phenotypes is unknown.
In Chapter~\ref{cha:assocation_study_in_agggressive_behavior_and_risk_taking} I have elucidated these previously unknown relationships with behavioral phenotypes, as well as in connection to psychiatric disorders (see Chapter~\ref{cha:psychiatric_corr_mr}).

In particular I have shown considerable genetic correlations between aggression and risk taking, as well as with neuroticism, smoking and depression. 
Especially interesting is the unusual high genetic correlation between aggression and depression.
However, as described in Section~\ref{sec:heritability_and_genetic_correlation}, genetic correlations can arise from a multitude of different scenarios.
Thus it remains unclear if these correlations arise from direct pleiotropic effects, are the cause of causal interactions across variables or are caused by other effects. 

Interestingly, while genetic correlations are able to give some insight into the biological interconnectedness of traits, it is faced with the same problem as other correlation analysis. 
For example, the high genetic correlations between depression and aggression could be caused by underlying but unobserved third, fourth or numerous other phenotypes.
This is a common problem in most, if not all of science.

In the past genetic correlations were only measurable in the presents of the raw genotyping and phenotype data.
However with the recent development of LD-score regression researchers are now able to compute genetic correlations with a number of phenotypes based on summary statistics alone.
This had led to an explosion of genetic correlations~\cite{Bulik-Sullivan2015b,Bulik-Sullivan2015a}.
However, it remains unclear how to interpret these genetic correlations in regards to disease etiology.

There have been some effort to understand genetic correlations in a more local sense.
\citet{Shi2016a} (under review) showed that it is possible to partition genetic correlation across different genetic loci.
The authors showed that the cross trait correlation of 27 traits was mostly driven by only 27 genomic regions. 
Further, a number of neglectable genetic correlations were highly correlated at 7 specific genomic areas.
While these attempts are important in order to foster our understanding of the shared biological mechanisms across different disorders, they also can be useful in order to understand potential causal relationships across traits.
Indeed,~\citet{Shi2016a} attempted to investigate putative causality by making use of genome wide significant SNPs within correlation clusters.
For example, the authors demonstrated that the genome wide correlation between BMI (Body Mass Index) and TG (Triglyceride) can be explained by considering the local correlations around SNPs which were significant in BMI but not TG\@.
In contrast the local genetic correlation of SNPs significant in TG  but not BMI was not significantly different from $0$.
Hence suggesting that BMI might have an causal effect on TG\@.
However, while this method is an interesting approach it does not account for potential hidden variables.
Specifically, shared genetic variants might affect a related or sub-phenotype that might be pleiotropic for both BMI and TG\@.
Furthermore, more complex causal relationships, such as those often found in behavioral traits, might be less straight forward than the given example by~\citet{Shi2016a}.
Hence, while local genetic correlations might help to improve our biological understanding it does only partially improve the interpretation.

Another problem in the analysis of genetic correlations is that most summary statistics were obtained from different samples.
This is commonly seen as something positive, given that sample overlap can lead to a bias in the estimation of genetic correlations when using polygenic risk scores. 
However, LD Score regression accounts for potential sample overlap~\cite{Bulik-Sullivan2015a} and gives an unbiased correlation estimate.  
Nevertheless, the availability of all relevant phenotype data within a single sample allows to adjust for potential confounder already at the GWAS state.
This makes the UK BioBank especially useful and further methodological development might be directed at using classical regression methods to further elucidate causal interrelationships among phenotypes variables.

An additional interesting approach to further investigate genetic correlation structures is the use of Mendelian Randomization.
I have used this popular approach in Chapter~\ref{cha:psychiatric_corr_mr} to demonstrated suggested causal effects of schizophrenia on both risk taking and aggression.
However, there are a number of theoretical and practical issues which might considerable question the validity of MR\@.

\subsection{Mendelian Randomization}
\label{sub:mendelian_randomization_discussion}

MR has been popularised in the recent years, mainly by new developments such as MR-egger and others~\cite{Bowden2015}.
However, it is mainly driven by a desire to understand the causal relationships among variables.

\subsubsection{Mendelian Randomization and Randomized Control Trials}
\label{ssub:causality_and_mendelian_randomization}

MR can be seen as classical randomized control trails in which genetic variants are used to randomly assign subjects to either treatment or control group~\cite{Hingorani2005}.
However, as~\citet{Burgess2016a} pointed out, the aim of randomized control trials and MR are different.
While randomized control trails aim to evaluate the effectiveness of different treatment strategies, the main goal of MR is to assess differences in risk factors between genetically determined subgroups.   
Only at a later stage MR aims to use this information to design non-genetic interventions.

This further implies that the effect of any designed intervention will differ from the genetically determined effects estimated by MR\@.
This is due to the length of the intervention (life long versus short term), the mechanism of intervention, as well as the magnitude of the intervention (genetic effects are usually small)~\cite{Evans2015}. 
Hence,~\citet{Vanderweele2015} questioned the use of  numerical estimates in MR and instead suggested to test only for the presences of a causal effect since magnitude of any intervention targeting the risk factor (exposure) does likely not correspond to the original MR estimate.

Thus the results presented in Chapter~\ref{cha:psychiatric_corr_mr} have to viewed in the context of experimental studies as well.
Indeed, as described in Section~\ref{sec:discussion_ukb_psych}, a number of randomized control trials support the notion of a causal relationship between schizophrenia and impulsive aggression as well as risk taking. 
However, it is unclear if other variables might mediate this relationship~\cite{Evans2015}.

Thus, while MR often have been described in terms of randomized control trials there are some important differences.
However, MR can be useful in mining potential causal connections for future clinical trails~\cite{Evans2015}. 
Furthermore, it provides a tool to understand the genetic correlations among multiple interconnected traits. 
Nevertheless, crucial for the success or failure of an MR is the selection of valid instruments.

\subsubsection{Instrument selection}
\label{ssub:instrument_selection}

A number of studies have been successful in identifying causal relationships between exposures and outcomes.
Most notable a number of studies have shown causal effects in cardiovascular disorders with the help of MR~\cite{Swerdlow,Ference2015,Lieb2013,Voight2012a}.
However, notable these successes were based on a clear biological understanding on how the instrument will affect the exposure in question.
Thus demonstrating that MR can produce valuable and important insight into the causal relationship between variables~\cite{Swerdlow}.

However, this luxury is often not given in many other studies and information about the biological implications of associated variants are unknown.
This scenario is common in behavioral traits, psychiatric disorders as well as other phenotypes.
Thus some MR studies make use of statistical driven methods to chose SNPs as instrumental variables.
However, as~\citet{Burgess2016a} pointed out, studies which only use a statistical driven choice of instrumental variables and lack clear biological information are not true MR\@.
These statistical driven MR might better be called `joint-association study'~\citet{Burgess2016a} in which the underlying assumptions are only investigated post-hoc.

It is important to note that a strong effect size of a given SNP, used as instrumental variable, on the exposure does not sign a strong causal connection between exposure and outcome.
However, a weak association between SNP and exposure can lead to an underpowered MR\@.
Since genetic effects are small this would suggest the usage of multiple genetic variants as valid instruments.
Indeed,~\citet{Burgess2016a} suggested that a specific causal effect of a risk factor on an outcome becomes more plausible if multiple SNPs, which are associated with the risk factor, agree with the effect.
It is important to note that this should include independent SNPs, preferably from different genes and genomic regions.
Nevertheless, it is unreasonable to assume that all genetic variants will be valid fulfill all underlying assumption of an MR\@.
Thus multiple different methods have been developed in order to perform MR in the absence of clear biological information as well as to deal with possible assumption violations.

\subsubsection{Joint-Association studies}
\label{ssub:sensitivity_analysis}

\citet{Burgess2016a} suggested three different approaches to select genetic variants as potential instrumental  variables in the absence of clear biological knowledge.
First, the conservative approach would be to use genetic variants which have a high certainty to fulfill MR criteria.
Thus there might be some indirect, but not conclusive, evidence that selected SNPs fulfill MR assumption criteria.
A more liberal approach would be to select those genetic variants which are associated with the exposure but not the outcome or other measured confounders.
While this approach is unable to address the influence of potential confounders, sensitivity analysis can be used to investigate potential assumption violations, as done in Chapter~\ref{cha:psychiatric_corr_mr}.
The use of this more liberal selection criteria is commonly used in the absence of clear biological information.
At last, all available SNPs withing the whole genome can also be used.
However, while this approach removes the necessity to justify  selection criteria of instrumental variables it lacks theoretical legitimation~\cite{Burgess2016a} and can result in a number of false findings~\cite{Evans2013}.

In Chapter~\ref{cha:psychiatric_corr_mr} I made use of a liberal approach due to the lack of clear biological functional information.
On the basis of three GWAS on depression, schizophrenia, and bipolar disorder I investigated potential causal effects on both risk taking and aggression.
Importantly I chose a liberal p-value threshold for the inclusion of SNPs as instrumental variables.
This was done in the desire to increase statistical power while assuring the robustness of the analysis.
Following, a sensitivity analysis was used to investigate these findings on their robustness.
However, while a sensitivity analysis, as shown in Figure~\ref{fig:sensitivity}, can provide valuable insight into possible assumption violations of instrumental variables.

Specially, funnel plots are commonly used to investigate potential directional pleiotropic effects.
Any asymmetry can be interpreted as a sign of directional pleiotropic effects.
However, the visual inspection of funnel plots can be questionable.
For example, a small number of used SNPs as instrumental variables can make the interpretation as either symmetric or asymmetric an difficult endeavour. 
This issue is also present in the funnel plots of Figure~\ref{fig:sensitivity}.
Specifically, the number of SNPs selected as instrumental variables for depressive symptoms and MDD are limited. 
While increasing the number of SNPs can lead to an increase in statistical power it can also result false positive or false negative findings due to induced pleiotropic effects.
This further demonstrates the importance of selecting valid SNPs as instrumental variables.
Interestingly, nevertheless is that one of the strongest assumptions of funnel plots, which are commonly used in a meta-analysis setting, might not be a problem in MR analysis.
Within an traditional meta-analysis funnel plots can provide an indication for potential publication bias.
A funnel plot is only a valid instrument to assess such a question when the effect size and sample of each used study are uncorrelated~\cite{Evans2013}.
This is often not the case in many meta-analysis.
For example, many studies use different instruments when measuring traits with larger sample sizes or might adjust sample size based on the effect size of previous studies~\cite{Simonsohn}.
In contrast sample size of SNPs is rather constant, therefore funnel plots in MR do not suffer from the same problem.
In addition, MR-egger allows to test for potential pleiotropic effects by inspecting its intercept (see Section~\ref{ssub:Used_Metheds}).
However, it remains difficult to fully test for potential hidden variables. 

Thus~\citet{Vanderweele2015} suggested that negative results might be more plausible when investigating potential causal connections.
In particular, in light of the strong assumptions as well as inability to fully verify those assumptions empirically it might be more beneficial to test assumed causal relationships with the help of MR\@.
However, this would require well powered studies with large sample sizes.
Indeed, the UK BioBank as well as recent other consortia have shown that it is possible to acquire large sample sizes.
While null results are not immune to biases, such bias would need to align perfectly in order to achieve an effect size of zero with a narrow confidence interval when in fact a true effect is present.
An obvious draw back from this approach would be its inability to discover new interventions for various disorders. 

However, it is important to emphasis that MR cannot make conclusive statements about the causal relationship between two variables.
In order to establish a causal relationship a string of multiple evidence is necessary.
MR can only play a part in this and numerous other additional studies are necessary to confirm a causal relationship.
Furthermore, it is important to remember that in contrast to randomized control trial, MR commonly aims to detect small effect sizes.
Hence increasing the difficulty to establish causal relationships. 

To conclude, `joint-analysis' studies are able to identify potential causal relationships without clear biological insight.
While empirical methods exist to test the underlying assumptions only additional evidences by randomized control trials or other form of studies will be able to fully confirm any potential causal relationship.

\section{Conclusion}
\label{sec:conclusion}

Aggressive behavior is a heritable trait which shows strong genetic correlations with a number of other phenotypes, such as smoking and risk taking.
However, it remains unclear on how to interprete genetic correlations in general and what impact these estimates have in explaining the underlying architecture of aggressive behavior.
Mendelian randomizations might be able to elucidate possible causal relationships among traits, application of such methods needs to be done with care while respecting the underlying strong assumptions. 
Nevertheless, a careful application of such techinique has shown that schizophrenia might have a causal effect on both risk taking and impulsive aggression.

Current genome wide association studies lack statistical power to detect specific genetic variants assocated with aggression.
Future studies should aim to increase sample size as well as improve phenotyping quality.
Furthermore, while simulations have shown that the newly developed methods is better able to detect clusters of rare variants, those clusters seem not to be present in impulsive aggression.
