%! TEX root = /home/robert/Documents/projects/thesis/header.tex
\section{Heritability of Aggression}
\label{sec:heritability_of_aggression}

The missing heritability problem is well known in genetics.
Thus it is not surprising to find that heritability estimates from the twin study and those computed on SNPs differ sharply in regards to aggression.
While I have showed that heritability of aggressive behavior in children is stable across age and ranged from 50 to 80\%, heritability estimates from the conducted GWAS point to 5\%. 

First of all it is important to note that these two heritability estimates are not directly comparable.
Specifically twin estimations were done on children while SNP heritability was derived from mostly middle aged adults.
However, previous heritability estimations in adult twins arrived to similar estimates~\cite{Miles1997a}, hence suggesting that the influence of genetic factors between children and adults might not differ.

The observed differences can also be the result of the differences in used instruments and definition of aggressive behavior.
While both used twin studies made use of a validated psychometric instrument to measure aggressive behavior in children, measurements in the UK BioBank were based on a single question with dichotomous answers choices.
Furthermore, aggressive behavior within the UK BioBank was defined as an impulsive act while the used instruments in the twin studies used a more general definition of aggression.
Hence the used instruments show considerable degree of diversity in both definition and psychometric properties, potentially affecting heritability estimations.

There are a number of additional potential reasons for the observed discrepancy as I already described in Section~\ref{sec:heritability}.
GWAS rarely capture potential epistatic effects, discount the influence of rare variants, commonly do not assess gene-environment interactions, and are unable to investigated the influence of regulatory components. 
However, to what extend these additional factors influence the size of the observed discrepancy remains unknown.

Interestingly~\citet{Munoz2016a} showed that heritability estimation which were taking shared environmental factors into account were closer to that of estimations based on SNPs.
However, this study only looked at diseases such as Stroke, various forms of Cancer and Diabetes.
Interestingly, the only behavioral related disorder assessed within this study, namely depression, showed the largest missing heritability.
A slightly different suggestion was made by~\citet{Yang2015} who argued that the missing heritability can be explained by taking rare variants into account.
Indeed, in their study rare and common variants were able to fully explain the missing heritability in BMI and height.
However, it is unclear to which extend this seems also the case for other phenotypes.

In this regards it is important to point out that aggressive behavior, like many other behavioral phenotypes, is rather complex.
This is also demonstrated in the complexity of its definitions (see Section~\ref{sub:definition}).
Further, as shown in Section~\ref{sub:evolutionary}, aggressive behavior has a variety of beneficial and harmful consequences depending on environmental circumstances. 
For example, while aggressive behavior in social situations can be beneficial to gain social status, it can also be highly penalized by others in the group~\cite{Buss1997}.
Another example are the findings by~\citet{Figueredo1995} who showed that aggressive behavior of husbands towards their wife was profoundly affected by the distance or presents of brothers or powerful fathers.
These findings would suggest considerable environmental influence on the expression of aggressive behavior.
However, twin studies commonly report high heritability estimates of aggression.
Hence one can speculate that the remaining heritability might be explainable by considering gene-environmental interactions.
This is in line with previous findings in regards to \textit{MAOA}-environment interactions described in Section~\ref{sub:interactions}.

Nevertheless, the here presented studies represent the first extensive investigation of heritability in human aggression.
The use of over $16,000$ twin pairs not only enabled a robust examination of genetic and environmental factors involved in childhood aggression, but also empowered a detailed analysis of potential sex and age differences.
This analysis demonstrated that sex and age only had a minor affect on the estimated heritability.
I further showed that SNP heritability in impulsive aggression is with only 5\% relatively small.
Next I will discuss performed genetic association studies on impulsive aggressive behavior as well as risk taking. 

\section{Genomic Associations in Aggression and Risk Taking}
\label{sec:genomic_associations_in_aggression_and_risk_taking}

\section{Phenotypical and Genomic Correlations}
\label{sec:phenotypical_and_genomic_correlations}

\subsection{Mendelian Randomization}
\label{sub:mendelian_randomization}

\section{Conclusion}
\label{sec:conclusion}

