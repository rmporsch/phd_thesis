\section{Discussion}

In the current paper we analyzed data from two large longitudinal cohorts with information on childhood aggression in twins to investigate the underlying sources of individual differences and stability of this trait.
A longitudinal twin model of aggression data assessed with the commonly used CBCL and SDQ reveal that genetic factors are the most important factor for individual differences in aggression at each age, but also the major player ones it comes to stability of aggressive behavior throughout childhood.
We observed significant sex differences, mainly regarding the influence of the common environment, as well as significant differences in estimates between the two studies.
However, careful inspection of estimates suggests that these difference between results based on the CBCL (18 items) and the SDQ (5 items) for age specific source of variation as well as phenotypic stability and its underlying sources are rather minor. 

The results generally agree with previous longitudinal studies on aggression.
In this large study we confirm the absence of qualitative sex-differences.
Based on the model fitting results we also conclude that there are significant quantitative sex-differences.
Careful observation of the unstandardized and standardized variance components reveal that for NTR this effect is mainly driven by the presence of shared environmental influences in males at younger ages.
Furthermore, the influences of non-shared environment effects are larger for girls than for boys.
For TEDS the quantitative sex-differences are less pronounced.
Quantitative sex-differences in heritability in our study with large sample sizes were small.
Previous studies, which reported sex-differences in heritability, tended to show a diminishing effect by age, similar to our results.
The absence of qualitative sex-differences, the relative small quantitative sex-differences, and the comparable genetic architecture of aggression throughout childhood based on two different assessment instruments is welcome news for large scale gene-finding studies.
Given that sample size is one of the major factors in these studies, results that the same genes with similar effect size might be of importance for boys and girls, for different age groups, for stability of aggression, and for aggressive behavior based on different assessment instruments enables worldwide collaborative projects for GWA studies. 
From a clinical point of view the stability of aggression and the stable influences of genetic factors from young age onwards indicates that a wait and see policy might not be the best approach to help children and their families who suffer from aggressive problems.
Detection and identification of aggressive problems at young age might help to prevent further suffering.
Common environmental influences shared by children from the same family are significant in boys for in the NTR and TEDS, but solely for females in TEDS.
This is an interesting finding which may point to cultural differences between the Netherlands and the UK, but may also result from the different instruments used in the two studies.
A longitudinal study of US-American twins aged 7 to 12 years old using CBCL demonstrated similar genetic correlations and parameter estimates as the here presented results in NTR~\cite{Haberstick2006}, suggesting that differences between the two studies might be driven by the used instruments.
We need to investigate whether some instruments may be more sensitive to detecting influences of the shared family environment than others and also whether the same instrument behaves differently across cultures.
Further for both NTR and TEDS non-shared environmental influences are stable and age specific.
While this can be partly explained by measurement errors an alternative explanation is that the effect of environmental events on aggressive behavior is temporary and decays over time.
Suggesting constant but heterogeneous environmental effects.
Taken together, these results, in combination with the relative high heritability estimates, plea for studies into gene-environment interplay to inform the development of new treatment strategies	to target aggressive behavior in children.
Some limitations should be taken into account while interpreting the results of this study.  First, the analyses are based on maternal ratings of overall aggression.
Previous research has shown that about 20\% of the variance in maternal ratings during childhood is accounted for by rater bias~\cite{Bartels2007}.
The study by~\citet{Haberstick2006} also demonstrated considerable heritability differences between maternal and teacher ratings (maternal: 76\% to 84\%; teacher: 42\% to 62\%), with higher non-shared environmental influence in teacher ratings, In addition maternal-teacher correlations are low, suggesting situation specific influences.
Furthermore, the focus on overall aggression ignores the differences in genetic architecture that may exist for subtypes of aggression.
For example,~\citet{Ligthart2005} identified two aggression subtypes (relational and direct aggression) within the aggression syndrome scale of the CBCL and report that both were influenced by one underlying set of shared environmental factors, but only partly by the same genes (the genetic correlation was .54 for boys and .43 for girls).
\citet{Ghodsian1987} showed that surveys which measure more subtle forms of aggression (such as teasing and noncompliance) seem to lead to lower heritability estimates than surveys assessing more extreme forms of aggression (such as destructiveness and insult).
Based on two large longitudinal samples of twin data from the UK and the Netherlands we conclude that childhood aggression is a stable trait.
Individual differences at the various ages are mainly accounted for by genetic variation between individuals.
Additionally, genetic influences are also found to be the major source of stability in aggressive behavior throughout childhood.
Based on the large sample size we can furthermore conclude that shared environmental influences are significant, especially for boys.
The picture for girls is less clear since inconsistent findings are observed for the NTR versus the TEDS sample.
