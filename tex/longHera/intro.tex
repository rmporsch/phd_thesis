\section{Introduction}
Aggression, as defined in Definintion~\ref{def:aggression}, is a criterion for disruptive behavior disorders such as oppositional defiant disorder (ODD) and conduct disorder (CD) (American Psychiatric Association, 2013).
Aggressive behavior is also implicated in neurodevelopmental disorders such as attention deficit hyperactivity disorder (ADHD)~\cite{Hamshere2013, Monuteaux2009} and antisocial personality disorder  (ASPD)~\cite{Nouvion2007, Schaeffer2003}.
Both low and high levels of aggression can be detrimental to survival and procreation, and it has been postulated that aggression is under stabilizing selection~\cite{Anholt2012} implying that variation in aggression will show significant heritability (see Section~\ref{sec:evolutionary_theories}).
Heritability estimates for human aggressive behavior indeed tend to be high.
In a meta-analysis by~\citet{Miles1997} of 24 different twin and adoption studies in children and adults, additive genetic effects were found to explain up to 48\% of the variance.
Similar estimates were observed in three related meta-analyses of anti-social behavior of children and young adults~\cite{Ferguson2010, Mason1994, Rhee2002}. 
While genetic influences emerge in most behavioral genetic studies of aggression in humans, their magnitude varies across studies.
Clearly, the heterogeneous nature of the aggression construct adds complexity as well as heterogeneity across age~\cite{Rhee2002}, and sex~\cite{Vierikko2003}.
For age, the meta-analysis by~\citet{Miles1997} reported similar estimates for the contribution of environmental and genetic factors on aggression during childhood and adolescence.
In contrast, two other meta-analyses by~\citet{Ferguson2010} and~\citet{Rhee2002} demonstrated a decrease in standardized additive genetic effects on aggression between young children and adults. 
Sex differences can also play a role.
In general, boys are consistently rated as more aggressive than girls at all ages by their mothers, fathers and teachers~\cite{Hudziak2003}.
However, mean differences do not necessarily imply differing etiologies between sexes.
Sex differences in genetic architecture may be quantitative (e.g.\ different heritabilities in boys and girls) or qualitative when different genes are expressed in the two sexes.
When qualitative sex differences are of importance, one prediction is that the resemblance in siblings or twins of opposite sex is lower than expected based on the resemblance in same-sex siblings or twins.
\citet{Vink2012} compared the correlations in same-sex and opposite-sex dizygotic twin pairs for aggression in 3 to 12 year old twins (N > 30.000 children at 3 years, and N > 6500 at age 12 years).
There was no evidence for qualitative sex differences, as assessed by maternal ratings on the Child Behavior Check List (CBCL) aggression subscale, although sex differences in mean scores were large.
Differences between raters may play a role in qualitative differences.
\citet{Vierikki2003} found lower correlations in opposite-sex compared to same-sex DZ pairs for teacher ratings of aggression, thus suggesting sex-specific variation.  
Some studies, but not all, find quantitative sex differences in the influence of genetic factors on aggression~\cite{Cadoret1995, Eley1999, Rhee2002}.
\citet{Eley1999} were unable to detect sex-differences in the genetic architecture for aggression assessed by the Strengths and Difficulties Questionnaires (SDQ) which was completed by parents in Swedish (aged 7 to 9) and British (aged 8--16) cohorts of 1,022 and 501 twin pairs respectively.
Other studies detected significant quantitative sex differences, but are inconclusive in regard to the direction of the effect~\cite{Silberg1994, Miles1997, Vierikko2003, vanBeijsterveldt2003}.
\citet{Silberg1994}, for example, measured aggression by parental CBCL report in 1,264 twins pairs (aged 8 to 16 years) and found higher heritability estimates in boys, with a diminishing effect during adolescence.
\citet{Vierikko2003} studied 1,651 Finnish twin pairs between the ages of 11 and 12 years, and observed lower heritability estimates in boys.
The discrepancies in these findings could be due to large age ranges and differences in raters, as indicated by the study of \citet{Hudziak2003} who assessed aggression by multiple raters (mothers, fathers, and teachers) in over 6,000 Dutch twins aged of 3, 7 and 10 year old.
Their study demonstrated sex differences in some, but not all raters and age groups.
In addition, differences were diminished when the variance of aggressive behavior on which all raters agreed on was analyzed.
These findings from individual studies echo the results of the meta-analysis by~\citet{Miles1997} which showed only slightly stronger effects of genetic influences in males than in age matched females.
Thus previous research has offered some indication for quantitative sex differences although of relative small effect, while other studies were unable to detect such differences.
However, many studies were underpowered to detect small differences in heritability and the extent of quantitative sex differences and the dependency of such differences as a function of age remain unclear.  
The stability of aggressive behavior in children generally is high.
For example, in a 22-year longitudinal US-American study of over 600 subjects, their parents, and their children aggression was found to be highly consistent from childhood until well into adulthood, and early aggressiveness was predictive of later serious antisocial behavior and self-reported physical aggression~\cite{Huesmann1984}.
A number of behavior genetic studies have investigated the etiology of the stability in aggressive behavior.
\citet{vanBeijsterveldt2003} observed in a longitudinal Dutch sample of 3- to 12-year olds that stability across age intervals ranged from 0.41 to 0.77 and genetic factors accounted for most of this stability.
A genetic longitudinal model suggested a dynamic developmental process consisting of transmission of existing genetic effects interacting with new genetic influences.
The authors identified some modification of genetic influences by age and sex.
At younger ages (3 and 7) heritability for aggression was around 60\% and about the same in boys and girls, but at ages 10 and 12 years, heritability was 67\% for boys, and 50--55\% in girls.
In contrast, for girls shared environmental factors were more important than for boys at 10 and 12 years (29\%; whereas the estimate in boys was 16 to 19\%).
Similar results were found in two studies of overlapping samples of 10,038 British twin pairs between the ages 4 to 16~\cite{Lewis2015, Pingault2015}.
In particular~\citet{Lewis2015} observed that genetic factors were the main source of stability in conduct problem between age 4 and 16, with constant effects of genetic influences throughout childhood (59\% at age 4 to 61\% at age 16).
Similar a study of twins aged 7 to 12 year old found that stability of aggressive behavior was largely due to genetic factors and heritability was estimated between 76\% and 84\%~\cite{Haberstick2006}.
\citet{Pingault2015} demonstrated the stable effect of genetic factors on conduct problems in a latent growth model and suggested that individual differences in the developmental trajectories might be under strong genetic influence.
\citet{Niv2013} during childhood (ages 9--10) and adolescence (ages 14--15) indicated that only part of the genetic factors that influence antisocial behavior (a latent factor that combined aggression and rule-breaking behavior) in adolescence overlaps with the genetic influences in childhood and that part of the genetic influence at ages 14--15 were adolescent specific.
This study also demonstrated overlapping longitudinal genetic influence between aggression and rule-breaking behavior, indicating that a similar set of genes is influencing the developmental dynamics of both constructs.
Thus, while some investigations have categorized subjects into different age groups~\cite{Niv2013} others have focused on developmental trajectory of the phenotype~\cite{Pingault2015} as well as on the influence of new and preexisting genetic factors~\cite{vanBeijsterveldt2003}. 
Here we make use of the ongoing longitudinal data collections in Twins Early Development Study (TEDS;~\cite{Haworth2012}) and Netherlands Twin Register (NTR;~\cite{vanBeijsterveldt2013}) to analyze data on aggressive behavior in children aged 7, 12 years.
The two cohorts assessed parental longitudinal ratings on aggression or conduct problems in very large samples, which enables the investigation of differences between sexes on the age-specific effects of genetic and environmental factors and the estimation of longitudinal genetic and environmental correlations.
Importantly, sample sizes are sufficiently large to assess effects of common environment, shared by children growing up in the same family/household~\cite{Martin1978, Posthuma2000}. 
Aggression was assessed by two commonly used parent-rated instruments for children, the Child Behavior Check List- aggressive problems syndrome scale (CBCL;~\cite{Achenbach2010}) and the Strengths and Difficulties Questionnaires conduct problems scale (SDQ;~\cite{Goodman2001}) for NTR and TEDS respectively. 
These two datasets have been used in previous studies, most prominently~\citet{Lewis2015},~\citet{Pingault2015} and~\citet{vanBeijsterveldt2003}.
However, previous studies on TEDS~\cite{Lewis2015, Pingault2015} did not consider sex differences and the currently analyzed NTR dataset is considerable larger than the one that was used before~\cite{vanBeijsterveldt2003} and thus has higher statistical power to detect differences across age and sex.
Further, this is the first study which investigates these two large scale longitudinal twin cohorts in a coherent framework.
This study will also aid future genome wide association studies (GWAS) by investigating qualitative and quantitative sex differences as well as genetic stability over time.
This indicates to what extent future GWA studies can collapse multiple studies across ages, sex, and instruments.
