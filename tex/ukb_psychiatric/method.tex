\section*{Method}
\label{sec:method}

\subsection*{UK BioBank}
\label{sub:uk_biobank}
The UK BioBank~\cite{Allen2014} is a population based study of people age 40--69 years old. 
Participants were recruited between 2006 and 2010 via mail invitation (response rate $5.47\%$)~\cite{Sudlow2015} and were assessed with the help of self-report questionaires as well as other anthropometirc assessments.
Written informed consent was aquired from all participants and the study was approved by National Research Ethics Service Committee North West–Haydock, UK\@.

\subsection*{Genetic analysis}
\label{sub:genetic_analysis}
This study made use of the imputed genetic data of the UK BioBank, compromising around $\sim73$ million SNPs, short indels and larger structural variations of $152,249$ subjects.
Individuals were genotyped on two different custom genotype arrays, namely the UL BiLEVE as well as the UK Biobank Axiom array from Affymetrix. 
UL BiLEVE was used for most samples ($\sim100,000$), while the array chip from Affymetrix, which was optimsied to support genotype imputation, was used for the remaining participants. 
Quality control and imputation was performed central with the UK BioBank.
However, further additional quality control was conducted.
Variants which whose missing call rate did exceed 10\% as well as those with minor allele frequency below 1\% were exlcluded.
Further variants which did not pass a Hardy-Weinberg equilibrium test threshold of $1e-9$ were also removed from the analysis.
In addition participants with `white European' ancestry identified by a k-mean cluser of the first four principle components of genotype data were included.
The total number of participants remaining after quality control of geno- and phenotype are displyed in Table~\ref{tab:descriptive_gwas}.

Assocation analysis of autosomal SNPs was performed with Plink~\cite{Purcell2007,Chang2015} with age, sex, genotype array chip and the first 10 principle components as covariants.

\subsection*{Genetic Correlation}
\label{sub:genetic_correlation}

Computation of genetic correlations were done with LD-score (see Secton~\ref{sub:ld_score_regression}).
Estiamted effect sizes from this study were used to computate the genetic correlations between risk taking and impulsive aggression.
Further, LD-hub~\cite{ZHENG2016}, a database containing over 200 different GWAS summary statistics, was used to compute genetic correlations between behavioral phenotypes and psychiatric disorders.
Multiple testing was addressed with the more stringend Bonferroni correction.

\subsection*{Mendelian Randomization}
\label{sub:joint_association_study}

In the absence of specific biological knowledge of individual SNPs the choice of instrumental variables for a Mendelian Randomization (MR) are primarily statistical motivated.
In this case, assumptions of MR are only assessed post-hoc and one cannot speak of a `true' Mendelian Randomization~\cite{Burgess2016a}.
Nevertheless, these statistical driven MR, also called `joint association study'~\cite{Burgess2016a}, can provide suggestive evidence for causal effects.

Here we perform a liberal approach to investigate any causal relationship between psychiatric disorders and impulsive aggression as well as risk taking.
Summary statistics were obtained from 4 different GWAS covering schizophrenia~\cite{Ripke2014}, bipolar disorder~\cite{PsychiatricGWASConsortiumBipolarDisorderWorkingGroup2011}, major depressive disorder~\cite{MajorDepressiveDisorderWorkingGroupofthePsychiatricGWASConsortium2013} as well as depressive symptoms~\cite{Okbay2016a}.
Only pruned variants ($r^2=0.001$) were selected as instrumentents with $p\leq 5\times 10^{-5}$.
Variants were then harmonized with summary statistics computed from the UK BioBank on risk taking and impulsive aggression. 

Joint association analysis was performed with MR-Base~\cite{Hemani2016} and multiple different methods were used.
A sensitivity analysis of each joint association study was performed to investigate vadility of the underlying assumptions~\cite{Burgess2016}.  
