% For LaTeX-Box: root = main.tex

\section*{Results}
\label{sec:results}

Genome Wide Association study on impulsive aggression and risk taking was performed.
We were able to replicated previous findings on risk taking and identified a genome wide significant signal in \textit{CADM2}.
%Specifically, \textit{rs13084531} ($p=9.826e-09$) reached genome wide significance in chromosome $3$ (see supplementary material). %TODO needs more specific data on the test stat
Association analysis in impulsive aggression did not yield any genome wide significant findings (see supplementary material).
SNP based heritability was estimated at $0.0552 (SE=0.0052)$ and $0.0532 (SE=0.012)$ for risk taking and impulsive aggression respectively.

Genetic correlations were assessed in two separate categories, that is psychiatric and behavioral phenotypes.
Estimated genetic correlations of risk taking and impulsive aggression were adjusted for multiple testing using Bonferroni.

%%%%%%%%%%%%%%%
% aggression  %
%%%%%%%%%%%%%%%




Genetic correlation between risk taking and aggressive behavior itself was with $0.4363$ ($SE=0.1039,p=2.6656e-05$) relatively high. 
Figure~\ref{fig:corr_agg} shows the genetic correlation between impulsive aggression and other phenotypes.
Substantial correlations are present in both behavioral and psychiatric phenotypes.
Specifically,
Depressive symptoms ($r=0.6741, SE=0.0919, p=\ensuremath{2.2326\times 10^{-13}}$)
as well as
Major depressive disorder ($r=0.6741, SE=0.0919, p=\ensuremath{2.2326\times 10^{-13}}$)
are significantly correlated with impulsive aggression.
Also behavioral traits show considerable genetic correlations.
Age of smoking initiation ($r=\ensuremath{-0.6442}, SE=0.2035, p=0.0016$),
Childhood IQ ($r=\ensuremath{-0.7035}, SE=0.1528, p=\ensuremath{4.1449\times 10^{-6}}$),
Neuroticism ($r=0.6408, SE=0.0817, p=\ensuremath{4.5484\times 10^{-15}}$),
Years of schooling (proxy cognitive performance) ($r=\ensuremath{-0.4928}, SE=0.0801, p=\ensuremath{7.7102\times 10^{-10}}$),
as well as
Age of first birth ($r=\ensuremath{-0.3696}, SE=0.0778, p=\ensuremath{2.0039\times 10^{-6}}$) are significantly correlated.

\begin{knitrout}
\definecolor{shadecolor}{rgb}{0.969, 0.969, 0.969}\color{fgcolor}\begin{figure}
\subfloat[Genetic correlation between impulsive aggression and behavioral phenotypes\label{fig:corr_agg1}]{\includegraphics[width=1\linewidth]{figure/corr_agg-1} }
\subfloat[Genetic Correlation between impusive aggression and psychiatric disorders\label{fig:corr_agg2}]{\includegraphics[width=1\linewidth]{figure/corr_agg-2} }\caption[Genetic correlation of impulsive aggression with a variety of different phenotypes]{Genetic correlation of impulsive aggression with a variety of different phenotypes. Error bars indicate the $95\%$ confidence interval. Trianguar shaped dots indicate that the correlation has passed Bonferoni at $0.05$}\label{fig:corr_agg}
\end{figure}


\end{knitrout}

%%%%%%%%%%%%%%%
% Risk taking %
%%%%%%%%%%%%%%%



Genetic correlations between risk taking and a number of phenotypes as shown in Figure~\ref{fig:corr_risk}.
Significant genetic correlations could be observed in two psychiatric disorders, namely
Bipolar disorder ($r=0.2561, SE=0.0606, p=\ensuremath{2.4034\times 10^{-5}}$)
and
Schizophrenia ($r=0.2561, SE=0.0606, p=\ensuremath{2.4034\times 10^{-5}}$).
Correlations between behavioral phenotypes are smaller but significant correlations can be observed at
Age of first birth ($r=\ensuremath{-0.2509}, SE=0.045, p=\ensuremath{2.4871\times 10^{-8}}$)
and
Number of children ever born ($r=0.283, SE=0.0575, p=\ensuremath{8.608\times 10^{-7}}$).

\begin{knitrout}
\definecolor{shadecolor}{rgb}{0.969, 0.969, 0.969}\color{fgcolor}\begin{figure}
\subfloat[Genetic correlation between risk taking and behavioral phenotypes\label{fig:corr_risk1}]{\includegraphics[width=1\linewidth]{figure/corr_risk-1} }
\subfloat[Genetic Correlation between risk taking and psychiatric disorders\label{fig:corr_risk2}]{\includegraphics[width=1\linewidth]{figure/corr_risk-2} }\caption[Genetic correlation of risk taking with a variety of different phenotypes]{Genetic correlation of risk taking with a variety of different phenotypes. Error bars indicate the $95\%$ confidence interval. Trianguar shaped dots indicate that the correlation has passed Bonferoni at $0.05$}\label{fig:corr_risk}
\end{figure}


\end{knitrout}

Thus there are a number of genetic correlations between impulsive aggression, risk taking and other psychiatric disorders.

