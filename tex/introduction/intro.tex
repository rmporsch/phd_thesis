%&tex
\chapter{Introduction}
\label{cha:introduction}

% Dramatic entry
World War I and II, the Holocaust, as well as more recent terror attacks have demonstrated the extend of which human aggression is capable of causing distruction and harm to an extend previously unknown. 
Thus aggressive behavior remains a continous issue in all human societies and significant reseach effort has been spend to identify causes of human aggression.
This ranges from gun control, global warning, computer games and movies, as well as others to more biological and evolutionary causes. %TODO needs some citation for each
Here I will present major genetic characteristics of aggressive behavior in children and adults.
Ranging from studies on twins to large scale genome wide association studies.

% Intro to the intro
However, first I will present current and past research on aggressive behavior.
Within this literature review I will first present different subtypes of aggresssion.
Following I will describe commonly associated enviormental factors contributing to aggressive behavior.
This is succeeded by an introduction into the evolutionary theories of aggression.
I will then describe to which extend genetic factors influence aggressive behavior followed by a sumamry of recent discoveries in molecular genetics.
I will conclude this section by a short describtion of the here presented studies.

\section{Definition of Aggression}
\label{sec:overview_of_reseach_in_aggression}

One can define human aggression as a behavior intended to cause physical or emotional harm to others \cite{Anderson2002}.
This definition is relativly broad and does also include behavior in which the target does not intend to avoid the aggressive behavior, such as in sexual masochism.
In addition to the intend to cause harm the preptrator must also believe that the target wants to avoid the behavior\cite{Berkowitz1993,Baumeister1989,Baron2007,Geen2001} (e.i. to avoid being harmed).  
It is also important to further distinguish between \textit{Hostile Aggression} and \textit{Instrumental Aggresssion}.
While the former is characerised as impulsive, thoughtless and unplanned behavior, \textit{Instrumental Aggression} is defined as a planed and proactive behavior to obtain a certain higher goal \cite{Berkowitz1993,Geen2001}.
Both forms of aggression can be either physical or verbal.
While physical aggression in humans is homologous to other animals, verbal aggression, also called indirect, relational and social aggression, is relativly distinct to humans \cite{Archer2005}.
These verbal behaviors cause harm to others by gossiping, spreading rumors, or excluding other from social groups.
While the terms \textit{indirect}, \textit{relational}, and \textit{social} aggression have been differently conceptulized in the past \cite{Archer2001}, they are expressed in common behaviors, can be contrasted to physical aggression to a similar extend as well as have been shown to demonstrate the similar differences between the two sexes \cite{Archer2004}.
Hence the terms are more similar than distinct and I will therefore proceed to call all them \textit{indirect} aggression in order to distiguish it more from the physical, more direct aggressive behavior\cite{Archer2005}.

Thus I define aggression as an behavior by an individual who intends to either direct or indirect cause harm to other, while holding the believe that the target of such behavior aims to avoid such behavior.

Historically, research of aggression has been devided into nurture versus nature\cite{Archer2009}. 
Proponents of the nurture side have argued that aggressive behavior is caused by enviormental influences, while supports of the nature side of the discussion supported the idea that only differences in the genetic architecture are able to explain the differences in aggressive behavior.
Today's view is less polarized and acknoweghes that both, nature and nurture play a crucial part in aggression.
While not disregarding the enviormental aspect of aggression completly I will mostly focus on the genetic and biological causes of aggressive behavior.
In the following section I will outline evolutionary concepts helpful in explaining individual differences of aggression. 

\section{Evolutionary Theories of Aggressive Behavior}
\label{sec:evolutionary_theories_on_aggressive_behavior}

Paleontological findings of broken bones, rips and smashed skulls, unexplainable without the consideration of weaponary force, and occasional findings of weapon fragments in skeletal rib cages all suggest that violence and aggression has been part of the human evolutionary history. 
~\cite{Buss1997}, one of the founder of Evolutionary Psychology, suggested that all psychological mechanisms and behavior originate in the evolutionary priciple of selection.  
These mechanisms are aimed to solve adaptive problems.
Hence some variants might solve certain problems better than others which will result in better fitness.
Resulting in the preservation, replication and spreading throught a population \cite{Buss1997}.
Hence from a evolutionary perspective every human behavior can be seen as a solution to a specific adaptive problem.

~\cite{Buss1997} suggested seven adaptive problems to which aggressive behavior might be an evolutionary solution.
This includes \textit{Co-Opt the Resources of Others}, which can be defined as the use of physical or psycholigcal force to obtain resources hold by another indivudual or group.
These resources could mean food, water, land or secual partners.
Indeed aggressive behavior in children is often about resouces, such as toys \cite{Campbell1995} suggesting that this behavioral adaptations is an old evelutionary strategy.
However, forceful taking ownership of resources carries a significant amount of risk to be hurt or even killed.
In particular aggression can also be  useful in \textit{defending against an attack}.
Attacking aggressors are a serious threat to valuable resources.
Aggression can be an effective strategy in defending against indivuduals or groups.
Further, it can be also an adaptive strategy to foster a reputation that would deter potential aggressors \cite{Buss1997}.

Another evolutionary benefit of aggression can be found in the  social hierarchies in groups.
For example, men who win fights and defeat opponents gain power and status in many societies \cite{Hill1996}.
The gain in social status can be beneficial in accessing resources and mates \cite{Archer2009}.
However, aggression can also result in an decline in status.
~\cite{Buss1997} suggested for example that a physical conflict between two professors in a facutly meeting would result in an decline in repuatation.
Thus display of aggression is not acceptable in all social situations.

Further aggressive behavior sometimes plays a crucial part in mating.
Aggression towards same-sex individuals is sometimes aimed to reduce their social status and therefore make them less attractive to the other sex \cite{Buss1990}.
Hence inflicting damage on a rival directly translates to an incresed benefit to the aggresssor.
In addition to aggression towards same-sex individuals, aggression is also prevelant towards the oposite sex.
Thus aggression can be used to deter a long-term mate from infidelity \cite{Daly1982}.
However, also here aggression can have negative consequences in form of retahilation.
For example, a husband might be reluctant to use aggression towards his wive when she is living close to a number of brothers and a pwerful father.
Indeed, a study in Madrid, Spain found that women who had a higher density of genetic kin in and around Madrid were less likely to be victim of domestic violence \cite{Figueredo1995}.

Hence aggression only brings an evolutionary advantage when the benefits outweight the potential costs.
In addition, nearly all mammals display sex differences in the expression of aggression.
Males are more likely to exhibit physical aggression than female
In fact sexual selection theory (SST) has been used to repeatedly to explain the observed greater physical aggression in males in nearly all mammals	 \cite{Archer2004,Anderson2002}. 
Sexual selection is concerned how a member of one sex choses another individual from the other sex, as well as the competion between members of the sex over access to the other \cite{Darwin1859}.
In most mammals the more competitive sex is the male \cite{Archer2009}. 
~\cite{Trivers1972} suggested that these sex differences can be explained by the commonly observed reduced parental investment by males.
Parental investment is the amount of resources a parent investes into his or her off spring to increase its survivial and reproduction \cite{Archer2009}.
The theory was first suggested by~\cite{0198504403} and suggests that a male can minimize his parental investment in favor in producing a higher amount of offspring.
In contrast, many female mammals have an large obligatory patrenatal investment, such as gestation and delivery.
Thus female partner selection is more careful compared to male.
Hence the female selection of good reproductive fitness is directly linked to offset the lack of male paternal investment.
Indeed, hirarical order in social groups is often established by means of aggressive behavior which enables high ranked individuals prioriy access to food and mating paterns\cite{Lindenfors2011}. 
Thus suggesting that aggressive behavior is under sexual selection.


One of the most crucial differences between male and female behavior is the extend to which an individual is willing to take during a conflict %TODO cite.



Despite these advatages, increased levels of aggressive behavior can also be harmful to the individual.
Expressing aggression towards another individual sheers energy away from other acitivies, such as hunting and foraging, and also carries a high risk of injury and death \cite{Packer1995}.  
Hence previous reseach on multiple different species suggested that aggressive behavior is under stabilizing selection.
Further this also shows strong evidence for a large role of genetic factors in the etiology of aggressive behavior. 


In addition one can also see aggressive behavior as part of a wider phenotypical complex. 

\begin{figure}
	\centering
	\scalebox{0.8}{\tikzstyle{box} = [rectangle, rounded corners, minimum width=3cm, minimum height=1cm,text centered, draw=black]
\begin{tikzpicture}
	\draw[line width=6, gray] (-7,0)-- (0,0) --(7,0);
	\draw[gray, fill=gray] (-1,-2)-- (0,0) --(1,-2) --(-1,-2);

	%Cost
	\node (injury) [box, fill=red!80] at (-5,0.6)  {Reproductive Cost};
	\node (grooming) [box, fill=red!70, above of=injury] {Foraging};
	\node (repo) [box, fill=red!60, above of=grooming] {Parental care};
	\node (foraging) [box, fill=red!50, above of=repo] {Grooming};
	\node (parCare) [box, fill=red!40, above of=foraging] {Injury risk};

	%Benefit
	\node (mat) [box, fill=green!80] at (5,0.6)  {Mating priority};
	\node (res) [box, fill=green!70, above of=mat] {Resource access};
	\node (def) [box, fill=green!60, above of=res] {Predator defense};
	\node (surv) [box, fill=green!50, above of=def] {Survival};
	\node (dom) [box, fill=green!40, above of=surv] {Social Dominance};

\end{tikzpicture}
}
	\caption{Stabilizing selection}
	\label{fig:stab}
\end{figure}

