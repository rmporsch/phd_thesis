\chapter{Introduction}
\label{cha:introduction}
%\begin{refsection}
World War I and II, the Holocaust, as well as more recent terror attacks have demonstrated the extend of which human aggression is capable of causing distruction and harm to others. 
However, despite these horrific events in recent history war and terror has been in decline over the last centuries %TODO citation
and the percentage of deaths due to war and murder have been in steady decline, thus contributing to the increased life expectency.
Despite this progress aggressive behavior remains a continous issue in all human cultures and significant reseach effort has been spend to identify causes of human aggression.  
This ranges from gun control, global warning, computer games and movies, as well as others to more biological and evolutionary causes. %TODO needs some citation for each
While potential enviormental causes of aggressive behavior in humans have enjoyed great attention by the public, genetic and evolutionary explantions are of lesser prominence.

However, aggressive behavior has distinct evolutionary advantages.
For example, given limited available resources, aggressive behavior enables individuals to compete in a potential life threatening situations \cite[]{Anholt2012}.
In addition, hirarical order in social groups of individuals is often established by means of aggressive behavior which enables high ranked individuals prioriy access to food and mating paterns. 
Thus it has been suggested that aggressive behavior is under sexual selection\cite{Lindenfors2011}.
Despite these advatages, increased levels of aggressive behavior can also be harmful to the individual.
Expressing aggression towards another individual sheers energy away from other acitivies, such as hunting and foraging, and also carries a high risk of injury and death \cite{Packer1995}.  
Hence previous reseach on multiple different species suggested that aggressive behavior is under stabilizing selection.
Further this also shows strong evidence for a large role of genetic factors in the etiology of aggressive behavior. 


In addition one can also see aggressive behavior as part of a wider phenotypical complex. 

One can define human aggression as a behavior intended to cause physical or emotional harm to others \cite{Anderson2002}.

%TODO what is aggression
%TODO effect of aggression on society (why is research in this important)
%TODO differences between forms of aggression
%TODO differences in children and adults

\section{Early studies on aggressive behavior}
\label{sec:early_studies_on_aggressive_behavior}

%TODO early ideas about genetics and aggressive behavior

\section{Twin and adoption studies}
\label{sec:twin_and_adoption_studies}

%TODO first adoption and twin studies
%TODO genetic estimates and heritability
%TODO problems of twin studies

\section{Molecular genetics and aggressive behavior}
\label{sec:molecular_genetics_and_aggressive_behavior}

%TODO early finding in mouse models
%TODO studies in human subjects
%TODO problems in those studies

\section{Research gaps and brief description of conducted studies}
\label{sec:research_gaps_and_brief_description_of_conducted_studies}

%\end{refsection}
