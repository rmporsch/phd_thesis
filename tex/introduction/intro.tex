%&tex
\chapter{Introduction}
\label{cha:introduction}

% Dramatic entry
World War I and II, the Holocaust, as well as more recent terror attacks have demonstrated the extend of which human aggression is capable of causing distruction and harm to an extend previously unknown. 
Thus aggressive behavior remains a continous issue in all human societies and significant reseach effort has been spend to identify causes of human aggression.
This ranges from gun control, global warning, computer games and movies, as well as others to more biological and evolutionary causes. %TODO needs some citation for each
Here I will present major genetic characteristics of aggressive behavior in children and adults.
Ranging from studies on twins to large scale genome wide association studies.

% Intro to the intro
However, first I will present current and past research on aggressive behavior.
Within this literature review I will first present different subtypes of aggresssion.
Following I will describe commonly associated enviormental factors contributing to aggressive behavior.
This is succeeded by an introduction into the evolutionary theories of aggression.
I will then describe to which extend genetic factors influence aggressive behavior followed by a sumamry of recent discoveries in molecular genetics.
I will conclude this section by a short describtion of the here presented studies.

\section{Definition of Aggression}
\label{sec:overview_of_reseach_in_aggression}

One can define human aggression as a behavior intended to cause physical or emotional harm to others \cite{Anderson2002}.
However, it is important to not only consider the motivation of the predator but also the unwilling participation of the victim.
Thus to also include behaviors in which the target does not intend to avoid the aggressive behavior, such as in sexual masochism one needs to extend the simple definition by the believe of the predator that the target wants to avoid the behavior (e.i. to avoid being harmed) \cite{Berkowitz1993,Baumeister1989,Baron2007,Geen2001} as well as the motivation of the victim to avoid such harm.  
Therefore I will use the following working definition:
\begin{mydef}[Aggression]
	\label{def:aggression}
	"Aggression is the delivery of an aversive stimulus form one person to another, with itent to harm and with an expectations of causing such harm, when the other person is motivated to escape or avoid the stimulus" \cite{Geen2001}
\end{mydef}

This basic definition does not cover all possible situations and involved factors.
For example it does not account for emotions or complex cognitive processes preceeding aggressive behavior.
However, it does provide a basic necessary working definition for aggressive behavior in adults and children.  
While aggression is most commonly associated with physical harm, definition~\ref{def:aggression} also includes a more broader spectrum.
This includes spreading gossips, damaging a victims property either due to emotional anger or as a planned action to gain an advantage to a higher goal.
The possible spectrum of aggression makes it necessary to specify some more broader dimension of this behavior.

\subsection{Forms of aggression}
\label{sub:forms_of_aggression}

I will futher distinguish between \textit{Affective Aggression} and \textit{Instrumental Aggresssion}.
While the former is characerised as emotional, impulsive, thoughtless and unplanned behavior, \textit{Instrumental Aggression} is defined as a planed and proactive behavior to obtain a certain higher goal \cite{Berkowitz1993,Geen2001}.
It is important to emphasise the strong negative emotional state of affective aggression.
This state, often considered as \textit{anger}, launches and guids affective aggression \citet{Geen2001} and is often caused by some form of provocation.
However, \cite{Frijda1994} suggested that \textit{affective aggression} is not necessary impulsive.
In some situations a delayed response between provocation and aggressive response is observed. 
In particular long term grudges, or \textit{hatreds} are preoccupations which go beyond the initial provocation but remain deeply emotional.
While those sentiments are accompanied by similar activity in the central and autonomic nervous system %TODO check this, seems a bit fishy need a citation for that maybe frijda1994
I will use the term \textit{impulsive aggression} to refer to impulsive, unplanned aggressive behavior.

\textit{Instrumental aggression} is characerised by the absence of an emotional strong cause to cause harm.
For example, the use of gossip and bad-mouthing of a colleague in order to obtain higher chances of receiving a promotion is done in a planned manner with the aim of a higher goal (receiving the promotion).
However, it is often difficult to distiguish actions into affective and instrumental aggression since both forms are not mutually exclusive.
\citet{Geen2001} gave the example of a mother who uses corporal punishment to modify her child's behavior, while still reacting in anger when observing the undesired child's behavior.

A number of other authors have refered to affective and instrumental aggresssion as \textit{reactive} and \textit{proactive} aggression.
Thus aggressive action in response to a provocation, such as self-defence and in anger, is \textit{reactive aggression} while planned, un-provoced aggression is called \textit{provocative} \cite{Geen2001}.

Both forms of aggression, affective and instrumental, can be either physical or verbal.
While physical aggression in humans is homologous to other animals, verbal aggression, also called indirect, relational and social aggression, is relativly distinct to humans \cite{Archer2005}.
These verbal behaviors cause harm to others by gossiping, spreading rumors, or excluding other from social groups.
While the terms \textit{indirect}, \textit{relational}, and \textit{social} aggression have been differently conceptulized in the past \cite{Archer2001}, they are expressed in common behaviors, can be contrasted to physical aggression to a similar extend as well as have been shown to demonstrate the similar differences between the two sexes \cite{Archer2004}.
Hence the terms are more similar than distinct and I will therefore proceed to call all them \textit{indirect} aggression in order to distiguish it more from the physical, more direct aggressive behavior\cite{Archer2005}.

Studies investigating aggression in animals have often distiguished between \textit{offensive} and \textit{deffensive} aggression \cite{Blanchard2005b}.
Similar to \textit{affective} aggression \textit{offensive} attacks arise from a response to a threat to the animals resources, thus are the response to a certain provocation.
These resources could be sexual partners, food, social status, or in the case of humans, also money.
On the other hand \textit{deffensive} aggression is the result to a direct threat to the subject life, a concept closer related to \textit{instrumental} aggression.
While this distingen might hold in mice and rats, the seperation between \textit{offensive} and \textit{deffensive} aggression is more blured in primates, including humans.
For example, humans are known to hunt lions and other predators.
In contrast to non-predators, these animals are, for the most part, not eaten which would suggest a form of deffensive aggressive behavior.
However, within most human cultures killing a large predator is seen to enlarge ones social status by showing strenght and courage towards the others.
A behavior which could be described as an \textit{offensive} action.
Thus the distinction between \textit{offensive} and \textit{deffensive} aggression is blured.
This blured distinction reflects the above described example of the punishing mother by \cite{Geen2001}.
However, \cite{Blanchard2005b} suggest, while the distinction between \textit{offensice/affective} a and \textit{defensive/instrumental} might be blured in humans, the distiction holds in general.
It is rather insufficient analysis and not a disconnetion between animal and human behavior.




Historically, research of aggression has been devided into nurture versus nature\cite{Archer2009}. 
Proponents of the nurture side have argued that aggressive behavior is caused by enviormental influences, while supports of the nature side of the discussion supported the idea that only differences in the genetic architecture are able to explain the differences in aggressive behavior.
Today's view is less polarized and acknoweghes that both, nature and nurture play a crucial part in aggression.
While not disregarding the enviormental aspect of aggression completly I will mostly focus on the genetic and biological causes of aggressive behavior.
In the following section I will outline evolutionary concepts helpful in explaining individual differences of aggression. 

\section{Evolutionary Theories of Aggressive Behavior}
\label{sec:evolutionary_theories_on_aggressive_behavior}

Paleontological findings of broken bones, rips and smashed skulls, unexplainable without the consideration of weaponary force, and occasional findings of weapon fragments in skeletal rib cages all suggest that violence and aggression has been part of the human evolutionary history. 
\citet{Buss1997}, one of the founder of Evolutionary Psychology, suggested that all psychological mechanisms and behavior originate in the evolutionary priciple of selection.  
These mechanisms are aimed to solve adaptive problems.
Hence some variants might solve certain problems better than others which will result in better fitness.
Resulting in the preservation, replication and spreading throught a population \cite{Buss1997}.
Hence from a evolutionary perspective every human behavior can be seen as a solution to a specific adaptive problem.

\citet{Buss1997} suggested seven adaptive problems to which aggressive behavior might be an evolutionary solution.
This includes \textit{Co-Opt the Resources of Others}, which can be defined as the use of physical or psycholigcal force to obtain resources hold by another indivudual or group.
These resources could mean food, water, land or secual partners.
Indeed aggressive behavior in children is often about resouces, such as toys \cite{Campbell1995} suggesting that this behavioral adaptations is an old evelutionary strategy.
However, forceful taking ownership of resources carries a significant amount of risk to be hurt or even killed.
In particular aggression can also be  useful in \textit{defending against an attack}.
Attacking aggressors are a serious threat to valuable resources.
Aggression can be an effective strategy in defending against indivuduals or groups.
Further, it can be also an adaptive strategy to foster a reputation that would deter potential aggressors \cite{Buss1997}.

Another evolutionary benefit of aggression can be found in the  social hierarchies in groups.
For example, men who win fights and defeat opponents gain power and status in many societies \cite{Hill1996}.
The gain in social status can be beneficial in accessing resources and mates \cite{Archer2009}.
Indeed, hirarical order in social groups is often established by means of aggressive behavior which enables high ranked individuals prioriy access to food and mating paterns\cite{Lindenfors2011}. 
However, aggression can also result in an decline in status.
\citet{Buss1997} suggested for example that a physical conflict between two professors in a facutly meeting would result in an decline in repuatation.
Thus display of aggression is not acceptable in all social situations.

Further aggressive behavior sometimes plays a crucial part in mating.
Aggression towards same-sex individuals is sometimes aimed to reduce their social status and therefore make them less attractive to the other sex \cite{Buss1990}.
Hence inflicting damage on a rival directly translates to an incresed benefit to the aggresssor.
In addition to aggression towards same-sex individuals, aggression is also prevelant towards the oposite sex.
Thus aggression can be used to deter a long-term mate from infidelity \cite{Daly1982}.
However, also here aggression can have negative consequences in form of retahilation.
For example, a husband might be reluctant to use aggression towards his wive when she is living close to a number of brothers and a pwerful father.
Indeed, a study in Madrid, Spain found that women who had a higher density of genetic kin in and around Madrid were less likely to be victim of domestic violence \cite{Figueredo1995}.
Hence aggression only brings an evolutionary advantage when the benefits outweight the potential costs.

In addition, nearly all mammals display sex differences in the expression of aggression.
Males are more likely to exhibit physical aggression than female
In fact sexual selection theory (SST) has been used to repeatedly to explain the observed greater physical aggression in males in nearly all mammals \cite{Archer2004,Anderson2002}. 
Sexual selection is concerned how a member of one sex choses another individual from the other sex, as well as the competion between members of the sex over access to the other \cite{Darwin1859}.
In most mammals the more competitive sex is the male \cite{Archer2009}. 
\citet{Trivers1972} suggested that these sex differences can be explained by the commonly observed reduced parental investment by males.
Parental investment is the amount of resources a parent investes into his or her off spring to increase its survivial and reproduction \cite{Archer2009}.
The theory was first suggested by \citet{0198504403} and proses that a male can minimize his parental investment in favor in producing a higher amount of offspring.
In contrast, many female mammals have an large obligatory patrenatal investment, such as gestation and delivery.
Thus female partner selection is more careful compared to male.
Hence the female selection of good reproductive fitness is directly linked to offset the lack of male paternal investment.
Since aggression is, as described above, often associated with an increased access to sexual partners one can conclude that this behavior is under sexual selection.

One of the most crucial differences between male and female behavior is the extend to which an individual is willing to take during a conflict %TODO cite.

Despite these advatages, increased levels of aggressive behavior can also be harmful to the individual.
Expressing aggression towards another individual sheers energy away from other acitivies, such as hunting and foraging, and also carries a high risk of injury and death \cite{Packer1995}.  
Hence previous reseach on multiple different species suggested that aggressive behavior is under stabilizing selection.
Further this also shows strong evidence for a large role of genetic factors in the etiology of aggressive behavior. 


In addition one can also see aggressive behavior as part of a wider phenotypical complex. 

\begin{figure}
	\centering
	\scalebox{0.8}{\tikzstyle{box} = [rectangle, rounded corners, minimum width=3cm, minimum height=1cm,text centered, draw=black]
\begin{tikzpicture}
	\draw[line width=6, gray] (-7,0)-- (0,0) --(7,0);
	\draw[gray, fill=gray] (-1,-2)-- (0,0) --(1,-2) --(-1,-2);

	%Cost
	\node (injury) [box, fill=red!80] at (-5,0.6)  {Reproductive Cost};
	\node (grooming) [box, fill=red!70, above of=injury] {Foraging};
	\node (repo) [box, fill=red!60, above of=grooming] {Parental care};
	\node (foraging) [box, fill=red!50, above of=repo] {Grooming};
	\node (parCare) [box, fill=red!40, above of=foraging] {Injury risk};

	%Benefit
	\node (mat) [box, fill=green!80] at (5,0.6)  {Mating priority};
	\node (res) [box, fill=green!70, above of=mat] {Resource access};
	\node (def) [box, fill=green!60, above of=res] {Predator defense};
	\node (surv) [box, fill=green!50, above of=def] {Survival};
	\node (dom) [box, fill=green!40, above of=surv] {Social Dominance};

\end{tikzpicture}
}
	\caption{Stabilizing selection}
	\label{fig:stab}
\end{figure}



