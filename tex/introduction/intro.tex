\chapter{Introduction}
\label{cha:introduction}

% Dramatic entry
World War I and II, the Holocaust, as well as more recent terror attacks have demonstrated the extend of which human aggression is capable of causing distruction and harm to an extend previously unknown. 
Thus aggressive behavior remains a continous issue in all human societies and significant reseach effort has been spend to identify causes of human aggression.
This ranges from gun control, global warning, computer games and movies, as well as others to more biological and evolutionary causes. %TODO needs some citation for each
Here I will present major genetic characteristics of aggressive behavior in children and adults.
Ranging from studies on twins to large scale genome wide association studies.

% Intro to the intro
However, first I will present current and past research on aggressive behavior.
Within this literature review I will first present different subtypes of aggresssion.
Following I will describe commonly associated enviormental factors contributing to aggressive behavior.
This is succeeded by an introduction into the evolutionary theories of aggression.
I will then describe to which extend genetic factors influence aggressive behavior followed by a sumamry of recent discoveries in molecular genetics.
I will conclude this section by a short describtion of the here presented studies.

\section{Overview of Reseach in Aggression}
\label{sec:overview_of_reseach_in_aggression}

One can define human aggression as a behavior intended to cause physical or emotional harm to others \cite{Anderson2002}.
This definition is relativly broad and does also include behavior in which the target does not intend to avoid the aggressive behavior, such as in sexual masochism.
Hence in addition to the intend to cause harm the preptrator must also believe that the target wants to avoid the behavior\cite{Berkowitz1993,Baumeister1989,Baron2007,Geen2001}.  
It is also important to further distinguish between \textit{Hostile Aggression} and \textit{Instrumental Aggresssion}.
While the former is characerised as impulsive, thoughtless and unplanned; \textit{Instrumental Aggression} is defined as a planed and proactive behavior to obtain a certain higher goal \cite{Berkowitz1993,Geen2001}.
Both forms of aggression can be either physical or verbal.
While physical aggression in humans is homologous to other animals, verbal aggression, also called indirect, relational and social aggression, is relativly distinct to humans \cite{Archer2005}.
These verbal behaviors cause harm to others by gossiping, spreading rumors, or excluding other from social groups.
While the terms \textit{indirect}, \textit{relational}, and \textit{social} aggression have been differently conceptulized in the past \cite{Archer2001}, they are expressed in common behaviors, can be contrasted to physical aggression to a similar extend as well as show the same sex differences \cite{Archer2004}.
Hence the terms are more similar than distinct and I will therefore proceed to call all them \textit{indirect} aggression in order to distiguish it more from the physical, more direct aggressive behavior.




However, aggressive behavior has distinct evolutionary advantages.
For example, given limited available resources, aggressive behavior enables individuals to compete in a potential life threatening situations \cite[]{Anholt2012}.
In addition, hirarical order in social groups of individuals is often established by means of aggressive behavior which enables high ranked individuals prioriy access to food and mating paterns. 
Thus it has been suggested that aggressive behavior is under sexual selection\cite{Lindenfors2011}.
Despite these advatages, increased levels of aggressive behavior can also be harmful to the individual.
Expressing aggression towards another individual sheers energy away from other acitivies, such as hunting and foraging, and also carries a high risk of injury and death \cite{Packer1995}.  
Hence previous reseach on multiple different species suggested that aggressive behavior is under stabilizing selection.
Further this also shows strong evidence for a large role of genetic factors in the etiology of aggressive behavior. 


In addition one can also see aggressive behavior as part of a wider phenotypical complex. 


