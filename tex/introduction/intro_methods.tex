\documentclass[../header.tex]{subfiles}
% Evo history
\newacronym{rhp}{RHP}{Resource Holding Power}
\newacronym{rv}{RV}{Resource Value}
\newacronym{adhd}{ADHD}{Attention Deficit Hyperactivity Disorder}
\newacronym{sst}{SST}{Sexual Selection Theory}
\newacronym{vntr}{VNTR}{Variable Number Tandem Repeat}


% Methods twins
\newacronym{sem}{SEM}{Structural Equation Model}
\newacronym{mz}{MZ}{monozygotic}
\newacronym{dz}{DZ}{dizygotic}

% Methods gwas
\newacronym{gwas}{GWAS}{Genome Wide Association Study}
\newacronym{snp}{SNP}{Single Nucleotide Polymorphism}
\newacronym{ld}{LD}{Linkage Disequilirium}
\newacronym{pca}{PCA}{Principle Component Analysis}
\newacronym{pc}{PC}{Principal Component}
\newacronym{fdr}{FDR}{False Discovery Rate}
\newacronym{gcta}{GCTA}{Genome-wide Complex Trait Analysis}
\newacronym{maf}{MAF}{Minor Allele Frequency}
\newacronym{skat}{SKAT}{sequence kernel association test}

\begin{document}
\chapter{General Methodology}
\label{cha:methods_applied_in_genetic_studies_on_humans}

The investigation of the underlying genetic architecture of human traits is limited to non-experimental studies and one can broadly distinguish between two types.
Twin studies and association studies.
Before the availability to identify specific genetic variants within the human genome, researchers have made use of twins.
Specifically, monozygotic and dizygotic twin pairs provide the ability to estimate the contribution of genetic and environmental components on a trait.
However, more recently new molecular methods have been developed to identify specific genetic variations within the human genome.
This has allowed researchers to identify specific genetic markers associated with a variety of human traits and diseases.

Within this section I will describe commonly used methods in both twin and molecular based association studies.
I will further describe methods used to estimate heritability and genetic correlations in association studies of common variants.

\section{Twin based studies}
\label{sec:twin_based_studies}

Twins have always been of special interest to scholars.
Indeed already Hippocrates has been reported to be interested in twins at around 5th century BCE\@.
While the original accounts are lost the Roman politician and author Cicero described Hippocrates's observations of two ill brothers, suspected to be twins, with similar identical disease progression~\cite{Cicero44BC}.
Much later Francis Galton was one of the first persons to use twins in order to investigate the effect of genes and the environment on human behavior~\cite{Rende1990} and with the discovery of two distinct types of twins, namely \acrfull{mz} and \acrfull{dz} twins, by~\citet{Simens1924}, twin studies became an established instrument in investigating genetic factors in humans.
MZ twins develop from a single fertilized egg and therefore share all of their genetic variations.
On the other hand, \acrfull{dz} twins develop from two fertilized eggs and share only 50\% of their genetic variations.

This distinction forms the basis of all twin studies and allows to state structural equations relating observed trait and theorized underlying genetic and environmental effects.
Genetic effects can be further distinguished between additive genetic effects (A) which represents the accumulated effect of all genetic variations and dominant (D) effects which represents interaction on the same genetic locus.
Environmental components are differentiated into shared environment (C) and unique environment (E).
Therefore the total variance of a any particular trait P is $P = A+D+C+E$.

Since one can assume different correlations between MZ and DZ twin pairs one can estimate components of $P$.
While the correlations between twins within C and E are the same in both MZ and DZ twins, namely $1$ and $0$ respectively.
MZ twins have a correlation of $1$ for both A and D, while DZ pairs have a correlation of $\frac{1}{2}$ and $\frac{1}{4}$ respectively.
Therefore differences within MZ twins can be attributed to E alone.
Further, when we assume that DZ and MZ twins are exposed to the same degree of similarity within their environment, the differences in similarity between MZ and DZ twins is an estimate of $A+D$.
This is also called Falconer's formula (see Formula~\ref{eq:falcon}) and can be used to estimate heritability, or the relative importance of genetic effects.

\begin{align}
  h^2 &= 2(r_{MZ}-r_{DZ})\label{eq:falcon} \\ 
  C &= r_{MZ} - h^2  \\
  E &= 1 - h^2 + c^2  
\end{align}

However, while the above is attractive it its simplicity, today's twin studies use \acrfull{sem} to model genetic and environment effects.
SEMs are more flexible in modeling specific hypothesis, are able to test for sex differences as well are able to handle multivariate data~\cite{Rijsdijk2002}.

Figure~\ref{fig:ace} displays such a classical path based model.
Variables within an SEM can be separated into latent and observed variables.
Additive genetic (A), common environment (C) and unique environment (E) are so called latent variables.
These variables are not directly observed but are inferred from actual measured variables.
Observed variables are commonly displayed in cornered boxes.
The double headed arrows represents the correlations among A and C.
The causal paths $a$, $c$, and $e$ represents the to estimate effect of the components on the trait $T$.
The square of these estimates are the variance components of A, C and E respectively.
However, effects of $D$ cannot be simultaneously estimated with $A$, and two separate models need to be constructed.

\begin{figure}[htpb]
  \centering
  \scalebox{0.6}{%\usetikzlibrary{external}
%\tikzset{external/system call={latex \tikzexternalcheckshellescape -halt-on-error
%		-interaction=batchmode -jobname "\image" "\texsource";
%		dvips -o "\image".ps "\image".dvi ;
%		ps2eps "\image.ps" "\image".eps}}
%\tikzexternalize
%\newcommand{\at}{\makeatletter @\makeatother}
\begin{tikzpicture}[auto,node distance=.5cm,
    latent/.style={circle,draw,very thick,inner sep=0pt,minimum size=15mm,align=center},
    manifest/.style={rectangle,draw,very thick,inner sep=0pt,minimum width=45mm,minimum height=10mm},
    paths/.style={->, ultra thick, >=stealth'},
    twopaths2/.style={<->, ultra thick,bend left=90, >=stealth'},
    twopaths1/.style={<->, ultra thick,bend right=90, >=stealth'},
    mean/.style={draw, regular polygon, regular polygon sides=3, node distance=1cm, minimum height=15mm}
]

% Define observed variables
\node [manifest] (T1) at (0,0) {T1};
\node [manifest] (T2) [below=of T1, below=5cm of T1]  {T2};


% Define latent variables
\node [latent] (C1) [left=3.5cm of T1] {C1};
\node [latent] (C2) [left=3.5cm of T2] {C2};
\node [latent] (A1) [above=of C1] {A1};
\node [latent] (A2) [above=of C2] {A2};
\node [latent] (E1) [below=of C1] {E1};
\node [latent] (E2) [below=of C2] {E2};

\node [mean] (mu) at($(T1)!0.5!(T2)$)  {$\mu$};

% paths to T1/T2
\draw [paths] (A1.east) to node {$a$} (T1);
\draw [paths] (A2.east) to node {$a$} (T2);
\draw [paths] (C1.east) to node {$c$} (T1);
\draw [paths] (C2.east) to node {$c$} (T2);
\draw [paths] (E1.east) to node {$e$} (T1);
\draw [paths] (E2.east) to node {$e$} (T2);

% path from mean
\draw [paths] (mu.south) to node [right] {} (T2);
\draw [paths] (mu.north) to node [right] {} (T1);

% variance
\draw [twopaths1] (A1.west) to node  [bend left=90, left]{0.5 / 1} (A2.west);
\draw [twopaths2] (C2.west) to node  [bend right=90, left]{1} (C1.west);

\end{tikzpicture}
}
  \caption{
    Basic ACE model.
    This basic model contains the latent variables A, C and E for twin 1 and 2, as well as the observed variable T with the mean $\mu$.
  }\label{fig:ace}
\end{figure}

The covariance matrix of the model in Figure~\ref{fig:ace} is therefore
\begin{equation}
  cov(MZ) = 
  \begin{pmatrix}
    a^2 + c^2 + e^2 & a^2 + c^2 \\
    a^2 + c^2 & a^2 + c^2 + e^2
  \end{pmatrix}
\end{equation}
and 
\begin{equation}
  cov(DZ) = 
  \begin{pmatrix}
    a^2 + c^2 + e^2 & \frac{1}{2}a^2 + c^2 \\
    \frac{1}{2}a^2 + c^2 & a^2 + c^2 + e^2
  \end{pmatrix}
\end{equation}

Modern SEM software is able to estimate parameters by minimising the goodness-of-fit statistic between the observed and predicted covariance matrices~\cite{Boker2011}.
Most commonly this is done via a maximum-likelihood function.
Further, the overall goodness-of-fit of the model relatively to a perfect fit, meaning that parameters are considered as `free' and their maximum-likelihood estimate will equal the sample covariance, are measured by a likelihood ratio square statistic ($\chi^2$).
Therefore, should we fail to reject the null hypothesis that our model in Figure~\ref{fig:ace} is different from a perfect fitted model we have reason to assume that our genetic model fits the data.

The use of SEM allows for great flexibility and a variety of models to be estimated.
In the past few decades numerous twin studies on a variety of traits have been performed.
It not only allowed to test for the differences in the genetic architecture between the sexes but also to look at how the influence of genetic factors change over age.
However, due to the new advancement in acquiring genetic information of individual more research has been shifted to \acrfull{gwas}.
Hence, in the next section I will outline the methods applied GWAS\@.

\section{Association studies of common variants}
\label{sec:association_studies_of_common_variants}

Large scale genomic association studies have enabled researchers to investigate specific genetic factors associated with a certain trait.
Hence, while twin studies were only able to give an estimate of the total contribution of genetic factors on a phenotype, association studies are able to elucidate the specific molecular basis of complex traits.
Specifically, these genome wide association studies utilize common \acrfull{snp} and other genetic variations, to identify specific genetic markers associated with a certain trait.

\paragraph{What are \acrfull{snp}}
\label{par:what_are_snp}
SNPs are variations within the genome at a specific position and underly differences in traits and disease susceptibility. 
For example, the replacement of the nucleotide cytosine (C) with tymine at a certain position within a strech of DNA is considered a \acrfull{snp}.
Most SNPs have no effect or little effect on specific traits.

Before explaining association studies in more detail it is important to mention the concept of \acrfull{ld}.
LD is `the nonrandom association of alleles at different loci'~\cite{Slatkin2008} and forms a marker of the population genetic mechanism that is at play within our genome.
For example, two loci are said to be in high LD when allele $A$ at one loci co-occurs with allele $B$ at a different loci at a higher frequency then you would expect if the two loci were independent.
Hence the level of LD can be quantified as $D_{AB}=p_{AB} - p_{A}p_{B}$ in which in which $p_{AB}$ is the frequency that $A$ and $B$ occurs together wile $p_A$ and $p_B$ is the frequency of $A$ and $B$ respectively.
If $D_{AB} \neq 0$ A and B are said to be in linkage disequilibrium, otherwise the two alleles are in linkage equailibrium ($D_{AB}=0$).
Nevertheless, $D_{AB}$ depends on the frequencies of the alleles in questions and are therefore not always convenient.
Therefore, LD between two loci is commonly measured in two different ways. 
That is $D'$ and $r^2$.
\citet{Lewontin1964} suggested to use
\begin{equation}\label{eq:dprime}
  D' = D/D_{\min}
\end{equation}
where 
\begin{equation*}
  D_{\min}= \begin{cases}
    \max\{-p_A p_B,\,-(1-p_A)(1-p_B)\} & \text{when } D < 0\\
    \min\{p_A (1-p_B),\,(1-p_A) p_B\} & \text{when } D > 0
  \end{cases} 
\end{equation*}
Alternatively, one can also use the correlation coefficient between the two loci 
\begin{equation}\label{eq:r2}
  r=\frac{D}{\sqrt{p_A(1-p_A)p_B (1-p_B)}}
\end{equation}
An important consequence for association studies is that an association between a trait and an allele is unlikely to be the actual causal SNP\@.
An association between an SNP and a trait can arise out of multiple reasons.
First, the association could represent the true effect and the particular SNP has a causal relationship to the trait in question.
Second and more likely, the associated SNP is a proxy of the causal SNP since both are in high \acrshort{ld}.
Third, the association is a random fluctuation within the sample, and 
fourth the association is due to confounding errors such as population stratification or genotyping errors.

During the last decade the cost of obtaining a person's genome reduced rapidly.
This enabled the analysis of not only a few genetic markers but the whole genome.
The analysis of the whole genome is called \acrfull{gwas}.
Most GWAS are population based studies of unrelated samples and analyse genetic variations with a substantial frequency within the population, usually $\ge 1\%$.
This frequency is also called \acrfull{maf}.
Here it is important to note that all samples have distant relatives and the term `unrelated' is used loosely.
In fact it becomes increasingly difficult to establish the exact genetic relationship between two participants by SNPs alone as further those subjects are apart within the family tree.

The association between an SNP and a trait can be expressed as a linear regression model for a continuous and binary phenotype.
While other methods to estimate the effect exist, such as the $\chi^2$-test, a regression model allows to include multiple covariates.
The inclusion of multiple covariates is crucial since it allows to adjust for confounding factors such as population stratification, sex, genotyping platforms and others.
I will now proceed with describing one of the most important confounder, population stratification.

\subsection{Population Stratification}
\label{ssec:population_stratification}
Population stratification takes place when differences in the frequencies of alleles among cases and controls are not due to an causal relationship between the SNP and the trait.
Rather it is caused by ancestral differences across populations.
An association is affected by population stratification if the trait is more prevalent in one population while the allele frequencies vary across the populations as well.

Commonly one can adjust for population stratification by the usage of \acrfull{pca}.
PCA is a procedure which transforms a set of correlated variables into a set of linear uncorrelated once, called principle components (\acrshort{pc}).
The number of PC can be smaller or equal that of the number of initial variables and the first PC accounts for most of the variability in the set of correlated variables.
Each following PC explains the most variance constrained that it is orthogonal to the previous.
Using PCA on a matrix of genotypes results in a set of PC which explain the genetic variation within the sample.
Given the sample is a mixture out of multiple population with different ancestry the commutated PC will often have geographic interpretation.
Therefore including PCs into the association model will adjust for population stratification arising due to differences in allele frequencies and disease prevalence.

\subsection{Multiple Testing}
\label{ssec:multiple_testing}
Testing a large amount of genetic variants without adjusting the significant threshold $\alpha$ results in a large number of falsely associated variants.
Therefore adjustments of the significant threshold is necessary.
For example, one could simply adjust $\alpha$ by the number of tests performed (Bonferroni threshold).
However, this would result in an overly conservative threshold and in a number of false negative associations~\cite{Benjamini1995} since computed test statistics are not independent.
Indeed,~\citet{Peer2008} estimated, based on data from the International Haplotype Map Consortium, the number of independent tests to be one million in Europeans and two million in African populations.
Therefore most GWASs have used a threshold of $5\times 10^{-8}$.
However, due to the introduction of larger sample sizes as well as better technology we are able to genotype variants with lower allele frequency.
This requires adjustment of the GWAS significant threshold since it also increases the number of independent tests.
Hence~\citet{Fadista2016} recently suggested to use $3\times10^{-8}$, $2\times10^{-8}$, $1\times10^{-8}$ for variants with $MAF\ge1\%$, $MAF\ge0.5\%$ and$ MAF\ge0.1\%$.

\acrshort{gwas} are useful in identifying molecular markers for traits and diseases.
While there are multiple possible confounding factors, such as population stratification, methods have been developed to approach these problems successfully.
The process of conducting a GWAS is now days well defined.
However, usually GWAS do not deal with genetic variations with a frequencies below 1\%.
Hence I will next give an overview of association studies of rare variants.

\section{Heritability and Genetic Correlation}
\label{sec:heritability_and_genetic_correlation}

As already described above, studies on twins were able to assess the heritability of traits by considering the inter-correlation among MZ and DZ twins.
Therefore, one would expect that the variance explained by all genetic variants combined would result in similar estimates.
Unfortunately, for many traits, this is not the case.
This is called the `Missing Heritability Problem'~\cite{Vineis2010}.
A number of reasons have been suggested for the discrepancy between estimates in twin studies and those in GWAS\@.

First, the assumptions in twin studies might be violated and estimates are too high.
In particular, studies on twins assume that the influences of shared enviormental factors influences MZ and DZ twins to the same extend.
However, one can argue that MZ twins are treated by parents, teachers and peers differently resulting in a potential violation of the this equal environment assumption. 

Second, GWASs only consider common variants and rare variants might account for the missing heritability.
While most rare variants have little or no effet on traits, some rare variants have very large effects.
Indeed, a study on hight found that while most common variants only had small effects, some rare genetic variants resulted in a large increase in hight of up to $2cm$~\cite{Marouli2017}.
This would indicate that rare variants can have a profound effect on common traits, suggesting that heritability estimations of common variantions alone might result in lower estimates.

Third, epistasis is only insufficiently captured in most studies and could account for the differences.
Indeed, many twin concordance rates indicate the presents of non-additive effects to some extend.
Further, complex gene-gene interaction have been shown in a number of experimental studies in nonhuman animals on complex traits.
Therefore suggesting that epistastic effect might play a major role in humans as well.
However, models which include gene-gene or even gene-gene-gene and higher order interactions are very computational intensive and require large datasets~\cite{Lippert2013}. 

Fourth, gene-environment interaction could also explain partly the differences between the two types of studies.
Similar to gene-gene interactions gene-enviorment interactions are computationally intensive and require large samples.
However, in contrast to epistastic effects gene-enviorment interactions are more complex and might change over age depending on the trait in question.
Hence assesment of the impact of gene-enviorment interactions on the missing heritability is diffucult to estiamte~\cite{Do2016,Kaprio2012}.

Fifth, gene regulatory components, such as RNA expression and methylation, might explain those differences.
Indeed only recently large scale population based method have been made available to estimate the effect of regulatory componets on heritability~\cite{Trerotola2015}.
Indeed gene expression altering epigenetic modifications have been shown to be heritable without affecting the underlying DNA sequence.
This seperate model of inheritence, next to Mendelian heredity, would be undetectable in GWAS and could contribute to the missing heritability.

While a single reason for the missing heritability seems unlikely it is still an ongoing research objective to account for the differences.
Further, the estimation of the narrow sense heritability, or the additive genetic effects, from genotyped data is not trivial.
Several methods have been proposed in the past, most notably \acrfull{gcta} and LD-score regression.
In the next section I will describe both methods as well as their shortcomings.

\subsection{\acrfull{gcta}}
\label{sub:gcta}

GCTA uses mixed linear model to the fit the effect of all SNPs by making use of the genetic relationship matrix of all included subjects~\cite{Yang2011}.
If $\textbf{A}$ is the genetic relatedness matrix then
\begin{equation}
  y = X\beta + g + \epsilon \text{ with } var(y) = V = A\sigma^2_g + I\sigma^2_\epsilon
\end{equation}
in which $y$ is the phenotype and $\beta$ are the estimated effect sizes of all covariates and the total genetic effects $g$ for each individual is $g \sim N(0, A\sigma^2_g)$.
GCTA is then able to estimate the variance explained by all SNPs $\sigma^2_g$ by restricted maximum likelihood.
Further one can extent this model to bivariate linear mixed models to estimate the genetic correlation between two traits.
If $y_1 = X_1\beta_1 + g_1 + \epsilon_1$ for trait 1 and $y_2= X_2\beta_2 + g_2 + \epsilon_2$ then the variance-covariance matrix $V$ is
\begin{equation}
  V = 
  \begin{pmatrix}
    Z_1AZ_1'\sigma_{g1} + I\sigma^2_{\epsilon 1} & Z_1AZ_2'\sigma_{g_1g_2} \\
    Z_2AZ_1'\sigma_{g_1g_2} & Z_2AZ_2'\sigma_{g2} + I\sigma^2_{\epsilon 2}
  \end{pmatrix}
\end{equation}
in which $X$ and $Z$ are the incidence matrices for the effects of $\beta$ and $g$.
However, GCTA requires considerable computation power as well as the availability of the raw genotyped data.
Hence LD-score regression has been developed to estimate heritability on summary statistics only.

\subsection{LD-score regression}
\label{sub:ld_score_regression}

LD-score regression makes use to the previously outlined LD among tagged and causal SNPs.
Test statistics of SNPs in high LD with the causal variant will be elevated proportional to their LD\@.
Thus the more genetic variation an SNP tags the higher the probability that it will tag a causal variant.
LD-score regression makes use of this relationship and regresses the estimated $\chi^2$ from the association study on the LD-score, which measures the overall LD of variant $j = 1, \ldots, M$ as $\ell_j = \sum^M_{k=1} r^2_{jk}$. 
The slope of this regression can then be interpreted as an estimate of heritability~\cite{Bulik-Sullivan2015}.
Similar if we replace the $\chi^2$ of a single study by the product of the z-score of two separate studies and regress it onto $\ell_j \sqrt{N_{1}N_{2}}$ the slope can be interpreted as the genetic covariance between trait 1 and 2~\cite{Bulik-Sullivan2015a}.

These developments have enabled recent research to uncover the inter-correlation among a variety of different traits.
However, genetic correlation can arise from a multitude of different sources.
Figure~\ref{fig:genetic_correlation} shows 4 different ways genetic correlation can arise.
First and foremost, genetic correlation can arise if two traits are caused by the same genetic variant (see Figure~\ref{fig:pleiotropy}).
Second, a genetic factor which causes phenotype 1 can in turn cause phenotype 2 (see mediated pleiotropy in Figure~\ref{fig:mediated_pleiotropy}).
Third, genetic correlation can also arise from assortative mating as shown in Figure~\ref{fig:assortative_mating}.
Assortative mating is a non-random mating pattern within a population.
For example, consider two traits which share no causal variant.
Trait 1 is desirable in male while trait 2 is desirable in female.
Over a few generation this will result in LD between causal variants of trait 1 and 2 despite sharing no initial causal SNPs.
Therefore resulting in a genetic correlation. 
Last, also parental effect can result in genetic corrections as displayed in Figure~\ref{fig:parental_effects}.
Specifically, genetic components which cause trait $1$ in the parents might influence the child environment which in turn results in trait 2. 
Nevertheless, the genetic correlations estimated by LD-score and GCTA are unable to distinguish between the possible sources of genetic correlations.
%TODO include MR here?

\begin{figure}[htp]
  \begin{subfigure}[t]{0.5\textwidth}
    \centering
    \resizebox{0.5\linewidth}{!}{\begin{tikzpicture}[auto,node distance=.5cm,
    latent/.style={circle,draw,very thick,inner sep=0pt,minimum size=10mm,align=center},
    manifest/.style={rectangle,draw,very thick,inner sep=0pt,minimum width=45mm,minimum height=10mm},
    paths/.style={->, ultra thick, >=stealth'},
    twopaths2/.style={<->, ultra thick,bend left=90, >=stealth'},
    twopaths1/.style={<->, ultra thick,bend right=90, >=stealth'},
    mean/.style={draw, regular polygon, regular polygon sides=3, node distance=1cm, minimum height=10mm}
]

\node [latent] (G) at (0,0) {G};
\node [latent] (P1) at (-2,-2) {P1};
\node [latent] (P2) at (2,-2) {P2};

\draw [paths] (G.south west) to node [right] {} (P1);
\draw [paths] (G.south east) to node [right] {} (P2);

\end{tikzpicture}
} 
    \caption{Pleiotropy}\label{fig:pleiotropy}
  \end{subfigure}
  \begin{subfigure}[t]{0.5\textwidth}
    \centering
    \resizebox{0.5\linewidth}{!}{\begin{tikzpicture}[auto,node distance=.5cm,
    latent/.style={circle,draw,very thick,inner sep=0pt,minimum size=10mm,align=center},
    manifest/.style={rectangle,draw,very thick,inner sep=0pt,minimum width=45mm,minimum height=10mm},
    paths/.style={->, ultra thick, >=stealth'},
    twopaths2/.style={<->, ultra thick,bend left=90, >=stealth'},
    twopaths1/.style={<->, ultra thick,bend right=90, >=stealth'},
    mean/.style={draw, regular polygon, regular polygon sides=3, node distance=1cm, minimum height=10mm}
]

\node [latent] (G) at  (0,0) {G};
\node [latent] (P1) at (2,0) {P1};
\node [latent] (P2) at (4,0) {P2};

\draw [paths] (G.east) to node [right] {} (P1);
\draw [paths] (P1.east) to node [right] {} (P2);

\end{tikzpicture}
} 
    \caption{Mediated Pleiotropy}\label{fig:mediated_pleiotropy}
  \end{subfigure}
  \begin{subfigure}[t]{0.5\textwidth}
    \centering
    \resizebox{0.6\linewidth}{!}{
\begin{tikzpicture}[auto,node distance=.5cm,
    latent/.style={circle,draw,very thick,inner sep=0pt,minimum size=10mm,align=center},
    manifest/.style={rectangle,draw,very thick,inner sep=0pt,minimum width=45mm,minimum height=10mm},
    paths/.style={->, ultra thick, >=stealth'},
    paths2/.style={<->, ultra thick, >=stealth'},
    twopaths2/.style={<->, ultra thick,bend left=90, >=stealth'},
    twopaths1/.style={<->, ultra thick,bend right=90, >=stealth'},
    mean/.style={draw, regular polygon, regular polygon sides=3, node distance=1cm, minimum height=10mm}
]

\node [latent] (P2) at (0,0) {P2};
\node [latent] (P1) at (2,0) {P1};

\node [latent] (G1) at (5,0) {G1};
\node [latent] (G2) at (-3,0) {G2};

\node [latent] (E1) at (4,2) {E};
\node [latent] (E2) at (-2,2) {E};

\node [latent] (P1c) at (2,-2) {P1};
\node [latent] (P2c) at (0,-2) {P2};

\node [latent] (G1c) at (5,-2) {G1};
\node [latent] (G2c) at (-3,-2) {G2};

\node [latent] (E1c) at (4,-4) {E};
\node [latent] (E2c) at (-2,-4) {E};

\draw [paths] (P1.east) to node [right] {} (G1);
\draw [paths] (P2.west) to node [right] {} (G2);
\draw [paths] (G1.south) to node [right] {} (G1c);
\draw [paths] (G2.south) to node [right] {} (G2c);
\draw [paths] (E1.south west) to node [right] {} (P1);
\draw [paths] (E2.south east) to node [right] {} (P2);

\draw [paths] (G1c.west) to node [right] {} (P1c);
\draw [paths] (G2c.east) to node [right] {} (P2c);

\draw [paths] (E1c.north west) to node [right] {} (P1c);
\draw [paths] (E2c.north east) to node [right] {} (P2c);

\draw [paths2] (P1.west) to node [right] {} (P2.east);


\node [text width=1cm] at (5,1) {Father};
\node [text width=1cm] at (-3,1) {Mother};
\node [text width=1cm] at (-3,-3) {Child};

\end{tikzpicture}
} 
    \caption{Assortative Mating}\label{fig:assortative_mating}
  \end{subfigure}
  \begin{subfigure}[t]{0.5\textwidth}
    \centering
    \resizebox{0.6\linewidth}{!}{\begin{tikzpicture}[auto,node distance=.5cm,
    latent/.style={circle,draw,very thick,inner sep=0pt,minimum size=10mm,align=center},
    manifest/.style={rectangle,draw,very thick,inner sep=0pt,minimum width=45mm,minimum height=10mm},
    paths/.style={->, ultra thick, >=stealth'},
    twopaths2/.style={<->, ultra thick,bend left=90, >=stealth'},
    twopaths1/.style={<->, ultra thick,bend right=90, >=stealth'},
    mean/.style={draw, regular polygon, regular polygon sides=3, node distance=1cm, minimum height=10mm}
]

\node [latent] (Gp) at (0,0) {G};
\node [latent] (P1) at (2,0) {P1};
\node [latent] (Ep) at  (4,0) {E};
\node [latent] (Gc) at (0,-2) {G};
\node [latent] (Ec) at  (2,-2) {E};
\node [latent] (P2) at (4,-2) {P2};

\node [text width=1cm] at (-1.5,0) {Parent};
\node [text width=1cm] at (-1.5,-2) {Child};

\draw [paths] (Gp.east) to node [right] {} (P1);
\draw [paths] (Gp.south) to node [right] {} (Gc);
\draw [paths] (P1.south) to node [right] {} (Ec);
\draw [paths] (Ep.west) to node [right] {} (P1);
\draw [paths] (Ec.east) to node [right] {} (P2);

\end{tikzpicture}
}
    \caption{Parental Effects}\label{fig:parental_effects}
  \end{subfigure}
  \caption{Different sources of genetic correlation according to~\citet{Pickrell2016}}\label{fig:genetic_correlation}
\end{figure}

To conclude, during the last few years large gains have been made in fostering our understanding of heritability and genetic correlations with the development of GCTA and LD-score.
While these tools are able to estimate heritability and genetic correlations of a number of traits it remains difficult to distinguish between different sources of genetic correlations.

\section{Association studies on rare variants}
\label{sec:association_studies_on_rare_varitants}

A possible suspect for the missing heritability are rare variants.
Rare variants, commonly variants with a \acrfull{maf} of equal or below 1\%, have been suggested to explain the bulk of missing heritability~\cite{Jiang2013,Li2009a}.
The drop in sequencing costs has allowed to conduct whole-exome and whole genome association studies~\cite{Goodwin2016}.
However, in contrast to GWAS single variants are associates are largely unfeasible, unless sample size is very large~\cite{Lee2014}.
Hence, considerable effort has been made to develop and deploy statistical methods to improve statistical power of rare variant association studies~\cite{Morris2010,Zeng2014,Daye2012,Manuscript2013}.
Hence, instead of testing individual genetic markers most statistical test evaluate the combined effect of multiple genetic variations in a biological relevant region, such as a gene.
In general one can divide such approaches into burden and variance component tests.
I will first introduce the general statistical model for the two tests.
Following I will outline each class of test and evaluate their benefits and drawbacks.

Assuming that $y_i$ for subject $i$ with mean $\mu_i$ follows a distribution in the quasi-likelihood family~\cite{Lee2014} with $n$ subjects in a region with $m$ variants, then
\begin{equation}
  h(\mu_i) = \alpha_0 + \alpha'X_i +\beta'G_i
\end{equation}
in which $h(\mu) = \mu$ or $h(\mu) = logit(\mu)$.
The regression coefficients of the covariants and allele counts are $\alpha = (\alpha_1, \ldots, \alpha_q)$ as well as $\beta = (\beta_1, \ldots, \beta_m)$ respectively.
The covariants are denoted as $X_i = (X_{i1}, \ldots, X_{iq})'$ and the allele counts as $G_{i1}, \ldots, G_{im}$.
The score statistic of the marginal model for variant $j$ is then
\begin{equation}
  S_j = \sum^n_{i=1} G_{ij}(y_i-\hat{\mu_i})
\end{equation}
where $\hat{\mu_i}$ is the estimated mean under $H_0: \beta = 0 $ and is obtained by $h(\mu_i) = \alpha_0 + \alpha'X_i$.

\subsection{Burden Test}
\label{sub:burden_test}
The simplest approach, called the burden test, tests the weighted sum of commutated scores
\begin{equation}\label{eq:burden}
  Q = {(\sum^{m}_{j=1} w_{j} S_{j})}^2
\end{equation}
in which $w_j$ is the weight for variant $j$.
Weights can be define by functional annotations, allele frequency or others.
The test assumes that all variants  have the same direction of effect.
This is a rather strong assumption and violations result in a loss in statistical power~\cite{Derkach2013a}.

\subsection{Variance Component Tests}
\label{sub:variance_component_tests}
In contrast to burden tests variance-component tests evaluate the distribution of effects within a certain genomic region.
The most prominent member of this test family is SKAT~\cite{Wu2011}.
SKAT assume that $\beta_j\sim N(0,w_j\tau)$ and test for $H_0: \tau = 0$ with a variance-component score test.
The test statistic is defined as
\begin{equation}\label{eq:skat}
  Q = \sum^{m}_{j=1} w_{j}^2 S_{j}^2
\end{equation}
The test is robust to groupings of protective and damaging effects within the same region.
However, also SKAT suffer from a reduction in statistical power in cases of high proportions of causal variants with the same direction~\cite{Derkach2013a}.

Therefore, variance-component tests are generally more powerful than burden tests if a region has many non-causal variants or when the effects of a genomic regions are bi-directional.
In contrast burden tests perform better in scenarios where most causal variants have the same direction of effect.
This has led to the development of omnibus tests to combine burden and variance component tests, most notably SKAT-O~\cite{Lee2012}.
\bigskip

To conclude, I have outlined various statistical methods to investigate the genetic contribution of aggressive behavior.
I have described the commonly applied twin model to estimate the contribution of genetic and environmental effects.
Further, I outlined the principles of genome wide association studies as well as the use of GCTA and LD-score to estimate the narrow sense heritability from common SNPs.
At last I have described two common used rare variant tests as well as their pros and cons.

In the next chapter I will present an investigation of the longitudinal heritability of childhood aggression with the use of two large twin cohorts. 
\end{document}
