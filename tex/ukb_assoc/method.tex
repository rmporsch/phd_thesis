\section{Methods}
\label{sec:methods}

\subsection{UK BioBank}
\label{sub:uk_biobank}
The UK BioBank~\cite{Allen2014} is a population based study of people argued 40--69 years old. 
Participants were recruited between 2006 and 2010 via mail invitation (response rate 5.47\%)~\cite{Sudlow2015} and were assessed with the help of self-report questionaires as well as other anthropometirc assessments.
Written informed consent was aquired from all participants and the study was approved by National Research Ethics Service Committee North West–Haydock, UK.

\subsection{Phenotype}
\label{sub:phenotype}

\subsubsection{Impulsive Aggression}
\label{ssub:impulsive_aggression}
Impulsive aggression is officially described as `Ever highly irritable/argumentative for 2 days' (ID:\@ 4653).
The item contains only one single question: 
\begin{displayquote}
  Have you ever had a period of time lasting at least two days when you were so irritable that you found yourself shouting at people or starting fights or arguments?
\end{displayquote}
While the item does not mention aggression directly it can be seen as a simplified measurement of aggressive behavior.
The question has not been validated in correspondence to other measurements of aggression.

\subsubsection{Risk Taking}
\label{ssub:risk_taking}
Risk Taking is also only composed of a single question:
\begin{displayquote}
  Would you describe yourself as someone who takes risks?
\end{displayquote}
Similar to impulsive aggression, this item is a yes or no question and is not a validated instrument.

\subsection{Genetic analysis}
\label{sub:genetic_analysis}
I made use of the imputed genetic data of the UK BioBank, compromising around $\sim73$ million SNPs, short indels and larger structural variations of $152,249$ subjects.
Individuals were genotyped on two different custom genotype arrays, namely the UL BiLEVE as well as the UK Biobank Axiom array from Affymetrix. 
UL BiLEVE was used for most samples ($\sim100,000$), while the array chip from Affymetrix, which was optimsied to support genotype imputation, was used for the remaining participants. 
Quality control and imputation was performed central with the UK BioBank.
However, I further conducted additional quality control.
Variants which whose missing call rate did exceed 10\% as well as those with minor allele frequency below 1\% were exlcluded.
Further variants which did not pass a Hardy-Weinberg equilibrium test threshold of $1e-9$ were also removed from the analysis.
In addition I only included participants with `white European' ancestry identified by a k-mean cluser of the first four principle components of genotype data.
The total number of participants remaining after quality control of geno- and phenotype are displyed in Table~\ref{tab:descriptive_gwas}.

\begin{table}[!htpb]
	\centering
	\caption{Sample size and missingness across Caucasians and non-Caucasians after QC}\label{tab:descriptive_gwas} 
	\resizebox{\textwidth}{!}{%latex.default(samples, title = "", cgroup = c.group, n.cgroup = c(2,     2), table.env = FALSE, file = "./tables/descriptive.tex",     digits = 2)%
\begin{tabular}{llrclr}
\hline\hline
\multicolumn{1}{l}{\bfseries }&\multicolumn{2}{c}{\bfseries All Samples}&\multicolumn{1}{c}{\bfseries }&\multicolumn{2}{c}{\bfseries Caucasian Samples}\tabularnewline
\cline{2-3} \cline{5-6}
\multicolumn{1}{l}{}&\multicolumn{1}{c}{Unaffected/Affected}&\multicolumn{1}{c}{Missingness (in \%)}&\multicolumn{1}{c}{}&\multicolumn{1}{c}{Unaffected/Affected}&\multicolumn{1}{c}{Missingness (in \%)}\tabularnewline
\hline
Risk Taking&107011/39436&$ 3.81$&&86552/29703&$ 3.351$\tabularnewline
Neuroticism&122511&$19.53$&&98086&$18.456$\tabularnewline
Smoking&80984/70678&$ 0.39$&&64115/55853&$ 0.264$\tabularnewline
Impulsive Aggression&40861/9397&$66.99$&&31513/6998&$67.984$\tabularnewline
Drinking&151985&$ 0.17$&&120208&$ 0.065$\tabularnewline
\hline
\end{tabular}
}
\end{table}

Assocation analysis of autosomal SNPs was performed with Plink~\cite{Purcell2007,Chang2015} with age, sex, genotype array chip and the first 10 principle components as covariants.

\subsection{Clumping}
\label{sub:Clumping}

Clumping was used to identify lead SNPs.
Specifically, SNPs within LD of each other ($r^2=0.2$) in a $250kb$ region were grouped together.
The lowest p-value within this window was defined as the lead SNP\@.
Clumped SNPs were used for the computation of polygenic risk scores as well as GWAS Cataloge lookup.

\subsection{GWAS Catalog Lookup}
\label{sub:GWAS_Cataloge_Lookup}

I made use of the GWAS catalog to look up genome wide significant lead SNPs for each GWAS~\cite{Welter2014}.
SNPs in the catalog and lead SNPs in my GWASs were matched if both were within a $250kb$ window and in LD ($\ge 0.2$) of each other (see~\ref{ssub:Clumping}).

\subsection{Polygenic Risk Score}
\label{sub:prs}

Polygenic Risk Scores (PRS) were computed using the computed regression coefficients of all analysed phenotypes.
The more classical approach uses the computed regression coefficients $\beta$ as weights and computes the PRS $\kappa$ for each subject $n$ as
\begin{equation}
	\kappa_n = \sum^j_{j=1} \beta_j x_{nj}
\end{equation}
in which $x_{jn}$ is the $j$th genotype of subject $n$.

However I used \textit{lassosum} a PRS framework, which I co-developed, to estimate polygenic scores~\cite{Mak2016}.
\textit{Lassosum} makes use of a penalized regression to estimate PRS from summary statistics.
The usage on summary statistics alone makes it easier to compute PRS scores for a multitude of phenotypes, especially given the size of the data set.
Further, to keep computational effort manageable, I only used clumped SNPs with a p-value $\leq 0.01$.
