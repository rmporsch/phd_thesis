\section{Discussion}
\label{sec:ukb_assc_discussion}

The aim of this study was to identify specific genetic markers associated with aggression and risk taking,
as well as estimate the genetic overlap across multiple related phenotypes.
Making use of a large population based sample I was able to identify two genetic loci associated with risk taking in chromosome three and six.
Conditional FDR analysis suggested a third independent variant in chromosome three.
Neither the GWAS nor cFDR was able to identify genome wide significant signals in impulsive aggression.
Nevertheless, considerable genetic correlations were identified with a multitude of related phenotypes, including smoking, alcohol consumption, and neuroticism.

Overall this study was able to replicate the results of previous studies on risk taking~\cite{Day2016}, but given the limited available sample size for impulsive aggressive behavior the null findings in this particular phenotype are not surprising.
Interestingly, cFDR analysis suggests a close genetic relationship between risk taking, alcohol consumption and smoking.
In particular, smoking was able to aid the identification of at least one additional SNPs associated with risk taking.
However, the exact function of this SNPs remains unknown and further functional studies are necessary in order to identify potential biological mechanism associated with this locus.

Nevertheless, as predicted, considerable genetic correlations were identified between risk taking and impulsive aggression.
This high genetic overlap supports the suggestion that these two phenotypes are part of a wider sensation seeking personality cluster~\cite{Zuckerman2000} which might have been evolutionary useful in exploring new territories, as well as foraging for food and mates.

However, contrary to a number of phenotypical studies showing a strong causal link between alcohol consumption and aggressive behavior~\cite{FRANZKOWIAX1987,Zuckerman2000,Dakwar2011} estimated genetic correlation between the two phenotypes display a negative direction.
Thus suggesting that genetic components increasing aggressive behavior might also reduce alcohol consumption.
Similar, phenotypical correlations between the two traits are rather small.
These results stand in contrast to studies investigating the environmental connection between aggressive behavior and alcohol consumption~\cite{Bushman1990}.
However, there are multiple possible explanation for this discrepancy.
Foremost, previous studies investigating the causal link between alcohol consumption and aggression were randomized control trials and used only small number of participants.
In addition, most participants were College students.
In contrast the here presented investigation made use of a population based sample of mostly older participants.
Thus could potentially explain the differences between previous studies and the here presented phenotypical results.
Nevertheless, it is still important to note that this would affect the genetic correlations between the two traits to a lesser extend.

It is further surprising that the overall genetic correlation between alcohol consumption and risk taking is relatively low.
This stands in contrast to the considerable inflation within the conditional QQ-plot.
A possible explanation for these discrepancy can be found the application of these two methodologies.
While cFDR only assesses certain SNPs passing a certain threshold, the genetic correlations takes all SNPs into account, hence accounting for a greater proportion of ploygenicity.
An alternative consideration is that the genetic correlation between risk taking and alcohol consumption is not significant, thus suggesting that the overall estimated genetic correlation might be random.
However, this argument stand in contrast to the considerable genetic correlations between impulsive aggression and alcohol consumption, while the cFDR for aggressive behavior displays no indication of an inflation.
Here it is important to note that genetic correlations can arise from a multitude of different factors (as described in Section~\ref{sec:heritability_and_genetic_correlation}).
Assessing the whole genome to estimate a pairwise correlation could potentially be more biased by additional third variables than assessing the overall inflation of SNPs selected under some $p-value$ threshold.
Similar a recent study by~\citet{Shi2016a} showed considerable regional genetic correlations among traits which had neglectable overall correlations.
Thus supporting the argument that the discrepancy between cFDR and genetic correlations is due to the selection of highly significant SNPs.

In addition to the identified genetic overlap of alcohol consumption with risk taking and aggressive behavior the involvement of smoking is also of relevance.
In particular, two genome wide significant genetic variants have, next to their association with risk taking, also associations with spirometric measurements of smokers.
This potential genetic overlap is supported by both cFDR as well as genetic correlations estimates.
These considerable genetic overlaps are surprising given the very limited phenotypical correlations between the smoking and risk taking as well as aggression.
A possible explanation is that smoking behavior is influenced by the ability of an individual to  inhibit temptation.
While this effect might not be strong enough to be observed in the phenotypic data, it does manifest itself by a genetic overlap with both risk taking and aggressive behavior.
Two phenotypes which have been liked to a sensation seeking personality cluster~\cite{Zuckerman2000}.
Thus one could hypothesis that risk taking and aggressive behavior might causally affect smoking status.
Unfortunately, there is currently no experimental data available which could test this hypothesis.

Similar, analysis of the interrelationship between neuroticism and risk taking as well as aggression suggests some genetic overlap.
Specifically, while the genetic correlations between impulsive aggression and neuroticism is high, the correlations between the personality trait and risk taking is not significant, nor of large effect.
This is very much in line with the phenotypic correlations which suggest relatively high correlations between neuroticism and aggression but nearly no correlation with risk taking.
Further, previous studies have shown that neurotic individuals are less likely to engage in risky behavior~\cite{Lauriola2001,InstituteofMedicine2011,Paulus2003}, while subjects with high level of neuroticism are known to be more likely act aggressively towards others~\cite{Meesters2007}.
There here presented genetic evidence support this phenotypical connection and further suggest that neuroticism and aggressive behavior share substantial degree of genetic factors.

Despite the here presented strong evidence for genetic overlap between risk taking, aggression, alcohol consumption, smoking and neuroticism there a few limitations.
Foremost, I was unable to detect a genome wide significant signal in impulsive aggression.
This can be explained by the limited sample size which potentially also affect the analysis of the cFDR\@.
However, the sample size is sufficient to compute genetic correlations across all analysed phenotypes.

While the instrument used to measure both neuroticism, alcohol consumption, risk taking and smoking status have been used in the past, the usage of single item questionnaires to assess aggression is not common.
However, as shown in Chapter~\ref{chap:longHera}, distribution of aggression scores often approach an L-like distribution and can often be dichotomised into either aggressive or not aggressive individuals.
Thus suggesting that the use of a single item to measure aggressive behavior is justified.

\subsection{Conclusion}
\label{sub:conclusion}

This study was able to replicated previous genome wide significant signals in risk taking.
Further, usage of conditional FDR was able to identify an additional locus associated with risk taking.
In addition, I showed considerable genetic overlap between risk taking and aggressive behavior, as well as with other traits.
In particular, between aggression and neuroticism as well as between smoking and risk taking.
Thus suggesting that these traits might be influenced by common genetic factors.
However, the exact causal connection between these traits remains unknown and further studies, possibly with the use of Mendelian randomizations, might be necessary to further understand the interrelationships between these complex human behaviors. 
