\section{Results}
\label{sec:results_ukb_assoc}

\subsection{Phenotype Correlations}
\label{sub:phenotype_correlations}

Phenotypical correlations are displayed in Figure~\ref{fig:corr_pheno}. 
All correlations were significant at $p\leq 0.05$ after adjusting for multiple testing using the Bonferroni correction.
Correlations between risk taking and impulsive aggression is with $r(48,540)=0.09 (p\leq0.0025)$ relatively small
and I was unable to detect larger effects between impulsive aggresion and alcohol consumption ($r(50,220)=-0.05, p\leq0.0025$)
as well as smoking ($r(50,106)=0.1, p\leq0.0025$).
Similar, risk taking displayes only small correlations between
smoking ($r(146,051)=0.00, p\leq0.0025$),
alcohol consumption ($r(146,350)=0.06, p\leq0.0025$)
and neuroticism ($r(120,201)=-0.03, p\leq0.0025$). 
Nevertheless, medium correlations were present between impulsive aggression and neuroticisim ($r(41,420)=0.33, p\leq0.0025$).

\begin{figure}[htpb]
  \centering
  \includegraphics[width=0.6\linewidth]{ukb_assoc/figure/phenotype/corr_plot_ci.pdf} 
  \caption{
    Phenotypical correlations among analysed phenotypes.
    Displayed correlations were all significant after adjusting for multiple testing using the Bonferroni correction.
    In contrast to the genetic correlations, Caucasion and non-Caucasian subjects were used.
  }\label{fig:corr_pheno}
\end{figure}

\subsection{GWAS on Impulsive Aggression}
\label{sub:impulsive_aggression_gwas}

Genome wide assocation analysis reveled no genome wide significant loci in $37,320$ subjects (see Figure~\ref{fig:gwas_impAgg}).
Observed SNP heritability was estiamted at $5.32\% (SE=0.012)$ and are in line with previous estiamted SNPs heritabilities in behavioral traits.

\begin{figure}[!htpb]
  \begin{subfigure}[t]{.5\textwidth}
    \centering
    \includegraphics[width=0.8\linewidth]{{ukb_assoc/figure/qq_plots/qq_plot_4526}.jpeg}
    \caption{QQ-plot for Impulsive aggression}\label{fig:ImpAgqq}
  \end{subfigure}
  \begin{subfigure}[t]{.5\textwidth}
    \centering
    \includegraphics[width=0.8\linewidth]{{ukb_assoc/figure/manhatten_plots/manhatten_plot_4526}.jpeg}
    \caption{Manhattan plot for Impulsive aggression}\label{fig:ImpAgmanhatten}
  \end{subfigure}
  \caption{
    Manhattan and QQ Plot for Impulsive aggression.
    No genome wide significant SNPs were identified in the given sample.\label{fig:gwas_impAgg}}
\end{figure}

\subsection{GWAS on Risk Taking}
\label{sub:risk_taking_gwas}

The genome wide association study on risk taking yield two genome wide significant loci in $120,286$ subjects in chromosome 3 and 6.
Total observed SNP heritability is with $5.52\% (SE=0.0052)$ similar to that of impulsive aggression.
The ratio between the LD-score regression intercept $\beta_0$ and the mean of the $\chi^2$-test statistics ($(\beta_0 - 1)/(mean(\chi^2)-1)$),
which indcates the proportion of inflation to factors other than polygenic heritability, is with $0.0422$ ($SE=0.052$) low.
In addition, $\beta_0$ approaches 1 ($\beta_0=1.0082, SE=0.0056$) thus suggesting no population stratification (see Figure~\ref{fig:riskqq}).

\begin{figure}[!htpb]
	\begin{subfigure}{.5\textwidth}
		\centering
		\includegraphics[width=0.8\linewidth]{{ukb_assoc/figure/qq_plots/qq_plot_2040}.jpeg}
		\caption{QQ-Plot of risk taking.}\label{fig:riskqq}
	\end{subfigure}
	\begin{subfigure}{.5\textwidth}
		\centering
		\includegraphics[width=0.8\linewidth]{{ukb_assoc/figure/manhatten_plots/manhatten_plot_2040}.jpeg}
		\caption{
      Manhattan plot of risk taking.
      The horizontal like indicate genome wide significance.
    }\label{fig:riskmanhatten}
	\end{subfigure}
  \caption{QQ and Manhattan plot for Risk Taking. 
    Two seperate genome wide significant loci were identfied, while a thrid is marginal significant.\label{fig:risktaking_gwas}}
\end{figure}

Closer inspection of the signal in chromosome 3 shows 2 lead SNPs namly \textit{rs1308431} and \textit{rs7639518} (see Table~\ref{tab:lead_snps_risk}).
However, \textit{rs1308431} and \textit{rs7639518} have opposite directions of effect.
The independency of these two signals was investigated using a conditional analysis.
Thus, genotypes of \textit{rs7639518} were regressed on the phenotype while adjusting for \textit{rs1308431}.
This conditional model indicates that \textit{rs7639518} is not independent of \textit{rs1308431} at a genome wide level ($OR=0.957, t(107737)=-3.787, p=0.0001523$).

\paragraph{rs9379971}
\label{par:rs9379971}
This particular variant has the strongest association among all tested SNPs with a p-value of $2.14e-09$ (see Figure~\ref{fig:rs9379971}). 
The SNP is an intronic variant, without any mentioning in the GWAS catalog~\cite{Welter2014}.
The SNPs has been associated with \textit{POM121L2} and the region has been connected with schizophrenia across multiple studies~\cite{Aberg2013,Shi2009}.
However, the function of \textit{POM121L2} is not well understood.

\paragraph{rs13084531}
\label{par:rs13084531}
This SNP has a comparable p-value ($9.826e-09$) and is an intron variant within the gene \textit{CADM2}.
The gene has been associated with BMI~\cite{Speliotes2010} as well as executive functions and processing speed~\cite{Ibrahim-Verbaas2015}.
Interestingly the SNP associated in the study by~\cite{Ibrahim-Verbaas2015} (\textit{rs17518584}) is in LD with \textit{rs13084531} ($r^2=0.4951;D'=0.9983$).
Hence suggesting a common loci for executive functions and risk taking.
Further associated region (3p12.1) has been connected to spirometric measures in smokers~\cite{Lutz2015}.

\paragraph{rs4386663}
\label{par:rs4386663}
This SNP is an intergenic variant with no known association to regulatory functions or genes.
However, the SNP is only marginal significant ($6.059e-08$).
Interestingly the chromosomal region of this SNP (chr4q12) was also associated with spirometric measures in smokers by the same study as in variant \textit{rs13084531} (see paragraph~\ref{par:rs13084531}).

\begin{table}
	\small
	\centering
	%latex.default(dat, title = "", file = paste0(outputfolder, "lead_snp_",     nameID, ".tex"), digits = 3, rowname = NULL, table.env = F)%
\begin{tabular}{rlrlrrrr}
\hline\hline
\multicolumn{1}{c}{CHR}&\multicolumn{1}{c}{SNP}&\multicolumn{1}{c}{BP}&\multicolumn{1}{c}{A1}&\multicolumn{1}{c}{N}&\multicolumn{1}{c}{OR}&\multicolumn{1}{c}{STAT}&\multicolumn{1}{c}{P}\tabularnewline
\hline
$3$ & rs13084531 & $85553994$ & G & $115264$ & $0.936$ & $5.73$ & $9.83e-09$\tabularnewline
$6$ & rs9379971  & $27259308$ & T & $109344$ & $1.065$ & $ 5.99$ & $2.14e-09$\tabularnewline
\hline
\end{tabular}

  \caption{
    Lead SNPs reaching genome wide significance in Risk Taking.
    SNPS are listed by chromosome (CHR) and position (BP).
  }\label{tab:lead_snps_risk}
\end{table}

\subsection{Conditional FDR}
\label{sub:conditional_fdr}

Figure~\ref{fig:cFDR} shows the conditional QQ plots for both aggression and risk taking for each of the chosen thresholds and conditional phenotyes.
Both aggression and risk taking, conditional on each other did not result in any noticiable pleiotropic effect.
However, while aggression conditional on neuroticisim did not result in an inflation within the conditional QQ plots, risk taking was noticiable inflated conditional on SNPs which reached a relatively low p-value cutt-off.
This was also the case for both smoking and alcohol consumption.
Interestingly, infaltion of risk taking SNPs conditional on neuroticism was not persistent across the chosen thresholds.
In particular, while considerale infaltion could be observed at $p\leq0.1$, at lower p-value threshold this effect was markly reduced.
In contrary, the inflation causased conditional on alcohol consumption and smoking was persistent across the three tresholds.

As already indicated in Figure~\ref{fig:cFDR}, impulsive aggresion conditional on the four reminaing phenotyes did not result in any loci with an $cFDR\leq0.01$.
Nevertheless, cFDR identified a single independent loci, rs570682061, in risk taking which passes both $cFDR\leq0.01$ as well replication in an independent sample (Table~\ref{tab:cFDR}).
This particular SNPs is independent of the genome wide significant loci identfied unconditional on other phenotypes.
This locus has not been associated with any known gene or phenotype.

\begin{figure}[!htpb]
  \centering
	\begin{subfigure}{1\textwidth}
		\centering
    \includegraphics[width=1\linewidth]{ukb_assoc/figure/cFDR/agg_cond.jpeg}
	\end{subfigure}
	\begin{subfigure}{1\textwidth}
		\centering
    \includegraphics[width=1\linewidth]{ukb_assoc/figure/cFDR/risk_cond.jpeg}
	\end{subfigure}
  \caption{
    Conditional QQ plots for both risk taking and aggression. 
    Figure a-c represents the conditional QQ-plots for impulsive aggression,
    conditional on the remaining phenotypes given threshold $p\leq0.1$, $p\leq0.01$, $p\leq0.001$ respectively.
    Similar, Figure d-f are the conditional QQ-plots for risk taking, conditional on the remaining phenotyes across the three thresholds.\label{fig:cFDR}}
\end{figure}


\begin{table}[htpb]
  %latex.default(dat, title = "", file = "risk_cFDR.tex", cellTexCmds = cell.format,     numeric.dollar = FALSE, digits = 3, rowname = NULL, table.env = F)%
\begin{center}
\begin{tabular}{rlrrlllr}
\hline\hline
\multicolumn{1}{c}{CHR}&\multicolumn{1}{c}{SNP}&\multicolumn{1}{c}{cFDR}&\multicolumn{1}{c}{Z}&\multicolumn{1}{c}{A1}&\multicolumn{1}{c}{A2}&\multicolumn{1}{c}{$\delta$}&\multicolumn{1}{c}{$Z_{\delta}$}\tabularnewline
\hline
    1&   rs1912231&   5.25e-03&    4.34&   C&   T&   Alcohol&    3.62\tabularnewline
    3&   rs9870448&   2.77e-04&   -5.24&   A&   G&   Alcohol&   -3.89\tabularnewline
\bfseries    3&\bfseries   rs570682061&\bfseries   6.64e-05&\bfseries    5.22&\bfseries   A&\bfseries   AT&\bfseries   Smoking&\bfseries    3.67\tabularnewline
    6&   rs7744605&   7.52e-05&    5.16&   C&   A&   Smoking&    3.27\tabularnewline
    6&   rs9468372&   2.06e-04&    4.88&   T&   A&   Smoking&    3.42\tabularnewline
    8&   rs1968400&   9.86e-03&    3.67&   G&   C&   Smoking&    3.23\tabularnewline
    8&   rs116807689&   2.62e-03&    4.16&   A&   G&   Smoking&    3.73\tabularnewline
   10&   rs67657945&   2.13e-03&   -4.23&   CT&   C&   Smoking&   -3.21\tabularnewline
\hline
\end{tabular}\end{center}

  \caption{
    Independent loci ($r^2 < 0.05$) with $cFDR\leq0.01$.
    SNPs are listed by location (CHR) and conditional phenotype ($\delta$).
    Z-scores are indicated for both risk taking and the conditional phenotye.
    Data was adjusted for genomic inflation.
    SNPs in bold indicate replication in an independent sample.
  }\label{tab:cFDR}
\end{table}

\subsection{Genetic Correlations}
\label{sub:genetic_correlations_internal}

Figure~\ref{fig:gcor} displays the genetic correlations of analysed phenotyes.
All correlations were significant after adjusting for multiple testing ($p\leq0.005$).
The largest genetic correlations could be observed between impulsive aggression and neuroticism ($r_g=0.63, SE=0.083, p=4.6496e-14$), 
followed by a consinderable correlation between risk taking and aggresion ($r_g=0.44, SE=0.1039, p=2.6656e-5$).
Interestingly, relatively large negative correlations are also present between alcohol consumption and aggression ($r_g=-0.39, SE=0.0892, p=1.6942e-5$)
as well as between smoking and aggression ($r_g=0.4, SE=0.0716, p=1.6442e-08$).
Correlation, between smoking and alcohol consumption is not significant as well of neglectable effect ($r_g=0.06, SE=0.044628, p=0.15502$).
Interestingly, these respectively high genetic correlations are not reflected in the corresponding conditional FDR (see Figure~\ref{fig:cFDR}).

Further, risk taking shows non-significant and small correlation with neuroticism ($r_g=-0.12, SE=0.0738, p=0.10629$) and
alcohol consumption ($r_g=0.07, SE=0.051506, p=0.19604$) but the correation between risk taking and smoking ($r_g=0.34, SE=0.0514, p=2.4044E-11$) is considerable.
In addition, medium correlations were present between neuroticism and alcohol consumption ($r_g=-0.17, SE=0.0406, p=2.6366e-4$). 
Correlations were also present between neuroticism and smoking, but the effect did not pass multiple testing ($r_g=0.16, SE=0.0662, p=0.017786$).

\begin{figure}[!h]
	\centering
  \includegraphics[width=0.8\linewidth]{ukb_assoc/figure/genetic_corr/gcorr_plot_circle_full_se.pdf}
  \caption{Genetic Correlations.
    Pairwise genetic correlations across analysed phenotypes.
    The lower triangular matrix indicate the numeric correlations, while the upper part is a visual representation of the strengths and direction of the correlation.
    A cross indicate a non-significant correlation after adjusting for multiple testing.
  }\label{fig:gcor}
\end{figure}


