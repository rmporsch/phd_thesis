\section{Results}
\label{sec:results}

The result section of this chapter is structured into two separate parts.
First I will present the result of the GWAS on impulsive aggression.
This is followed up by the presentation of the GWAS result on risk taking.
At last I will look at genetic correlations between phenotypes as well as PRS and heritability estiamtions for each of the phenotypes.

\subsection{Impulsive Aggression}
\label{sub:impulsive_aggression_gwas}

No significant hits were identified in $\sim37,320$ subjects (see Figure~\ref{fig:gwas_impAgg}).
This is unfortunate but not surprising giving the low number of individuals.
Further no population stratification has been identified (Table~\ref{tab:LDscImpAgg}) and SNP heritability is relatively low.

\begin{figure}[!h]
  \begin{subfigure}{.5\textwidth}
    \centering
    \includegraphics[width=0.8\linewidth]{{ukb_assoc/figure/qq_plots/qq_plot_4526}.jpeg}
    \caption{QQ-plot for Impulsive aggression}\label{fig:ImpAgqq}
  \end{subfigure}
  \begin{subfigure}{.5\textwidth}
    \centering
    \includegraphics[width=0.8\linewidth]{{ukb_assoc/figure/manhatten_plots/manhatten_plot_4526}.jpeg}
    \caption{Manhattan plot for Impulsive aggression}\label{fig:ImpAgmanhatten}
  \end{subfigure}
  \caption{GWAS on Impulsive aggression\label{fig:gwas_impAgg}}
\end{figure}

\begin{table}[!htpb]
	\centering
	%latex.default(out, file = paste0(outputfolder, "ldscore_results_",     variable_name, ".tex"), table.env = F, rowname = NULL)%
\begin{center}
\begin{tabular}{llll}
\hline\hline
\multicolumn{1}{c}{Total Observed scale h2}&\multicolumn{1}{c}{Lambda GC}&\multicolumn{1}{c}{Intercept}&\multicolumn{1}{c}{Ratio}\tabularnewline
\hline
0.0532 (0.012)&1.0496&1.0082 (0.0056)&0.1693 (0.1163)\tabularnewline
\hline
\end{tabular}\end{center}

	\caption{LDsc results for Impulsive aggression}\label{tab:LDscImpAgg}
\end{table}

\subsection{Risk Taking}
\label{sub:risk_taking_gwas}

In contrast to impulsive aggression, risk taking has a considerable sample size.
In total there were $120,286$ subjects available with a complete phenotype.
The corresponding QQ and Manhattan plot are displayed in figure~\ref{fig:risktaking_gwas}, and the lead SNPs are shown in table~\ref{tab:lead_snps_risk}.
As mentioned above, lead SNPs were identified using clumping.
LD-score regression indicates no major population stratification since both intercept and ratio approaches $0$ (see table~\ref{tab:ldscrisk}).

The Manhattan plot in figure~\ref{fig:riskmanhatten} shows two clear signal in chromosome 3 and 6, as well as a signal which is just below genome wide significant at chromosome 4.
A closer inspection of the signal in chromosome 3 shows 2 lead SNPs (see table~\ref{tab:lead_snps_risk}).
However, \textit{rs1308431} and \textit{rs7639518} have opposite directions of effect.
In order to investigate if the signals are independent of each other I performed a conditional analysis.
Using the top SNP (\textit{rs13084531}) as a covariate, the analysis showed no independent genome-wide significant effect.
Thus suggesting that \textit{rs7639518} is not independent of \textit{rs13084531}.
However, a closer look at the conditional regression model suggest that the \textit{rs7639518} might be a potentially independent signal (see table~\ref{tab:conditional}) since the resulting p-value is relatively low. 

\begin{figure}[!h]
	\caption{GWAS on Risk Taking}
	\label{fig:risktaking_gwas}
	\begin{subfigure}{.5\textwidth}
		\centering
		\caption{QQ-Plot of risk taking}
		\label{fig:riskqq}
		\includegraphics[width=0.8\linewidth]{{ukb_assoc/figure/qq_plots/qq_plot_2040}.jpeg}
	\end{subfigure}
	\begin{subfigure}{.5\textwidth}
		\centering
		\caption{Manhattan plot of risk taking}
		\label{fig:riskmanhatten}
		\includegraphics[width=0.8\linewidth]{{ukb_assoc/figure/manhatten_plots/manhatten_plot_2040}.jpeg}
	\end{subfigure}
\end{figure}

Below I will describe the three identified regions of interest (the two genome wide significant signals as well as the one approaching significance).
I also performed a quick literature review as well as looked up the three SNPs in the GWAS catalog (see table~\ref{tab:gwas_risk_catalog})

\paragraph{rs9379971}
\label{par:rs9379971}
This particular variant has the strongest association among all tested SNPs with a p-value of $2.14e-09$. 
The SNP is an intronic variant, without any mentioning in the GWAS catalog~\cite{Welter2014}.
As you can see in the LocusZoom plot (figure~\ref{fig:rs9379971}) the region round the identified SNPs is rather dense. 
The SNPs has been associated with \textit{POM121L2} and the region has been connected with schizophrenia across multiple studies~\cite{Aberg2013,Shi2009}.
However, the function of \textit{POM121L2} is not well understood.

\paragraph{rs13084531}
\label{par:rs13084531}
As you can see in figure~\ref{fig:rs13084531} the variant is in a relatively large LD region with a comparable p-value to the previous SNP ($9.826e-09$).
The SNPs is an intron variant within the gene \textit{CADM2}.
The gene has been associated with BMI~\cite{Speliotes2010} as well as executive functions and processing speed~\cite{Ibrahim-Verbaas2015}.
Interestingly the SNP associated in the study by~\cite{Ibrahim-Verbaas2015} (rs17518584) is in LD to our loci (rs13084531 with $R^2=0.4951;D'=0.9983$).
Hence suggesting a common loci for executive functions and risk taking.
Further the region of this SNP (3p12.1) has been connected to spirometric measures in smokers~\cite{Lutz2015}.

\paragraph{rs4386663}
\label{par:rs4386663}
This SNP is an intergenic variant with no known association with any regulatory functions.
However, the SNP is only marginal significant ($6.059e-08$).
Interestingly, however, the chromosomal region of this SNP (chr4q12) was also associated with spirometric measures in smokers by the same study as in variant rs13084531 (see paragraph~\ref{par:rs13084531}).
The LocusZoom plot (see figure~\ref{fig:rs13084531}) suggests a relatively narrow LD region. 

\begin{table}
	\small
	\centering
	\caption{Lead SNPs reaching genome wide significance in Risk Taking}\label{tab:lead_snps_risk}
	%latex.default(dat, title = "", file = paste0(outputfolder, "lead_snp_",     nameID, ".tex"), digits = 3, rowname = NULL, table.env = F)%
\begin{tabular}{rlrlrrrr}
\hline\hline
\multicolumn{1}{c}{CHR}&\multicolumn{1}{c}{SNP}&\multicolumn{1}{c}{BP}&\multicolumn{1}{c}{A1}&\multicolumn{1}{c}{N}&\multicolumn{1}{c}{OR}&\multicolumn{1}{c}{STAT}&\multicolumn{1}{c}{P}\tabularnewline
\hline
$3$ & rs13084531 & $85553994$ & G & $115264$ & $0.936$ & $5.73$ & $9.83e-09$\tabularnewline
$6$ & rs9379971  & $27259308$ & T & $109344$ & $1.065$ & $ 5.99$ & $2.14e-09$\tabularnewline
\hline
\end{tabular}

\end{table}

\begin{table}[!htpb]
	\centering
	\caption{Additional SNP catalog look up for Risk Taking}\label{tab:gwas_risk_catalog}
	\resizebox{\textwidth}{!}{%latex.default(dat, title = "", file = paste0(outputfolder, "gwas_catalog_",     nameID, ".tex"), digits = 3, rowname = NULL, table.env = F)%
\begin{tabular}{lrlrlrl}
\hline\hline
\multicolumn{1}{c}{Lead SNP}&\multicolumn{1}{c}{P}&\multicolumn{1}{c}{Study SNP}&\multicolumn{1}{c}{Study P}&\multicolumn{1}{c}{Disease Trait}&\multicolumn{1}{c}{PubMedID}&\multicolumn{1}{c}{Direction}\tabularnewline
\hline
rs13084531&$9.83e-09$&rs9841144&$9e-07$&Longevity (90 years and older)&$25199915$&yes\tabularnewline
rs13084531&$9.83e-09$&rs13323436&$3e-06$&Visceral fat&$22589738$&\tabularnewline
rs9379971&$2.14e-09$&rs16897515&$4e-07$&Schizophrenia&$23894747$&no\tabularnewline
rs9379971&$2.14e-09$&rs6932590&$1e-12$&Schizophrenia&$19571808$&yes\tabularnewline
\hline
\end{tabular}
}
\end{table}

\begin{table}[!htpb]
	\small
	\centering
	\caption{Risk Taking with \textit{rs13084531} as covariate for selected SNP}\label{tab:conditional}
	\resizebox{\textwidth}{!}{%latex.default(dat, title = "", file = "conditional_analysis_rs13084531.tex",     table.env = F, col.names = F)%
\begin{tabular}{lrlrlrrrr}
\hline\hline
\multicolumn{1}{l}{}&\multicolumn{1}{c}{CHR}&\multicolumn{1}{c}{SNP}&\multicolumn{1}{c}{BP}&\multicolumn{1}{c}{A1}&\multicolumn{1}{c}{NMISS}&\multicolumn{1}{c}{OR}&\multicolumn{1}{c}{STAT}&\multicolumn{1}{c}{P}\tabularnewline
\hline
2&$3$&rs76395182&$85547337$&G&$107738$&$0.957$&$-3.787$&$0.0001523$\tabularnewline
\hline
\end{tabular}
}
\end{table}

\begin{table}[!htpb]
	\centering
	\caption{LDsc results for risk taking}\label{tab:ldscrisk}
	%latex.default(out, file = paste0(outputfolder, "ldscore_results_",     variable_name, ".tex"), table.env = F, rowname = NULL)%
\begin{center}
\begin{tabular}{llll}
\hline\hline
\multicolumn{1}{c}{Total Observed scale h2}&\multicolumn{1}{c}{Lambda GC}&\multicolumn{1}{c}{Intercept}&\multicolumn{1}{c}{Ratio}\tabularnewline
\hline
0.0552 (0.0052)&1.127&1.0056 (0.0069)&0.0422 (0.052)\tabularnewline
\hline
\end{tabular}\end{center}

\end{table}

\begin{figure}[!htpb]
	\centering
	\includegraphics[width=0.8\linewidth, page=1]{{ukb_assoc/figure/locuszoom/chr3_85055567-86052757}.pdf}
	\caption{LocusZoom Plot for risk taking around rs13084531}\label{fig:rs13084531}
\end{figure}

\begin{figure}[!htpb]
	\centering
	\includegraphics[width=0.8\linewidth, page=1]{{ukb_assoc/figure/locuszoom//chr6_26759903-27759115}.pdf}
	\caption{LocusZoom Plot for risk taking around rs9379971}\label{fig:rs9379971}
\end{figure}

\begin{figure}[!htpb]
	\centering
	\includegraphics[width=0.8\linewidth, page=1]{{ukb_assoc/figure/locuszoom/chr4_58477059-59476876}.pdf}
	\caption{LocusZoom Plot for risk taking around rs4386663}\label{fig:rs4386663}
\end{figure}
