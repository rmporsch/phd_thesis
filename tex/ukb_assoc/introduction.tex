\section{Introduction}
\label{sec:introduction}

As described in Chapter~\ref{chap:longHera}, aggressive behavior is influenced by genetic factors to a considerable degree.
However, while twin studies are able to estimate the overall influence of genetic and environmental factors on a phenotype, these studies are unable to identify specific genetic loci.
The identification of specific genetic markers can foster our knowledge of the biological components involved in aggressive behavior. 
This chapter will describe and present the result of two Genome Wide Association Studies (see Section~\ref{sec:descriptive_gwas}).
Namely impulsive aggression and risk taking.

As outlined in Section~\ref{sec:evolutionary_theories}, aggressive behavior is associated with a significant risk for ones own well-being.
An aggressive action could result in bodily harm, death and reduced survival and reproductive fitness and is therefore an especially risky behavior.
Thus it is not surprising that risk taking and aggression have been shown to be closely linked in previous studies.
For example, a study of rural middle school children found that higher level of risk taking was  predictive for general acceptance of aggression as well as aggressive behavior in general~\cite{Swaim2004}.
In another example,~\citet{Deffenbacher2003} showed that high anger drivers engaged in more risky behavior than low anger drivers.
Further, \citet{Zuckerman2000} suggested that certain personality characterises are associated with risk taking, namely impulsive sensation seeking, aggression and sociability, are part of a evolutionary based sensation seeking behavior which facilitated risky behavior.
These behaviors could support the exploration of new territories, and foraging for food and mates.
Therefore, provide an potential evolutionary advantage.
However, these behaviors can also result in maladaptive adaptations within our modern times, such as reckless driving or drug abuse, as well as aggression.
Unfortunately, while there is considerable literature regarding the heritability of both risk taking and aggression (see Chapter~\ref{chap:longHera}) there is no study investigating the genetic correlations between the two phenotypes.
Thus it remains unclear to which extend risk taking and aggression might share specific functional pathways or how close these two phenotypes are genetically.

Previously testosterone~\cite{Vermeersch2008} has been suggested to influence both risk taking and aggressive behavior since both behaviors are disproportional present in male~\cite{Byrnes1999}.
However, recent meta-analysis have shown that  the influence of this sex hormone is unclear and there is currently no evidence that the hormone affects either aggression~\cite{Archer2005a} or risk taking~\cite{Vermeersch2008}.

Overall heritability estimates are rather similar between risk taking~\cite{Anokhin2009} and aggression (see Chapter~\ref{chap:longHera})
However, there is some indication that the proportion of variance explained by genetic factors might depend on age in risk taking~\cite{Anokhin2009} but not in aggression~\cite{Porsch2016}.
In a study on 12 and 14 year old twin pairs who took part in risk taking task showed only small genetic effects at age 12, but considerable genetic influence at age 14 (50\%)~\cite{Anokhin2009}.
This is in contrast to the study presented in Chapter~\ref{chap:longHera} which reported stable influence of genetic factors across three different age groups.
However the study by~\citet{Anokhin2009} used only 169 MZ and 203 DZ twins across the two age groups.
These low sample sizes could considerable impair the detection of common environment and genetic effects. 
Hence it is unclear if heritability estimates remain stable across age groups in risk taking.

In addition to previous efforts to estimate heritability of aggression and risk taking there have been a number of studies aimed to identify specific genetic loci associated with aggression and risk taking in GWAS\@.
Specifically there have been a number of GWASs on aggressive behavior~\cite{Fernandez-Castillo2016}.
However, most studies have little statistical power and were unable to identify genome wide associated loci.
In one of the most comprehensive meta-analysis of aggressive behavior~\citet{Vassos2014} the authors found no genome wide significant signal.
However, studies included in this meta-analysis showed a considerable degree of heterogeneity in the phenotype.
Specifically, the study included psychiatric as well as population based samples, hence making it difficult to identify loci specific signals.
Indeed, in their review of association studies in aggressive behavior,~\citet{Fernandez-Castillo2016} found that most hypothesis free approaches lacked sample size, assessed often very heterogeneous phenotypes and did not differentiated between direct and indirect aggression. 
Thus explaining the lack of genome wide associations in aggressive behavior.

In contrast to aggression, a very recent study was able to identify a significant genome wide associated signal in risk taking~\cite{Day2016}. 
In this analysis of a large population based sample the authors identified a genome wide association in \textit{CADM2}.
These results were also confirmed in a separate study on the same dataset~\cite{Boutwell2017} which also showed that \textit{CADM2} is associated with a number of other behavioral phenotypes.
However, no analysis has been done in consideration of the overall genetic relationship between risk taking and impulsive aggression with other related behavioral phenotypes.

Indeed, while the phenotypic relationship between risk taking and aggressive behavior with alcohol consumption and smoking is well known~\cite{FRANZKOWIAX1987,Zuckerman2000,Dakwar2011}.
The genetic overlap across these different phenotypes are not understood.
For example, numerous studies have suggested a causal link between alcohol consumption and aggression (see~\citet{Bushman1990} for a review), but there are no studies investigating potential non-environmental causal pathways.

Similar, the causal connection between alcohol consumption and risk taking is well established~\cite{Lane2004}, but potential shared genetic pathways remain not well understood.
An exception is the study by~\cite{Kogan2010} who showed an interaction between \textit{5-HTTLPR} and alcohol consumption which affected the onset of sexual behavior. 
However, the sample size is with 185 participants relatively low, thus it is difficult to draw a conclusion about a potential gene-environment interaction affecting risk taking behavior~\cite{Rubens2016}.

While little research has been conducted on the genetic overlap of different phenotypes related to aggressive behavior studies have investigated the genetic correlations among subtypes of aggressive behavior~\cite{Tuvblad2011a} as well as risk taking and age of first sexual intercourse~\cite{Day2016}.
However, the genetic correlations between impulsive aggression, risk taking and other impulse control behaviors, such as drinking and smoking, are unknown.
A better understanding of these genetic interrelationships could provide valuable inside into shared molecular pathways of these complex human behaviors.

At last a number of personality measurements have been associated with risk taking and aggressive behavior~\cite{Anderson2002a}.
Most prominently, Neuroticism.
Neurotic individuals have been found to express higher levels of anger, as well as other negative emotions such as anxiety, sadness and guilt~\cite{Watson1984}.
Specifically, Neuroticism is linked to coping strategies which increase interpersonal conflicts~\cite{Bolger1991,Ode2008}
Further, neurotic individuals are more reactive to negative stressors therefore resulting in higher levels of expressed anger and aggression~\cite{Ode2008}.
Similar a number of other studies reported a lower level of risk taking in subjects with higher degree of Neuroticism~\cite{Lauriola2001,InstituteofMedicine2011,Paulus2003}.
These phenotypic findings would suggest a considerable genetic overlap between risk taking, aggression and Neuroticism.
However, currently there is no study addressing the potential genetic overlap of Neuroticism, aggression and risk taking.

In conclusion, while there have been a number of studies conducted to identify specific genetic loci associated  with either risk taking and aggression, only one genome wide significant signal could be identified for risk taking.
Attempts to identify specific molecular markers with aggression were handicapped by low sample size and high phenotypic heterogeneity.
Further, there are currently no studies looking at the genetic overlap between aggression, risk taking and related phenotypes.
Specifically, alcohol consumption, tobacco usage, and Neuroticism.

Within this study I will make use of the UK BioBank and aim to identify specific genetic loci associated with aggression and risk taking.
In addition, I will estimate the genetic correlations between alcohol consumption, risk taking, aggression, Neuroticism, as well as smoking behaviors in order to foster our understanding of the relationship between these complex human behaviors.
