\section{Introduction}
\label{sec:introduction}

As described in Chapter~\ref{chap:longHera}, aggressive behavior is influenced by genetic factors to a considerable degree.
However, twin studies are unable to identidy specific genetic loci associated with this phenotype of interest.
Hence, this chapter will describe and present the result of Genome Wide Assocation Studies (see Section~\ref{sec:descriptive_gwas}) on two related phenotypes, namly impulsive aggression and risk taking.

As outlined in Section~\ref{sec:evolutionary_theories}, aggressive behavior is associated with a significant risk for ones own well-being.
An aggressive action could result in bodily harm, death and reduced survivial and reproductive fitness and is therefore an especially risky behavior.
Thus it is not surprising that risk taking and aggression have been shown to be closly linked in previous studies.
For example, a study of rural middle school children found that higher level of risk taking was  predictive for general acceptance of aggression as well as aggressive behavior in general~\cite{Swaim2004}.
In another example,~\citet{Deffenbacher2003} showed that high anger drivers engaged in more risky behavior than low anger drivers.
\citet{Zuckerman2000} suggested that certain personaltiy charactieriscs are assocated with risk taking, namly impuslive sensation seeking, aggression and sociability, are part of a evolutionary based sensation seeking behavior which facilitated risky behavior.
These behaviors could supporting the exploration of new territories, and foraging for food and mates.
However, this behavior can also result in maladaptive behavior within our modern times, such as reckless driving or drug abuse, as well as aggression.
Unfortunatly, while there is consinderable literature regarding the heritability of both risk taking and aggression (see Chapter~\ref{chap:longHera}) there is no study investiagting the genetic correlations between the two phenotypes.
Thus it remains unclear to which extend risk taking and aggression might share spcific functional pathways.

Previously testosteron has been suggested to influece both risk taking ang aggressive behavior.
Similar to aggression, risk taking is disproportional present in male~\cite{Byrnes1999} so the influence of sex hormones on level of risk taking had been suspected~\cite{Vermeersch2008}.
However, recent meta-analysis have shown that  the influence of testosteron on both phenotyes is unclear and there is currently no evidence that the hormon affects either aggression~\cite{Archer2005a} and risk taking~\cite{Vermeersch2008}.

Nevertheless, while heritability estiamtes are rather similar between risk taking~\cite{Anokhin2009} and aggression (see Chapter~\ref{chap:longHera}) stability of these estimes seem to change by age in risk taking, but not in aggression.
In a study on 12 and 14 year old twin pairs who took part in risk taking task showed only small genetic effects at age 12, but consinderable genetic influence at age 14 (50\%)~\cite{Anokhin2009}.
This is in contrast to the study presented in Chapter~\ref{chap:longHera} which reported stable influece of genetic factos actross three different age groups.
However the study by~\citet{Anokhin2009} used only 169 MZ and 203 DZ twins across the two age groups.
These low sample sizes could consinderable impair the detection of common enviorment and genetic effects. 
Hence it is unclear if heritability estiamtes remain stable across age groups in risk taking.

In addition to previous efforts to estiamte heritability of aggression and risk taking there have been a number of studies aimed to identify specific genetic loci associated with aggression and risk taking in GWAS\@.
Specifically there have been a number of GWASs on aggressive behavior~\cite{Fernandez-Castillo2016}, but most studies have little statistical power and were unable to identfy genome wide associated loci.
In one of the most comprehensive meta-analysis of aggressive behavior~\citet{Vassos2014} found no genome wide significant signal.
However, studies included in this meta-analysis showed a considerable degree of heterogeneity in the phenotype.
Specifically, the study included psychiatric as well as population based samples, hence making it difficult to identify loci specific signals.
Ideed, in their review of association studies in aggressive behavior,~\citet{Fernandez-Castillo2016} found that most hypothesis free approaches lacked sample size, assessed often very heterogenous phenotypes and did not differentiated between direct and indirect aggression. 
Thus explaining the lack of genome wide associations in aggressive behavior.

In contrast to aggression, a very recent study was able to identify a significant genome wide assocated signal in risk taking~\cite{Day2016}. 
In this analysis of a large population based sample the authors identified a genome wide association in \textit{CADM2}.
These results were also confirmed in a seperate study on the same dataset~\cite{Boutwell2017} which also showed that \textit{CADM2} is associated with a number of other behavioral phenotypes.
However, no analysis has been done in considerdation of the overall genetic relatioship between risk taking and impulsive aggression with other related behavioral phenotypes.

While the phenotypic relationship between risk taking and aggressive behavior with alcohol consumption and smoking is well known~\cite{FRANZKOWIAX1987,Zuckerman2000,Dakwar2011}.
the genetic relatioship overlap between the two phenotypes are not understood.
For example, numerous studies have suggested a causal link between alcohol consumption and aggression (see~\citet{Bushman1990} for a review), but there are no studies investigating potential non-enviormental causal pathways.
Specifically, while most studies investigated the causal effect of alcohol consumption on aggression via enviormental factors, common genetic factors have been largly ignored.
Similar, the causal connection between alcohol consumption and risk taking is well established~\cite{Lane2004}, but specific genetic patways remain unknown.
An exception is the study by~\cite{Kogan2010} who showed an interaction between 5-HTTLPR and alcohol consumption which affected the onset of sexual behavior. 
However, the sample size is with 185 participants relativly low, thus it is difficult to draw a conclusion about a potential gene-enviorment interaction affecting risk taking behavior~\cite{Rubens2016}.

Smoking, on the other hand, has been commonly associated with poor impulse control.



Indeed, studies have investigated the genetic correlations among subtypes of aggressive behavior~\cite{Tuvblad2011a} as well as risk taking and age of first sexual intercourse~\cite{Day2016} but the genetic correlations between impulsive aggression, risk taking and other impulse control behaviors, such as drinking and smoking, are unknown.
A better understanding of these genetic interrelationships could provide valuable inside into shared molecular pathways of these complex human behaviors.

In conclusion, while a number of genome wide association studies have been conducted, only one genome wide association signal has been found in risk taking.
Further, it remains unclear to which extend risk taking and impulsive aggression overlapp, both phenotypically and geneticly as well as with other behavioral phenotypes.
Within this study I will make use of the UK BioBank and aim to identify specific genetic loci associated with aggression and risk taking.
I selected 7 phenotypes of inteterest in connection to impulsive aggression and risk taking.
Next to alcohol consumption and smoking behavior I also investigated the relationship among Neuroticism, Happiness, and IQ.
