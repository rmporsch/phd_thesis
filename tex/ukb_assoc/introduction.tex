\section{Introduction}
\label{sec:introduction}

As I have already established in Chapter~\ref{chap:longHera}, aggressive behavior is influenced by genetic factors to a considerable degree.
However, twin studies are unable to identidy specific genetic loci associated with this phenotype of interest.
This is the aim of this study, where I aim to identify genetic susceptible loci associated with impulsive aggressive behavior as well as risk taking.
Further, this chaper will estiamte SNP heritability and genetic correlations across a variety of different phenotypes.
Hence a second aim of this study is to explore the genetic interrelationship between aggressive behavior and other phenotypes.
Specifically, between impulsive aggressive behavior and risk taking.

As outlined in Section~\ref{sec:evolutionary_theories}, aggressive behavior is associated with a significant risk for ones own well-being.
An aggressive action could result in bodily harm, death and reduced survivial and reproductive fitness.
Thus it is not surprising that risk taking and aggression have been shown to be closly associated.
Indeed, a study of rural middle school children found that higher level of risk taking was  predictive for general acceptance of aggression as well as aggressive behavior in general~\cite{Swain2004}.
This close relationship has also been shown in drivers~\cite{Deffenbacher2003} as well as in collage students and other adults~\cite{Zuckerman2000}.
\citet{Zuckerman2000} suggested that personaltiy charactieriscs assocated with risk taking, namly impuslive sensation seeking, aggression and sociability, are part of a evolutionary based sensation seeking behavior which facilitated risky behavior.
Thus supporting the exploration of new territories, and foraging for food and mates.
This behavior can result in maladaptive behavior within our modern times, such as reckless driving or drug abuse, as well as aggression.
There have been a number of studies aimed to identify specific genetic loci associated with aggression and risk taking in GWAS\@.

While recent association studies in aggressive behavior have yield no significant findings, very recent large scale genome wide studies were able to identify specific genetic markers for risk taking.
These studies, are based on the same data here presented and were only published shortly after or during the analysis of this study.

Most genome wide associaton studies of aggresssive behavior have little statistical power and were unable to identfy genome wide associated loci~\cite{Fernandez-Castillo2016}
In one of the most comprehensive meta-analysis of aggressive behavior~\citet{Vassos2014} no genome wide significant signal was identified.
However, studies included in this meta-analysis showed a considerable degree of heterogeneity in the phenotype.
Specifically, the study included psychiatric as well as population based samples.
Hence making it difficult to identify loci specific signals.
Ideed, in their review of association studies in aggressive behavior,~\citet{Fernandez-Castillo2016} found that most hypothesis free approaches lacked sample size, assessed often very heterogenous phenotypes and did not differentiated between direct and indirect aggression. 
Thus explaining the lack of genome wide associations in aggressive behavior.

In contrast to aggression, a very recent study was able to identify a significant genome wide assocated signal in risk taking~\cite{Day2016}. 
In this analysis of the UK BioBank the authors identified a genome wide association in \textit{CADM2}.
The study demosntrated consinderable genetic correaltion ($0.49$) between risk taking and age at first sexual intercourse.
These results were also confirmed in a seperate study on the same dataset~\cite{Boutwell2017}.
This additional study also showed that \textit{CADM2} is associated with a number of other behavioral phenotypes as well.
However, no analysis has been done in considerdation of the genetic relatioship between risk taking and impulsive aggression as well as other behavioral phenotypes.

In conclusion, while a number of genome wide association studies have been conducted, only one genome wide association signal has been found in risk taking.
Further, it remains unclear to which extend risk taking and impulsive aggression overlapp, both phenotypically and geneticly.
Within this study I will make use of the UK BioBank and aim to identify specific genetic loci associated with aggression and risk taking.
I will further show genetic and phenotypical overlap between the two phenotypes as well as with others.
