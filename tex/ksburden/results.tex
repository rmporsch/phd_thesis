\section{Results}
\label{sec:results}

Across our simulations we have choosen a liability threshold $q$ of 1\% while we increased heritability from 0.2\% to 1\%.
Simulation of KS, KSburden and burden was done in c++ while for we used the SKAT R-package for SKAT and SKAT-O. %TODO add citation

\subsection{Type-I Error Rates}
\label{sub:type_i_error_rates}

Type-I error rate was accessed under the null for all tests and results are displayed in Table~1. 
%TODO make type 1 error rate

\subsection{Power Comparisons}
\label{sub:power_comparisons}

Statistical power of each scenario was evaluated with $10\times 100$ replications.
Figure~\ref{fig:simulatedGeneRealData} displays the empirical power for each test under various scenarions.
In each scenario we continously expanded a single causal region by 10\% of the total genes variants until the complete gene was covered by causal mutatations.
We compared the power of the SKAT, SKAT-O, KS, burden (CMC, BURDEN), KSburden tests.
As expected the KS test looses dramatically statistical power in situations if all variants in a given gene are causally related to the phenotype.
In contrast if only a small fraction of the gene is causally relavant the KS tests outperforms both burden tests as well as SKAT.
Further, the combined KSburden is less powerful that SKAT-O when 100\% of rare variants are causal but is able to outperform SKAT-O when only 50\% of variants are causal.

\begin{figure}[ht!]
	\centering
	\includegraphics[width=1.0\linewidth]{example-image-a}
	\caption{Estimated statistical power for four different causal cluster size.\label{fig:simulatedGeneRealData}}
\end{figure}

We further simulated $10000$ p-values under the null for both burden and KS tests (Figure~\ref{fig:nullCor})
As one can see we have strong evidence that the two tests are not dependent and in fact independent of each other.
While this simulation cannot exclude the posibility that the two test statistics are independent it indicates that under most common situations the two test statistics can be considered independent.

\begin{figure}[ht!]
	\centering
	\includegraphics[width=0.8\linewidth]{example-image-a}
	\caption{\label{fig:nullCor} Correlation between KSsum and Burden under the null}
\end{figure}

\subsection{Application to the DDD Hong Kong Data}
\label{sub:application_to_the_ddd_hong_kong_data}

In addtion to our simulation study we also applied our test on the DDD dataset.
Figure~\ref{fig:ddd} shows the qq-plot for all applied tests.
Unsurspringsly we were unable to find any significant associations, persumably due to the lack of statistical power in all tests.
Nevertheless, we were able to identify two genes which approched genome wide significants.
Namly XXX and XXXX (see table~%todo add table
Further we would like to note that the correlation between the KS and burden test is wtih $somenumber$ close to $0$, thus indicating that both tests are independent in probable scenarios.

\begin{figure}[ht!]
	\centering
	\includegraphics[width=0.8\linewidth]{example-image-a}
	\caption{\label{fig:ddd} QQ-plot for DDD}
\end{figure}
