\section{Results}
\label{sec:results}

Across our simulations we have choosen a liability threshold $q$ of 1\% while we increased heritability from 0.1\% to 1\%.
Simulation of KS, KSburden and burden was done in c++ while for we used the SKAT R-package for SKAT and SKAT-O. %TODO add citation

\subsection{Type-I Error Rates}
\label{sub:type_i_error_rates}

Type-I error rate was accessed under the null for all tests and results are displayed in Table~1. 
%TODO make type 1 error rate

\subsection{Relationship between KS and Burden}
\label{sub:relationship_between_ks_and_burden}

\begin{figure}[ht!]
  \centering
  \includegraphics[width=0.8\linewidth]{example-image-a}
  \caption{Correlation between test statistics of KS and Burden under the Null acorss selected genes. No obvious correlation pattern can be detected.}\label{fig:correlation_ks_burden}
\end{figure}

\subsection{Power Comparisons}
\label{sub:power_comparisons}

Statistical power of each scenario was evaluated with $10\times 100$ replications.
Figure~\ref{fig:simulatedGeneRealData} displays the empirical power for each test under various scenarions.
In each scenario we continously expanded a single causal region by 10\% of the total genes variants until the complete gene was covered by causal mutatations.
We compared the power of the SKAT, SKAT-O, KS, burden (CMC, BURDEN), KSburden tests.
As expected the KS test looses dramatically statistical power in situations if all variants in a given gene are causally related to the phenotype.
In contrast if only a small fraction of the gene is causally relavant the KS tests outperforms both burden tests as well as SKAT\@.
Further, the combined KSburden is less powerful that SKAT-O when 100\% of rare variants are causal but is able to outperform SKAT-O when only 50\% of variants are causal.

\begin{figure}[ht!]
	\centering
	\includegraphics[width=1.0\linewidth]{example-image-a}
	\caption{Estimated statistical power for four different causal cluster size.\label{fig:simulatedGeneRealData}}
\end{figure}

We further simulated $10000$ p-values under the null for both burden and KS tests (Figure~\ref{fig:nullCor})
As one can see we have strong evidence that the two tests are not dependent and in fact independent of each other.
While this simulation cannot exclude the posibility that the two test statistics are independent it indicates that under most common situations the two test statistics can be considered independent.

\begin{figure}[ht!]
	\centering
	\includegraphics[width=0.8\linewidth]{example-image-a}
	\caption{\label{fig:nullCor} Correlation between KSsum and Burden under the null}
\end{figure}
