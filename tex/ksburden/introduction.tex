\section{Introduction}
\label{sec:introduction}

The development of sequencing based technologies and their drop in costs have provided rich opportunities to study the relationship between genetic variants and complex human traits\cite{Goodwin2016}.
Especially rare variants, which can be defined as genetic variations occurring in less than 1\% of the population, have been suggested to play an important role in the etiology of human traits and potentially account for the missing heritability\cite{Jiang2013,Li2009a}. 

Thus considerable effort has been made to develop and deploy statistical methods to discover important causal relationships between rare variants and complex traits\cite{Morris2010,Zeng2014,Daye2012,Manuscript2013}.
In Genome wide association studies (GWAS) single variants are associated with the trait in question.
This approach is largely unfeasible in rare variants due to their low frequency\cite{Lee2014}.
Thus most approaches have been focused on combining multiple rare variants in order to increase statistical power.
This can be either done on the gene or pathway level, but for simplicity we will only consider gene based tests within this paper.

In general one can classify gene based tests into two larger categories, namely burden and variance-component tests\cite{Lee2014}.
These separate methods have differing assumption about the underlying genetic architecture.
Thus effort has been made to combine these two methods into an omnibus test.

In general, burden tests aggregate single rare genetic variations.
Thus assuming that all variants in a given genomic region have the same direction of effect.
Violation of this assumption results in considerable reduction in statistical power.
Examples of this methods include the Combined Multivariate and Collapsing (CMC) test\cite{Li2008}, as well as the weighted sum statistic\cite{Madsen2009}.
Alternatively, variance-component tests do not assume uni-directional effect of all included variants. 
These methods investigate the distribution of genetic effects for a genomic region and are robust to variants with differing direction of effects.
Prominent example of variance-component tests are SKAT\cite{Wu2011} and C-Alpha\cite{Neale2011}.
These tests are more powerful compared to the burden approach in cases when the majority of rare variants are of neutral effect or bi-directional effects are present.
However, the burden tests in general outperform variance-component based tests when a large proportion of variants have the same direction of effect.
This has lead to the development of omnibus tests to combine both approaches.
An example if these omnibus tests is SKAT-O\cite{Lee2012a} which uses a linear combination of SKAT and burden tests statistics to derive a combined p-value.
Alternative methods include the use of Fischer's methods to combine p-values from burden and SKAT\cite{Derkach2013a}.
These omnibus tests, although less powerful if one of the assumptions of burden or variance-component tests hold true, have demonstrated robust power\cite{Lee2014}. 

An often neglected aspect of rare variant tests are the position of these genetic variation.
Multiple biological evidence has been reported in the past demonstrating clustering of causal rare variants within the genome\cite{Ionita-Laza2012}.%TODO needed citation [more than 2] 
It is biological plausible to suggest that rare deleterious mutations causally related to a considered trait might be more likely to be located in protein functional domains or gene regulatory elements. %TODO ciation needed here
A number of different tests have been proposed in order to take the location of rare variants into consideration when performing association tests.
Examples of spatial approaches of rare variant association tests are IL-K\cite{Ionita-Laza2012}, KERNEL\cite{Schaid2013} and CLUSTER\cite{Lin2014}.
These methods relay on kernel distance clustering and are therefore computational intensive. %TODO is it computational intensive???

We are here proposing a way to assess the differences in spatial organization by assessing the distributional differences of rare variants between cases and controls.
To do so we make use of the well known Kolmogorov-Smirnov (KS) test.
We demonstrate, through extensive simulation, that our methods outperforms commonly used tests, such as SKAT and burden when the assumption of the KS-test are true.
Further, we combine the KS and burden test to provide an omnibus approach to gene based tests.
